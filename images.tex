\batchmode
\documentclass{book}
\RequirePackage{ifthen}




\usepackage[paperheight=8.5in,paperwidth=6in,margin=1in,heightrounded]{geometry}
\usepackage{verbatim}
\usepackage{fancyvrb}
\usepackage{hyperref}
\hypersetup{colorlinks=true}
\usepackage{glossaries}
\usepackage{endnotes}
\usepackage{parskip}
\usepackage{multicol}
\usepackage{csquotes}%
\providecommand{\squeezeup}{\vspace{-3cm}}     
\usepackage{makeidx}
\makeindex
\usepackage{graphicx}
\usepackage{listings}
\usepackage{xcolor}
\usepackage{xparse}
\usepackage{lineno}
\usepackage{graphics}


\definecolor{codegreen}{rgb}{0,0.6,0}
\definecolor{codegray}{rgb}{0.5,0.5,0.5}
\definecolor{codepurple}{rgb}{0.58,0,0.82}
\definecolor{backcolour}{rgb}{0.95,0.95,0.92}


\lstdefinestyle{mystyle}{
    language=Java,
    backgroundcolor=\color{backcolour},   
    commentstyle=\color{codegreen},
    keywordstyle=\color{magenta},
    numberstyle=\tiny\color{codegray},
    stringstyle=\color{codepurple},
    basicstyle=\ttfamily\footnotesize ,
    breakatwhitespace=false,         
    breaklines=true,                 
    captionpos=b,                    
    keepspaces=true,                 
    numbers=left,                    
    numbersep=5pt,                  
    showspaces=false,                
    showstringspaces=false,
    showtabs=false,                  
    tabsize=2
}


\lstset{style=mystyle}


\makeglossaries


\title{{\small {A CONCISE MANUAL ON}} \\JAVA PROGRAMMING}  


\author{Kamakshaiah Musunuru}
\date{}




\usepackage[]{inputenc}



\makeatletter

\makeatletter
\count@=\the\catcode`\_ \catcode`\_=8 
\newenvironment{tex2html_wrap}{}{}%
\catcode`\<=12\catcode`\_=\count@
\newcommand{\providedcommand}[1]{\expandafter\providecommand\csname #1\endcsname}%
\newcommand{\renewedcommand}[1]{\expandafter\providecommand\csname #1\endcsname{}%
  \expandafter\renewcommand\csname #1\endcsname}%
\newcommand{\newedenvironment}[1]{\newenvironment{#1}{}{}\renewenvironment{#1}}%
\let\newedcommand\renewedcommand
\let\renewedenvironment\newedenvironment
\makeatother
\let\mathon=$
\let\mathoff=$
\ifx\AtBeginDocument\undefined \newcommand{\AtBeginDocument}[1]{}\fi
\newbox\sizebox
\setlength{\hoffset}{0pt}\setlength{\voffset}{0pt}
\addtolength{\textheight}{\footskip}\setlength{\footskip}{0pt}
\addtolength{\textheight}{\topmargin}\setlength{\topmargin}{0pt}
\addtolength{\textheight}{\headheight}\setlength{\headheight}{0pt}
\addtolength{\textheight}{\headsep}\setlength{\headsep}{0pt}
\setlength{\textwidth}{349pt}
\newwrite\lthtmlwrite
\makeatletter
\let\realnormalsize=\normalsize
\global\topskip=2sp
\def\preveqno{}\let\real@float=\@float \let\realend@float=\end@float
\def\@float{\let\@savefreelist\@freelist\real@float}
\def\liih@math{\ifmmode$\else\bad@math\fi}
\def\end@float{\realend@float\global\let\@freelist\@savefreelist}
\let\real@dbflt=\@dbflt \let\end@dblfloat=\end@float
\let\@largefloatcheck=\relax
\let\if@boxedmulticols=\iftrue
\def\@dbflt{\let\@savefreelist\@freelist\real@dbflt}
\def\adjustnormalsize{\def\normalsize{\mathsurround=0pt \realnormalsize
 \parindent=0pt\abovedisplayskip=0pt\belowdisplayskip=0pt}%
 \def\phantompar{\csname par\endcsname}\normalsize}%
\def\lthtmltypeout#1{{\let\protect\string \immediate\write\lthtmlwrite{#1}}}%
\usepackage[tightpage,active]{preview}
\newbox\lthtmlPageBox
\newdimen\lthtmlCropMarkHeight
\newdimen\lthtmlCropMarkDepth
\long\def\lthtmlTightVBox#1#2{%
    \setbox\lthtmlPageBox\vbox{\hbox{\catcode`\_=8 #2}}%
    \lthtmlCropMarkHeight=\ht\lthtmlPageBox \advance \lthtmlCropMarkHeight 6pt
    \lthtmlCropMarkDepth=\dp\lthtmlPageBox
    \lthtmltypeout{^^J:#1:lthtmlCropMarkHeight:=\the\lthtmlCropMarkHeight}%
    \lthtmltypeout{^^J:#1:lthtmlCropMarkDepth:=\the\lthtmlCropMarkDepth:1ex:=\the \dimexpr 1ex}%
    \begin{preview}\copy\lthtmlPageBox\end{preview}}%
\long\def\lthtmlTightFBox#1#2{%
    \adjustnormalsize\setbox\lthtmlPageBox=\vbox\bgroup %
    \let\ifinner=\iffalse \let\)\liih@math %
    {\catcode`\_=8 #2}%
    \@next\next\@currlist{}{\def\next{\voidb@x}}%
    \expandafter\box\next\egroup %
    \lthtmlCropMarkHeight=\ht\lthtmlPageBox \advance \lthtmlCropMarkHeight 6pt
    \lthtmlCropMarkDepth=\dp\lthtmlPageBox
    \lthtmltypeout{^^J:#1:lthtmlCropMarkHeight:=\the\lthtmlCropMarkHeight}%
    \lthtmltypeout{^^J:#1:lthtmlCropMarkDepth:=\the\lthtmlCropMarkDepth:1ex:=\the \dimexpr 1ex}%
    \begin{preview}\copy\lthtmlPageBox\end{preview}}%
    \long\def\lthtmlinlinemathA#1#2\lthtmlindisplaymathZ{\lthtmlTightVBox{#1}{#2}}
    \def\lthtmlinlineA#1#2\lthtmlinlineZ{\lthtmlTightVBox{#1}{#2}}
    \long\def\lthtmldisplayA#1#2\lthtmldisplayZ{\lthtmlTightVBox{#1}{#2}}
    \long\def\lthtmlinlinemathA#1#2\lthtmlindisplaymathZ{\lthtmlTightVBox{#1}{#2}}
    \def\lthtmlinlineA#1#2\lthtmlinlineZ{\lthtmlTightVBox{#1}{#2}}
    \long\def\lthtmldisplayA#1#2\lthtmldisplayZ{\lthtmlTightVBox{#1}{#2}}
    \long\def\lthtmldisplayB#1#2\lthtmldisplayZ{\\edef\preveqno{(\theequation)}%
        \lthtmlTightVBox{#1}{\let\@eqnnum\relax#2}}
    \long\def\lthtmlfigureA#1#2\lthtmlfigureZ{\let\@savefreelist\@freelist
        \lthtmlTightFBox{#1}{#2}\global\let\@freelist\@savefreelist}
    \long\def\lthtmlpictureA#1#2\lthtmlpictureZ{\let\@savefreelist\@freelist
        \lthtmlTightVBox{#1}{#2}\global\let\@freelist\@savefreelist}
\def\lthtmlcheckvsize{\ifdim\ht\sizebox<\vsize 
  \ifdim\wd\sizebox<\hsize\expandafter\hfill\fi \expandafter\vfill
  \else\expandafter\vss\fi}%
\providecommand{\selectlanguage}[1]{}%
\makeatletter \tracingstats = 1 


\begin{document}
\pagestyle{empty}\thispagestyle{empty}\lthtmltypeout{}%
\lthtmltypeout{latex2htmlLength hsize=\the\hsize}\lthtmltypeout{}%
\lthtmltypeout{latex2htmlLength vsize=\the\vsize}\lthtmltypeout{}%
\lthtmltypeout{latex2htmlLength hoffset=\the\hoffset}\lthtmltypeout{}%
\lthtmltypeout{latex2htmlLength voffset=\the\voffset}\lthtmltypeout{}%
\lthtmltypeout{latex2htmlLength topmargin=\the\topmargin}\lthtmltypeout{}%
\lthtmltypeout{latex2htmlLength topskip=\the\topskip}\lthtmltypeout{}%
\lthtmltypeout{latex2htmlLength headheight=\the\headheight}\lthtmltypeout{}%
\lthtmltypeout{latex2htmlLength headsep=\the\headsep}\lthtmltypeout{}%
\lthtmltypeout{latex2htmlLength parskip=\the\parskip}\lthtmltypeout{}%
\lthtmltypeout{latex2htmlLength oddsidemargin=\the\oddsidemargin}\lthtmltypeout{}%
\makeatletter
\if@twoside\lthtmltypeout{latex2htmlLength evensidemargin=\the\evensidemargin}%
\else\lthtmltypeout{latex2htmlLength evensidemargin=\the\oddsidemargin}\fi%
\lthtmltypeout{}%
\makeatother
\setcounter{page}{1}
\onecolumn

% !!! IMAGES START HERE !!!

\stepcounter{chapter}
\stepcounter{section}
\stepcounter{section}
\stepcounter{section}
\stepcounter{section}
\stepcounter{section}
\stepcounter{subsection}
\stepcounter{subsection}
\stepcounter{subsection}
\stepcounter{subsection}
\stepcounter{section}
{\newpage\clearpage
\lthtmlinlinemathA{tex2html_wrap_inline4325}%
$\lstinline{printf}$%
\lthtmlindisplaymathZ
\lthtmlcheckvsize\clearpage}

{\newpage\clearpage
\lthtmlinlinemathA{tex2html_wrap_inline4327}%
$\lstinline{(//)}$%
\lthtmlindisplaymathZ
\lthtmlcheckvsize\clearpage}

{\newpage\clearpage
\lthtmlinlinemathA{tex2html_wrap_inline4329}%
$\lstinline{/*}$%
\lthtmlindisplaymathZ
\lthtmlcheckvsize\clearpage}

{\newpage\clearpage
\lthtmlinlinemathA{tex2html_wrap_inline4331}%
$\lstinline{*/}$%
\lthtmlindisplaymathZ
\lthtmlcheckvsize\clearpage}

{\newpage\clearpage
\lthtmlinlinemathA{tex2html_wrap_inline4333}%
$\lstinline{/**}$%
\lthtmlindisplaymathZ
\lthtmlcheckvsize\clearpage}

\stepcounter{subsection}
{\newpage\clearpage
\lthtmlfigureA{lstlisting83}%
\begin{lstlisting}
public class HelloWorldApp {
    public static void main(String[] args) {
        System.out.println("Hello World!"); // Prints the string to the console.
    }
}
\end{lstlisting}%
\lthtmlfigureZ
\lthtmlcheckvsize\clearpage}

{\newpage\clearpage
\lthtmlinlinemathA{tex2html_wrap_inline4337}%
$\lstinline{.java}$%
\lthtmlindisplaymathZ
\lthtmlcheckvsize\clearpage}

{\newpage\clearpage
\lthtmlinlinemathA{tex2html_wrap_inline4339}%
$\lstinline{HelloWorldApp.java}$%
\lthtmlindisplaymathZ
\lthtmlcheckvsize\clearpage}

{\newpage\clearpage
\lthtmlinlinemathA{tex2html_wrap_inline4341}%
$\lstinline{.class}$%
\lthtmlindisplaymathZ
\lthtmlcheckvsize\clearpage}

{\newpage\clearpage
\lthtmlinlinemathA{tex2html_wrap_inline4343}%
$\lstinline{HelloWorldApp.class}$%
\lthtmlindisplaymathZ
\lthtmlcheckvsize\clearpage}

{\newpage\clearpage
\lthtmlinlinemathA{tex2html_wrap_inline4345}%
$\lstinline{public}$%
\lthtmlindisplaymathZ
\lthtmlcheckvsize\clearpage}

{\newpage\clearpage
\lthtmlinlinemathA{tex2html_wrap_inline4355}%
$\lstinline{private}$%
\lthtmlindisplaymathZ
\lthtmlcheckvsize\clearpage}

{\newpage\clearpage
\lthtmlinlinemathA{tex2html_wrap_inline4357}%
$\lstinline{protected}$%
\lthtmlindisplaymathZ
\lthtmlcheckvsize\clearpage}

{\newpage\clearpage
\lthtmlinlinemathA{tex2html_wrap_inline4359}%
$\lstinline{SecurityException}$%
\lthtmlindisplaymathZ
\lthtmlcheckvsize\clearpage}

{\newpage\clearpage
\lthtmlinlinemathA{tex2html_wrap_inline4361}%
$\lstinline{static}$%
\lthtmlindisplaymathZ
\lthtmlcheckvsize\clearpage}

{\newpage\clearpage
\lthtmlinlinemathA{tex2html_wrap_inline4363}%
$\lstinline{void}$%
\lthtmlindisplaymathZ
\lthtmlcheckvsize\clearpage}

{\newpage\clearpage
\lthtmlinlinemathA{tex2html_wrap_inline4365}%
$\lstinline{System.exit()}$%
\lthtmlindisplaymathZ
\lthtmlcheckvsize\clearpage}

{\newpage\clearpage
\lthtmlinlinemathA{tex2html_wrap_inline4367}%
$\lstinline{String}$%
\lthtmlindisplaymathZ
\lthtmlcheckvsize\clearpage}

{\newpage\clearpage
\lthtmlinlinemathA{tex2html_wrap_inline4369}%
$\lstinline{args}$%
\lthtmlindisplaymathZ
\lthtmlcheckvsize\clearpage}

{\newpage\clearpage
\lthtmlinlinemathA{tex2html_wrap_inline4371}%
$\lstinline{public static void main(String... args)}$%
\lthtmlindisplaymathZ
\lthtmlcheckvsize\clearpage}

{\newpage\clearpage
\lthtmlinlinemathA{tex2html_wrap_inline4377}%
$\lstinline{public static void main(String[])}$%
\lthtmlindisplaymathZ
\lthtmlcheckvsize\clearpage}

{\newpage\clearpage
\lthtmlinlinemathA{tex2html_wrap_inline4379}%
$\lstinline{String[] args}$%
\lthtmlindisplaymathZ
\lthtmlcheckvsize\clearpage}

{\newpage\clearpage
\lthtmlinlinemathA{tex2html_wrap_inline4383}%
$\lstinline{main}$%
\lthtmlindisplaymathZ
\lthtmlcheckvsize\clearpage}

{\newpage\clearpage
\lthtmlinlinemathA{tex2html_wrap_inline4385}%
$\lstinline{System}$%
\lthtmlindisplaymathZ
\lthtmlcheckvsize\clearpage}

{\newpage\clearpage
\lthtmlinlinemathA{tex2html_wrap_inline4387}%
$\lstinline{out}$%
\lthtmlindisplaymathZ
\lthtmlcheckvsize\clearpage}

{\newpage\clearpage
\lthtmlinlinemathA{tex2html_wrap_inline4389}%
$\lstinline{PrintStream}$%
\lthtmlindisplaymathZ
\lthtmlcheckvsize\clearpage}

{\newpage\clearpage
\lthtmlinlinemathA{tex2html_wrap_inline4391}%
$\lstinline{println(String)}$%
\lthtmlindisplaymathZ
\lthtmlcheckvsize\clearpage}

\stepcounter{subsection}
{\newpage\clearpage
\lthtmlfigureA{lstlisting116}%
\begin{lstlisting}
// This is an example of a single line comment using two slashes
\par
/*
 * This is an example of a multiple line comment using the slash and asterisk.
 * This type of comment can be used to hold a lot of information or deactivate
 * code, but it is very important to remember to close the comment.
 */
\par
package fibsandlies;
\par
import java.util.Map;
import java.util.HashMap;
\par
/**
 * This is an example of a Javadoc comment; Javadoc can compile documentation
 * from this text. Javadoc comments must immediately precede the class, method,
 * or field being documented.
 * @author Wikipedia Volunteers
 */
public class FibCalculator extends Fibonacci implements Calculator {
    private static Map<Integer, Integer> memoized = new HashMap<>();
\par
/*
     * The main method written as follows is used by the JVM as a starting point
     * for the program.
     */
    public static void main(String[] args) {
        memoized.put(1, 1);
        memoized.put(2, 1);
        System.out.println(fibonacci(12)); // Get the 12th Fibonacci number and print to console
    }
\par
/**
     * An example of a method written in Java, wrapped in a class.
     * Given a non-negative number FIBINDEX, returns
     * the Nth Fibonacci number, where N equals FIBINDEX.
     * 
     * @param fibIndex The index of the Fibonacci number
     * @return the Fibonacci number
     */
    public static int fibonacci(int fibIndex) {
        if (memoized.containsKey(fibIndex)) return memoized.get(fibIndex);
        else {
            int answer = fibonacci(fibIndex - 1) + fibonacci(fibIndex - 2);
            memoized.put(fibIndex, answer);
            return answer;
        }
    }
}
\par
\end{lstlisting}%
\lthtmlfigureZ
\lthtmlcheckvsize\clearpage}

\stepcounter{section}
\stepcounter{subsection}
\stepcounter{subsubsection}
\stepcounter{subsection}
\stepcounter{subsubsection}
{\newpage\clearpage
\lthtmlinlinemathA{tex2html_wrap_inline4393}%
$\lstinline{service()}$%
\lthtmlindisplaymathZ
\lthtmlcheckvsize\clearpage}

{\newpage\clearpage
\lthtmlinlinemathA{tex2html_wrap_inline4395}%
$\lstinline{HttpServlet}$%
\lthtmlindisplaymathZ
\lthtmlcheckvsize\clearpage}

{\newpage\clearpage
\lthtmlinlinemathA{tex2html_wrap_inline4397}%
$\lstinline{GenericServlet}$%
\lthtmlindisplaymathZ
\lthtmlcheckvsize\clearpage}

{\newpage\clearpage
\lthtmlinlinemathA{tex2html_wrap_inline4403}%
$\lstinline{doGet(), doPost(), doPut(), doDelete()}$%
\lthtmlindisplaymathZ
\lthtmlcheckvsize\clearpage}

{\newpage\clearpage
\lthtmlfigureA{lstlisting138}%
\begin{lstlisting}
import java.io.IOException;
\par
import javax.servlet.ServletConfig;
import javax.servlet.ServletException;
import javax.servlet.http.HttpServlet;
import javax.servlet.http.HttpServletRequest;
import javax.servlet.http.HttpServletResponse;
\par
public class ServletLifeCycleExample extends HttpServlet {
\par
private Integer sharedCounter;                 
\par
@Override
    public void init(final ServletConfig config) throws ServletException {
        super.init(config);
        getServletContext().log("init() called");
        sharedCounter = 0;
    }
\par
@Override
    protected void service(final HttpServletRequest request, final HttpServletResponse response) throws ServletException, IOException {
        getServletContext().log("service() called");
\par
int localCounter;                       
\par
synchronized (sharedCounter) {
              sharedCounter++;                  
\par
localCounter = sharedCounter;       
        }
\par
response.getWriter().write("Incrementing the count to " + localCounter);  // accessing a local variable
    }
\par
@Override
    public void destroy() {
        getServletContext().log("destroy() called");
    }
}
\end{lstlisting}%
\lthtmlfigureZ
\lthtmlcheckvsize\clearpage}

\stepcounter{subsection}
{\newpage\clearpage
\lthtmlfigureA{lstlisting144}%
\begin{lstlisting}
<p>Counting to three:</p>
<% for (int i=1; i<4; i++) { %>
    <p>This number is <%= i %>.</p>
<% } %>
<p>OK.</p>
\end{lstlisting}%
\lthtmlfigureZ
\lthtmlcheckvsize\clearpage}

{\newpage\clearpage
\lthtmlfigureA{lstlisting146}%
\begin{lstlisting}
Counting to three:
\par
This number is 1.
\par
This number is 2.
\par
This number is 3.
\par
OK.
\end{lstlisting}%
\lthtmlfigureZ
\lthtmlcheckvsize\clearpage}

\stepcounter{subsection}
\stepcounter{subsubsection}
{\newpage\clearpage
\lthtmlfigureA{lstlisting151}%
\begin{lstlisting}
// Hello.java (Java SE 5)
import javax.swing.*;
\par
public class Hello extends JFrame {
    public Hello() {
        super("hello");
        this.setDefaultCloseOperation(WindowConstants.EXIT_ON_CLOSE);
        this.add(new JLabel("Hello, world!"));
        this.pack();
        this.setVisible(true);
    }
\par
public static void main(final String[] args) {
        new Hello();
    }
}
\end{lstlisting}%
\lthtmlfigureZ
\lthtmlcheckvsize\clearpage}

{\newpage\clearpage
\lthtmlinlinemathA{tex2html_wrap_inline4407}%
$\lstinline{import}$%
\lthtmlindisplaymathZ
\lthtmlcheckvsize\clearpage}

{\newpage\clearpage
\lthtmlinlinemathA{tex2html_wrap_inline4409}%
$\lstinline{javax.swing}$%
\lthtmlindisplaymathZ
\lthtmlcheckvsize\clearpage}

{\newpage\clearpage
\lthtmlinlinemathA{tex2html_wrap_inline4411}%
$\lstinline{Hello}$%
\lthtmlindisplaymathZ
\lthtmlcheckvsize\clearpage}

{\newpage\clearpage
\lthtmlinlinemathA{tex2html_wrap_inline4413}%
$\lstinline{extends}$%
\lthtmlindisplaymathZ
\lthtmlcheckvsize\clearpage}

{\newpage\clearpage
\lthtmlinlinemathA{tex2html_wrap_inline4415}%
$\lstinline{JFrame}$%
\lthtmlindisplaymathZ
\lthtmlcheckvsize\clearpage}

{\newpage\clearpage
\lthtmlinlinemathA{tex2html_wrap_inline4419}%
$\lstinline{Hello()}$%
\lthtmlindisplaymathZ
\lthtmlcheckvsize\clearpage}

{\newpage\clearpage
\lthtmlinlinemathA{tex2html_wrap_inline4421}%
$\lstinline{setDefaultCloseOperation(int)}$%
\lthtmlindisplaymathZ
\lthtmlcheckvsize\clearpage}

{\newpage\clearpage
\lthtmlinlinemathA{tex2html_wrap_inline4425}%
$\lstinline{WindowConstants.EXIT_ON_CLOSE}$%
\lthtmlindisplaymathZ
\lthtmlcheckvsize\clearpage}

{\newpage\clearpage
\lthtmlinlinemathA{tex2html_wrap_inline4429}%
$\lstinline{JLabel}$%
\lthtmlindisplaymathZ
\lthtmlcheckvsize\clearpage}

{\newpage\clearpage
\lthtmlinlinemathA{tex2html_wrap_inline4431}%
$\lstinline{add(Component)}$%
\lthtmlindisplaymathZ
\lthtmlcheckvsize\clearpage}

{\newpage\clearpage
\lthtmlinlinemathA{tex2html_wrap_inline4433}%
$\lstinline{Container}$%
\lthtmlindisplaymathZ
\lthtmlcheckvsize\clearpage}

{\newpage\clearpage
\lthtmlinlinemathA{tex2html_wrap_inline4435}%
$\lstinline{pack()}$%
\lthtmlindisplaymathZ
\lthtmlcheckvsize\clearpage}

{\newpage\clearpage
\lthtmlinlinemathA{tex2html_wrap_inline4437}%
$\lstinline{Window}$%
\lthtmlindisplaymathZ
\lthtmlcheckvsize\clearpage}

{\newpage\clearpage
\lthtmlinlinemathA{tex2html_wrap_inline4439}%
$\lstinline{main()}$%
\lthtmlindisplaymathZ
\lthtmlcheckvsize\clearpage}

{\newpage\clearpage
\lthtmlinlinemathA{tex2html_wrap_inline4441}%
$\lstinline{setVisible(boolean)}$%
\lthtmlindisplaymathZ
\lthtmlcheckvsize\clearpage}

{\newpage\clearpage
\lthtmlinlinemathA{tex2html_wrap_inline4443}%
$\lstinline{Component}$%
\lthtmlindisplaymathZ
\lthtmlcheckvsize\clearpage}

{\newpage\clearpage
\lthtmlinlinemathA{tex2html_wrap_inline4445}%
$\lstinline{true}$%
\lthtmlindisplaymathZ
\lthtmlcheckvsize\clearpage}

{\newpage\clearpage
\lthtmlfigureA{lstlisting182}%
\begin{lstlisting}
import java.awt.FlowLayout;
import javax.swing.JButton;
import javax.swing.JFrame;
import javax.swing.JLabel;
import javax.swing.WindowConstants;
import javax.swing.SwingUtilities;
\par
public class SwingExample implements Runnable {
\par
@Override
    public void run() {
        // Create the window
        JFrame f = new JFrame("Hello, !");
        // Sets the behavior for when the window is closed
        f.setDefaultCloseOperation(WindowConstants.EXIT_ON_CLOSE);
        // Add a layout manager so that the button is not placed on top of the label
        f.setLayout(new FlowLayout());
        // Add a label and a button
        f.add(new JLabel("Hello, world!"));
        f.add(new JButton("Press me!"));
        // Arrange the components inside the window
        f.pack();
        // By default, the window is not visible. Make it visible.
        f.setVisible(true);
    }
\par
public static void main(String[] args) {
        SwingExample se = new SwingExample();
        // Schedules the application to be run at the correct time in the event queue.
        SwingUtilities.invokeLater(se);
    }
\par
}
\end{lstlisting}%
\lthtmlfigureZ
\lthtmlcheckvsize\clearpage}

{\newpage\clearpage
\lthtmlinlinemathA{tex2html_wrap_inline4447}%
$\lstinline{SwingUtilities.invokeLater(Runnable))}$%
\lthtmlindisplaymathZ
\lthtmlcheckvsize\clearpage}

\stepcounter{subsection}
\stepcounter{subsubsection}
\stepcounter{subsubsection}
{\newpage\clearpage
\lthtmlfigureA{lstlisting194}%
\begin{lstlisting}
package javafxtuts;
\par
import javafx.application.Application;
import javafx.event.ActionEvent;
import javafx.event.EventHandler;
import javafx.scene.Scene;
import javafx.scene.control.Button;
import javafx.scene.layout.StackPane;
import javafx.stage.Stage;
\par
public class JavaFxTuts extends Application {
    public JavaFxTuts() {
        //Optional constructor
    }
    @Override
    public void init() {
         //By default this does nothing, but it
         //can carry out code to set up your app.
         //It runs once before the start method,
         //and after the constructor.
    }
\par
@Override
    public void start(Stage primaryStage) {
        // Creating the Java button
        final Button button = new Button();
        // Setting text to button
        button.setText("Hello World");
        // Registering a handler for button
        button.setOnAction((ActionEvent event) -> {
            // Printing Hello World! to the console
            System.out.println("Hello World!");
        });
        // Initializing the StackPane class
        final StackPane root = new StackPane();
        // Adding all the nodes to the StackPane
        root.getChildren().add(button);
        // Creating a scene object
        final Scene scene = new Scene(root, 300, 250);
        // Adding the title to the window (primaryStage)
        primaryStage.setTitle("Hello World!");
        primaryStage.setScene(scene);
        // Show the window(primaryStage)
        primaryStage.show();
    }
    @Override
    public void stop() {
        //By default this does nothing
        //It runs if the user clicks the go-away button
        //closing the window or if Platform.exit() is called.
        //Use Platform.exit() instead of System.exit(0).
        //This is where you should offer to save any unsaved
        //stuff that the user may have generated.
    }
\par
/**
     * Main function that opens the "Hello World!" window
     * 
     * @param arguments the command line arguments
     */
    public static void main(final String[] arguments) {
        launch(arguments);
    }
}
\end{lstlisting}%
\lthtmlfigureZ
\lthtmlcheckvsize\clearpage}

\stepcounter{section}
\stepcounter{chapter}
\stepcounter{section}
{\newpage\clearpage
\lthtmlfigureA{lstlisting224}%
\begin{lstlisting}
// Your First Program
\par
class HelloWorld {
    public static void main(String[] args) {
        System.out.println("Hello, World!"); 
    }
}
\end{lstlisting}%
\lthtmlfigureZ
\lthtmlcheckvsize\clearpage}

{\newpage\clearpage
\lthtmlfigureA{lstlisting230}%
\begin{lstlisting}
javac HelloWorld.java
java HelloWorld
\end{lstlisting}%
\lthtmlfigureZ
\lthtmlcheckvsize\clearpage}

{\newpage\clearpage
\lthtmlinlinemathA{tex2html_wrap_inline4449}%
$\lstinline{java HelloWorld}$%
\lthtmlindisplaymathZ
\lthtmlcheckvsize\clearpage}

\stepcounter{subsection}
{\newpage\clearpage
\lthtmlfigureA{lstlisting242}%
\begin{lstlisting}
class HelloWorld {
... .. ...
}
\end{lstlisting}%
\lthtmlfigureZ
\lthtmlcheckvsize\clearpage}

{\newpage\clearpage
\lthtmlfigureA{lstlisting248}%
\begin{lstlisting}
public static void main(String[] args) {
... .. ...
}
\end{lstlisting}%
\lthtmlfigureZ
\lthtmlcheckvsize\clearpage}

\stepcounter{section}
\stepcounter{section}
\stepcounter{subsection}
\stepcounter{subsubsection}
{\newpage\clearpage
\lthtmlfigureA{lstlisting313}%
\begin{lstlisting}
E:\work\java\eclipse\HelloWorld\src>dir
Volume in drive E has no label .
Volume Serial Number is #### -####
\par
Directory of E:\work\java\eclipse\HelloWorld\src
6 .
7 ..
179 helloworld . java
	p1
	p2
\par
1 File (s) 179 bytes
	4 Dir ( s) 85 ,037 ,547 ,520 bytes free
\end{lstlisting}%
\lthtmlfigureZ
\lthtmlcheckvsize\clearpage}

{\newpage\clearpage
\lthtmlfigureA{lstlisting326}%
\begin{lstlisting}
// package p1
package p1 ;
\par
public class A {
public void display () {
System.out.println ("Class A is called here !");
}
// package p2
package p2 ;
import p1 .*;
\par
public class B {
\par
static void display () {
System.out.println ("This is class B!");
}
public static void main (String[] args){
A obj = new A () ;
obj . display () ;
display () ;
}
}
\end{lstlisting}%
\lthtmlfigureZ
\lthtmlcheckvsize\clearpage}

\stepcounter{subsubsection}
{\newpage\clearpage
\lthtmlfigureA{lstlisting333}%
\begin{lstlisting}
// Java program to illustrate error while
// using class from different package with
// private modifier
package p1 ;
\par
class A
{
	private void display ()
	{
		System.out.println(" This is class A of p1");
	}
}
\par
class B
{
	public static void main (String args[]) {
		A obj = new A () ;	
		// trying to access private method of another class
		obj.display () ;
	}
}
\end{lstlisting}%
\lthtmlfigureZ
\lthtmlcheckvsize\clearpage}

\stepcounter{subsubsection}
{\newpage\clearpage
\lthtmlfigureA{lstlisting349}%
\begin{lstlisting}
// Java program to illustrate
// protected modifier
\par
package p1 ;
\par
// Class A
public class A {
	protected void display () {
		System.out.println("This is class A of p1");
	}
}
\par
// Java program to illustrate
// protected modifier
\par
package p2 ;
import p1 .*; // importing all classes in package p1
\par
// Class B is subclass of A
\par
class B extends A {
	public static void main (String args[]) {
		B obj = new B() ;
		obj.display() ;
	}
}
\end{lstlisting}%
\lthtmlfigureZ
\lthtmlcheckvsize\clearpage}

\stepcounter{subsection}
\stepcounter{subsubsection}
\stepcounter{subsubsection}
{\newpage\clearpage
\lthtmlfigureA{lstlisting361}%
\begin{lstlisting}
package teststatic ;
\par
public class TestStatic {
\par
static void callStatic() {
		System.out.println (" Static method called in main method !");
	}
\par
void callNonStatic() {
		System . out . println ("Non - static method called in main method ! using ’instance ’");
	}
\par
public static void main(String[] args) {
		// this static method
		callStatic();
\par
// this is non - staic method
		TestStatic obj = new TestStatic();
		obj.callNonStatic();
		// this is also possible
		obj.callStatic();
	}
}
\end{lstlisting}%
\lthtmlfigureZ
\lthtmlcheckvsize\clearpage}

\stepcounter{subsubsection}
\stepcounter{subsubsection}
{\newpage\clearpage
\lthtmlfigureA{lstlisting368}%
\begin{lstlisting}
package teststatic ;
\par
class Parent {
	void show() {
		System.out.println("Parent");
	}
}
\par
// Parent inherit in Child class
\par
class Child extends Parent{
	// override show () of Parent
	void show(){
		System.out.println("Child");
	}
}
\par
class TestStatic {
	public static void main (String[] args){
		Parent p = new Parent();
		// calling Parent ’s show ()
		p.show();
		Parent c = new Child();
		// calling Child ’s show ()
		c.show();
	}
}
\end{lstlisting}%
\lthtmlfigureZ
\lthtmlcheckvsize\clearpage}

{\newpage\clearpage
\lthtmlfigureA{lstlisting373}%
\begin{lstlisting}
package teststatic ;
\par
class Parent {
	static void show() {
		System.out.println("Parent");
	}	
}
\par
// Parent inherit in Child class
class Child extends Parent {
	// override show () of Parent
	void show() {
		System.out.println("Child");
	}	
}
\par
class TestStatic {
	public static void main(String[] args) {
		Parent p = new Parent();
		// calling Parent ’s show ()
		p.show();
		Parent c = new Child () ;
		// calling Child ’s show ()
		c.show();
	}
}
\end{lstlisting}%
\lthtmlfigureZ
\lthtmlcheckvsize\clearpage}

{\newpage\clearpage
\lthtmlfigureA{lstlisting378}%
\begin{lstlisting}
Multiple markers at this line
- This instance method cannot override the static
	method
	from Parent
\par
\end{lstlisting}%
\lthtmlfigureZ
\lthtmlcheckvsize\clearpage}

\stepcounter{subsection}
{\newpage\clearpage
\lthtmlfigureA{lstlisting381}%
\begin{lstlisting}
int score;
\end{lstlisting}%
\lthtmlfigureZ
\lthtmlcheckvsize\clearpage}

{\newpage\clearpage
\lthtmlinlinemathA{tex2html_wrap_inline4451}%
$\lstinline{int}$%
\lthtmlindisplaymathZ
\lthtmlcheckvsize\clearpage}

{\newpage\clearpage
\lthtmlinlinemathA{tex2html_wrap_inline4453}%
$\lstinline{score}$%
\lthtmlindisplaymathZ
\lthtmlcheckvsize\clearpage}

{\newpage\clearpage
\lthtmlinlinemathA{tex2html_wrap_inline4455}%
$\lstinline{int, for, class}$%
\lthtmlindisplaymathZ
\lthtmlcheckvsize\clearpage}

\stepcounter{subsubsection}
{\newpage\clearpage
\lthtmlinlinemathA{tex2html_wrap_inline4457}%
$\lstinline{true, false}$%
\lthtmlindisplaymathZ
\lthtmlcheckvsize\clearpage}

{\newpage\clearpage
\lthtmlinlinemathA{tex2html_wrap_inline4459}%
$\lstinline{null}$%
\lthtmlindisplaymathZ
\lthtmlcheckvsize\clearpage}

\stepcounter{subsection}
{\newpage\clearpage
\lthtmlfigureA{lstlisting399}%
\begin{lstlisting}
int float;
\end{lstlisting}%
\lthtmlfigureZ
\lthtmlcheckvsize\clearpage}

{\newpage\clearpage
\lthtmlinlinemathA{tex2html_wrap_inline4463}%
$\lstinline{float}$%
\lthtmlindisplaymathZ
\lthtmlcheckvsize\clearpage}

\stepcounter{subsection}
\stepcounter{section}
{\newpage\clearpage
\lthtmlfigureA{lstlisting413}%
\begin{lstlisting}
int speedLimit = 80;
\end{lstlisting}%
\lthtmlfigureZ
\lthtmlcheckvsize\clearpage}

{\newpage\clearpage
\lthtmlinlinemathA{tex2html_wrap_inline4465}%
$\lstinline{speedLimit}$%
\lthtmlindisplaymathZ
\lthtmlcheckvsize\clearpage}

{\newpage\clearpage
\lthtmlfigureA{lstlisting418}%
\begin{lstlisting}
int speedLimit;
speedLimit = 80;
\end{lstlisting}%
\lthtmlfigureZ
\lthtmlcheckvsize\clearpage}

{\newpage\clearpage
\lthtmlfigureA{lstlisting421}%
\begin{lstlisting}
int speedLimit = 80;
... ... ...
speedLimit = 90; 
\end{lstlisting}%
\lthtmlfigureZ
\lthtmlcheckvsize\clearpage}

{\newpage\clearpage
\lthtmlfigureA{lstlisting423}%
\begin{lstlisting}
int speedLimit = 80;
... ... ...
float speedLimit;
\end{lstlisting}%
\lthtmlfigureZ
\lthtmlcheckvsize\clearpage}

\stepcounter{subsection}
\stepcounter{subsection}
{\newpage\clearpage
\lthtmlfigureA{lstlisting439}%
\begin{lstlisting}
int speed;
\end{lstlisting}%
\lthtmlfigureZ
\lthtmlcheckvsize\clearpage}

{\newpage\clearpage
\lthtmlinlinemathA{tex2html_wrap_inline4471}%
$\lstinline{speed}$%
\lthtmlindisplaymathZ
\lthtmlcheckvsize\clearpage}

\stepcounter{subsection}
\stepcounter{subsubsection}
{\newpage\clearpage
\lthtmlinlinemathA{tex2html_wrap_inline4477}%
$\lstinline{boolean}$%
\lthtmlindisplaymathZ
\lthtmlcheckvsize\clearpage}

{\newpage\clearpage
\lthtmlinlinemathA{tex2html_wrap_inline4481}%
$\lstinline{false}$%
\lthtmlindisplaymathZ
\lthtmlcheckvsize\clearpage}

{\newpage\clearpage
\lthtmlfigureA{lstlisting450}%
\begin{lstlisting}
class BooleanExample {
    public static void main(String[] args) {
\par
boolean flag = true;
        System.out.println(flag);
    }
}
\end{lstlisting}%
\lthtmlfigureZ
\lthtmlcheckvsize\clearpage}

\stepcounter{subsubsection}
{\newpage\clearpage
\lthtmlinlinemathA{tex2html_wrap_inline4485}%
$\lstinline{byte}$%
\lthtmlindisplaymathZ
\lthtmlcheckvsize\clearpage}

{\newpage\clearpage
\lthtmlinlinemathA{tex2html_wrap_inline4487}%
$-128$%
\lthtmlindisplaymathZ
\lthtmlcheckvsize\clearpage}

{\newpage\clearpage
\lthtmlinlinemathA{tex2html_wrap_inline4489}%
$127$%
\lthtmlindisplaymathZ
\lthtmlcheckvsize\clearpage}

{\newpage\clearpage
\lthtmlinlinemathA{tex2html_wrap_inline4493}%
$[-128, 127]$%
\lthtmlindisplaymathZ
\lthtmlcheckvsize\clearpage}

{\newpage\clearpage
\lthtmlfigureA{lstlisting456}%
\begin{lstlisting}
class ByteExample {
    public static void main(String[] args) {
\par
byte range;
        range = 124;
        System.out.println(range);
    }
}
\end{lstlisting}%
\lthtmlfigureZ
\lthtmlcheckvsize\clearpage}

\stepcounter{subsubsection}
{\newpage\clearpage
\lthtmlinlinemathA{tex2html_wrap_inline4495}%
$\lstinline{short}$%
\lthtmlindisplaymathZ
\lthtmlcheckvsize\clearpage}

{\newpage\clearpage
\lthtmlinlinemathA{tex2html_wrap_inline4497}%
$-32768 to 32767$%
\lthtmlindisplaymathZ
\lthtmlcheckvsize\clearpage}

{\newpage\clearpage
\lthtmlinlinemathA{tex2html_wrap_inline4499}%
$[-32768, 32767]$%
\lthtmlindisplaymathZ
\lthtmlcheckvsize\clearpage}

{\newpage\clearpage
\lthtmlfigureA{lstlisting461}%
\begin{lstlisting}
class ShortExample {
    public static void main(String[] args) {
\par
short temperature;
        temperature = -200;
        System.out.println(temperature);
    }
}
\end{lstlisting}%
\lthtmlfigureZ
\lthtmlcheckvsize\clearpage}

\stepcounter{subsubsection}
{\newpage\clearpage
\lthtmlinlinemathA{tex2html_wrap_inline4503}%
$-2^31$%
\lthtmlindisplaymathZ
\lthtmlcheckvsize\clearpage}

{\newpage\clearpage
\lthtmlinlinemathA{tex2html_wrap_inline4505}%
$2^{31}-1$%
\lthtmlindisplaymathZ
\lthtmlcheckvsize\clearpage}

{\newpage\clearpage
\lthtmlinlinemathA{tex2html_wrap_inline4507}%
$2^{32}-1$%
\lthtmlindisplaymathZ
\lthtmlcheckvsize\clearpage}

{\newpage\clearpage
\lthtmlfigureA{lstlisting468}%
\begin{lstlisting}
class IntExample {
    public static void main(String[] args) {
\par
int range = -4250000;
        System.out.println(range);
    }
}
\end{lstlisting}%
\lthtmlfigureZ
\lthtmlcheckvsize\clearpage}

\stepcounter{subsubsection}
{\newpage\clearpage
\lthtmlinlinemathA{tex2html_wrap_inline4509}%
$\lstinline{long}$%
\lthtmlindisplaymathZ
\lthtmlcheckvsize\clearpage}

{\newpage\clearpage
\lthtmlinlinemathA{tex2html_wrap_inline4511}%
$-2^63$%
\lthtmlindisplaymathZ
\lthtmlcheckvsize\clearpage}

{\newpage\clearpage
\lthtmlinlinemathA{tex2html_wrap_inline4513}%
$2^{63}-1$%
\lthtmlindisplaymathZ
\lthtmlcheckvsize\clearpage}

{\newpage\clearpage
\lthtmlinlinemathA{tex2html_wrap_inline4515}%
$2^{64}-1$%
\lthtmlindisplaymathZ
\lthtmlcheckvsize\clearpage}

{\newpage\clearpage
\lthtmlfigureA{lstlisting475}%
\begin{lstlisting}
class LongExample {
    public static void main(String[] args) {
\par
long range = -42332200000L;
        System.out.println(range);
    }
}
\end{lstlisting}%
\lthtmlfigureZ
\lthtmlcheckvsize\clearpage}

{\newpage\clearpage
\lthtmlinlinemathA{tex2html_wrap_inline4517}%
$-42332200000$%
\lthtmlindisplaymathZ
\lthtmlcheckvsize\clearpage}

\stepcounter{subsubsection}
{\newpage\clearpage
\lthtmlinlinemathA{tex2html_wrap_inline4519}%
$\lstinline{double}$%
\lthtmlindisplaymathZ
\lthtmlcheckvsize\clearpage}

{\newpage\clearpage
\lthtmlfigureA{lstlisting481}%
\begin{lstlisting}
class DoubleExample {
    public static void main(String[] args) {
\par
double number = -42.3;
        System.out.println(number);
    }
}
\end{lstlisting}%
\lthtmlfigureZ
\lthtmlcheckvsize\clearpage}

\stepcounter{subsubsection}
{\newpage\clearpage
\lthtmlfigureA{lstlisting486}%
\begin{lstlisting}
class FloatExample {
    public static void main(String[] args) {
\par
float number = -42.3f;
        System.out.println(number);
    }
}
\par
\end{lstlisting}%
\lthtmlfigureZ
\lthtmlcheckvsize\clearpage}

{\newpage\clearpage
\lthtmlinlinemathA{tex2html_wrap_inline4523}%
$-42.3f$%
\lthtmlindisplaymathZ
\lthtmlcheckvsize\clearpage}

{\newpage\clearpage
\lthtmlinlinemathA{tex2html_wrap_inline4525}%
$-42.3$%
\lthtmlindisplaymathZ
\lthtmlcheckvsize\clearpage}

\stepcounter{subsubsection}
{\newpage\clearpage
\lthtmlinlinemathA{tex2html_wrap_inline4535}%
$'\symbol{92}u0000'$%
\lthtmlindisplaymathZ
\lthtmlcheckvsize\clearpage}

{\newpage\clearpage
\lthtmlinlinemathA{tex2html_wrap_inline4537}%
$'\symbol{92}uffff'$%
\lthtmlindisplaymathZ
\lthtmlcheckvsize\clearpage}

{\newpage\clearpage
\lthtmlfigureA{lstlisting497}%
\begin{lstlisting}
class CharExample {
    public static void main(String[] args) {
\par
char letter = '\u0051';
        System.out.println(letter);
    }
}
\end{lstlisting}%
\lthtmlfigureZ
\lthtmlcheckvsize\clearpage}

{\newpage\clearpage
\lthtmlinlinemathA{tex2html_wrap_inline4541}%
$'\symbol{92}u0051'$%
\lthtmlindisplaymathZ
\lthtmlcheckvsize\clearpage}

{\newpage\clearpage
\lthtmlfigureA{lstlisting503}%
\begin{lstlisting}
class CharExample {
    public static void main(String[] args) {
\par
char letter1 = '9';
        System.out.println(letter1);
\par
char letter2 = 65;
        System.out.println(letter2);
\par
}
}
\end{lstlisting}%
\lthtmlfigureZ
\lthtmlcheckvsize\clearpage}

{\newpage\clearpage
\lthtmlinlinemathA{tex2html_wrap_inline4543}%
$\lstinline{letter1}$%
\lthtmlindisplaymathZ
\lthtmlcheckvsize\clearpage}

{\newpage\clearpage
\lthtmlinlinemathA{tex2html_wrap_inline4547}%
$\lstinline{letter2}$%
\lthtmlindisplaymathZ
\lthtmlcheckvsize\clearpage}

\stepcounter{subsubsection}
{\newpage\clearpage
\lthtmlfigureA{lstlisting510}%
\begin{lstlisting}
myString = "Programming is awesome";
\end{lstlisting}%
\lthtmlfigureZ
\lthtmlcheckvsize\clearpage}

\stepcounter{section}
{\newpage\clearpage
\lthtmlfigureA{lstlisting513}%
\begin{lstlisting}
boolean flag = false;
\end{lstlisting}%
\lthtmlfigureZ
\lthtmlcheckvsize\clearpage}

{\newpage\clearpage
\lthtmlinlinemathA{tex2html_wrap_inline4551}%
$\lstinline{flag}$%
\lthtmlindisplaymathZ
\lthtmlcheckvsize\clearpage}

{\newpage\clearpage
\lthtmlinlinemathA{tex2html_wrap_inline4555}%
$1.5, 4, true, '\symbol{92}u0050'$%
\lthtmlindisplaymathZ
\lthtmlcheckvsize\clearpage}

{\newpage\clearpage
\lthtmlinlinemathA{tex2html_wrap_inline4559}%
$-5, 'a', true$%
\lthtmlindisplaymathZ
\lthtmlcheckvsize\clearpage}

\stepcounter{subsection}
{\newpage\clearpage
\lthtmlfigureA{lstlisting528}%
\begin{lstlisting}
// Error! literal 42332200000 of type int is out of range
int myVariable1 = 42332200000;// 42332200000L is of type long, and it's not out of range
long myVariable2 = 42332200000L;
\end{lstlisting}%
\lthtmlfigureZ
\lthtmlcheckvsize\clearpage}

{\newpage\clearpage
\lthtmlinlinemathA{tex2html_wrap_inline4561}%
$0x$%
\lthtmlindisplaymathZ
\lthtmlcheckvsize\clearpage}

{\newpage\clearpage
\lthtmlinlinemathA{tex2html_wrap_inline4563}%
$0b$%
\lthtmlindisplaymathZ
\lthtmlcheckvsize\clearpage}

{\newpage\clearpage
\lthtmlfigureA{lstlisting530}%
\begin{lstlisting}
// decimal
int decNumber = 34;
int hexNumber = 0x2F; // 0x represents hexadecimal
int binNumber = 0b10010; // 0b represents binary
\end{lstlisting}%
\lthtmlfigureZ
\lthtmlcheckvsize\clearpage}

\stepcounter{subsection}
{\newpage\clearpage
\lthtmlfigureA{lstlisting541}%
\begin{lstlisting}
class DoubleExample {
    public static void main(String[] args) {
\par
double myDouble = 3.4;
        float myFloat = 3.4F;
\par
// 3.445*10^2
        double myDoubleScientific = 3.445e2;
\par
System.out.println(myDouble);
        System.out.println(myFloat);
        System.out.println(myDoubleScientific);
    }
}
\end{lstlisting}%
\lthtmlfigureZ
\lthtmlcheckvsize\clearpage}

\stepcounter{subsection}
{\newpage\clearpage
\lthtmlinlinemathA{tex2html_wrap_inline4565}%
$\lstinline{char}$%
\lthtmlindisplaymathZ
\lthtmlcheckvsize\clearpage}

{\newpage\clearpage
\lthtmlinlinemathA{tex2html_wrap_inline4567}%
$'a', '\symbol{92}u0111'$%
\lthtmlindisplaymathZ
\lthtmlcheckvsize\clearpage}

{\newpage\clearpage
\lthtmlinlinemathA{tex2html_wrap_inline4571}%
$\symbol{92}b$%
\lthtmlindisplaymathZ
\lthtmlcheckvsize\clearpage}

{\newpage\clearpage
\lthtmlinlinemathA{tex2html_wrap_inline4573}%
$\symbol{92}t$%
\lthtmlindisplaymathZ
\lthtmlcheckvsize\clearpage}

{\newpage\clearpage
\lthtmlinlinemathA{tex2html_wrap_inline4575}%
$\symbol{92}n$%
\lthtmlindisplaymathZ
\lthtmlcheckvsize\clearpage}

{\newpage\clearpage
\lthtmlinlinemathA{tex2html_wrap_inline4577}%
$\symbol{92}f$%
\lthtmlindisplaymathZ
\lthtmlcheckvsize\clearpage}

{\newpage\clearpage
\lthtmlinlinemathA{tex2html_wrap_inline4579}%
$\symbol{92}r$%
\lthtmlindisplaymathZ
\lthtmlcheckvsize\clearpage}

{\newpage\clearpage
\lthtmlinlinemathA{tex2html_wrap_inline4581}%
$\symbol{92}"$%
\lthtmlindisplaymathZ
\lthtmlcheckvsize\clearpage}

{\newpage\clearpage
\lthtmlinlinemathA{tex2html_wrap_inline4583}%
$\symbol{92}'$%
\lthtmlindisplaymathZ
\lthtmlcheckvsize\clearpage}

{\newpage\clearpage
\lthtmlinlinemathA{tex2html_wrap_inline4585}%
$\symbol{92}\symbol{92}$%
\lthtmlindisplaymathZ
\lthtmlcheckvsize\clearpage}

{\newpage\clearpage
\lthtmlfigureA{lstlisting559}%
\begin{lstlisting}
class DoubleExample {
    public static void main(String[] args) {
\par
char myChar = 'g';
        char newLine = '\n';
        String myString = "Java 8";
\par
System.out.println(myChar);
        System.out.println(newLine);
        System.out.println(myString);
    }
}
\end{lstlisting}%
\lthtmlfigureZ
\lthtmlcheckvsize\clearpage}

\stepcounter{section}
\stepcounter{subsection}
{\newpage\clearpage
\lthtmlfigureA{lstlisting565}%
\begin{lstlisting}
int age;
age = 5;
\end{lstlisting}%
\lthtmlfigureZ
\lthtmlcheckvsize\clearpage}

\stepcounter{subsubsection}
{\newpage\clearpage
\lthtmlfigureA{lstlisting570}%
\begin{lstlisting}
class AssignmentOperator {
    public static void main(String[] args) {
\par
int number1, number2;
\par
// Assigning 5 to number1 
        number1 = 5;
        System.out.println(number1);
\par
// Assigning value of variable number2 to number1
        number2 = number1;
        System.out.println(number2);
    }
}
\end{lstlisting}%
\lthtmlfigureZ
\lthtmlcheckvsize\clearpage}

\stepcounter{subsection}
\stepcounter{subsubsection}
{\newpage\clearpage
\lthtmlfigureA{lstlisting582}%
\begin{lstlisting}
class ArithmeticOperator {
    public static void main(String[] args) {
\par
double number1 = 12.5, number2 = 3.5, result;
\par
// Using addition operator
        result = number1 + number2;
        System.out.println("number1 + number2 = " + result);
\par
// Using subtraction operator
        result = number1 - number2;
        System.out.println("number1 - number2 = " + result);
\par
// Using multiplication operator
        result = number1 * number2;
        System.out.println("number1 * number2 = " + result);
\par
// Using division operator
        result = number1 / number2;
        System.out.println("number1 / number2 = " + result);
\par
// Using remainder operator
        result = number1 % number2;
        System.out.println("number1 % number2 = " + result);
    }
}
\end{lstlisting}%
\lthtmlfigureZ
\lthtmlcheckvsize\clearpage}

{\newpage\clearpage
\lthtmlfigureA{lstlisting585}%
\begin{lstlisting}
result = number1 + 5.2;
result = 2.3 + 4.5;
number2 = number1 -2.9;
\end{lstlisting}%
\lthtmlfigureZ
\lthtmlcheckvsize\clearpage}

{\newpage\clearpage
\lthtmlfigureA{lstlisting587}%
\begin{lstlisting}
class ArithmeticOperator {
    public static void main(String[] args) {
\par
String start, middle, end, result;
\par
start = "Talk is cheap. ";
        middle = "Show me the code. ";
        end = "- Linus Torvalds";
\par
result = start + middle + end;
        System.out.println(result);
    }
}
\end{lstlisting}%
\lthtmlfigureZ
\lthtmlcheckvsize\clearpage}

\stepcounter{subsection}
\stepcounter{subsubsection}
{\newpage\clearpage
\lthtmlfigureA{lstlisting601}%
\begin{lstlisting}
class UnaryOperator {
    public static void main(String[] args) {
\par
double number = 5.2, resultNumber;
        boolean flag = false;
\par
System.out.println("+number = " + +number);
        // number is equal to 5.2 here.
\par
System.out.println("-number = " + -number);
        // number is equal to 5.2 here.
\par
// ++number is equivalent to number = number + 1
        System.out.println("number = " + ++number);
        // number is equal to 6.2 here.
\par
// -- number is equivalent to number = number - 1
        System.out.println("number = " + --number);
        // number is equal to 5.2 here.
\par
System.out.println("!flag = " + !flag);
        // flag is still false.
    }
}
\end{lstlisting}%
\lthtmlfigureZ
\lthtmlcheckvsize\clearpage}

{\newpage\clearpage
\lthtmlfigureA{lstlisting604}%
\begin{lstlisting}
+number = 5.2
-number = -5.2
number = 6.2
number = 5.2
!flag = true
\end{lstlisting}%
\lthtmlfigureZ
\lthtmlcheckvsize\clearpage}

\stepcounter{subsubsection}
{\newpage\clearpage
\lthtmlinlinemathA{tex2html_wrap_inline4587}%
$\lstinline{++}$%
\lthtmlindisplaymathZ
\lthtmlcheckvsize\clearpage}

{\newpage\clearpage
\lthtmlinlinemathA{tex2html_wrap_inline4589}%
$\lstinline{--}$%
\lthtmlindisplaymathZ
\lthtmlcheckvsize\clearpage}

{\newpage\clearpage
\lthtmlfigureA{lstlisting610}%
\begin{lstlisting}
int myInt = 5;
++myInt   // myInt becomes 6
myInt++   // myInt becomes 7
--myInt   // myInt becomes 6
myInt--   // myInt becomes 5
\end{lstlisting}%
\lthtmlfigureZ
\lthtmlcheckvsize\clearpage}

\stepcounter{subsubsection}
{\newpage\clearpage
\lthtmlfigureA{lstlisting613}%
\begin{lstlisting}
class UnaryOperator {
    public static void main(String[] args) {
\par
double number = 5.2;
\par
System.out.println(number++);
        System.out.println(number);
\par
System.out.println(++number);
        System.out.println(number);
    }
}
\end{lstlisting}%
\lthtmlfigureZ
\lthtmlcheckvsize\clearpage}

{\newpage\clearpage
\lthtmlfigureA{lstlisting616}%
\begin{lstlisting}
5.2
6.2
7.2
7.2
\end{lstlisting}%
\lthtmlfigureZ
\lthtmlcheckvsize\clearpage}

{\newpage\clearpage
\lthtmlinlinemathA{tex2html_wrap_inline4593}%
$\lstinline{System.out.println(number++);}$%
\lthtmlindisplaymathZ
\lthtmlcheckvsize\clearpage}

{\newpage\clearpage
\lthtmlinlinemathA{tex2html_wrap_inline4595}%
$\lstinline{System.out.println(number);}$%
\lthtmlindisplaymathZ
\lthtmlcheckvsize\clearpage}

{\newpage\clearpage
\lthtmlinlinemathA{tex2html_wrap_inline4597}%
$\texttt{System.out.println(++number);}$%
\lthtmlindisplaymathZ
\lthtmlcheckvsize\clearpage}

\stepcounter{subsection}
\stepcounter{subsubsection}
{\newpage\clearpage
\lthtmlfigureA{lstlisting632}%
\begin{lstlisting}
class RelationalOperator {
    public static void main(String[] args) {
\par
int number1 = 5, number2 = 6;
\par
if (number1 > number2) {
            System.out.println("number1 is greater than number2.");
        }
        else {
            System.out.println("number2 is greater than number1.");
        }
    }
}
\end{lstlisting}%
\lthtmlfigureZ
\lthtmlcheckvsize\clearpage}

\stepcounter{subsection}
{\newpage\clearpage
\lthtmlfigureA{lstlisting644}%
\begin{lstlisting}
class instanceofOperator {
    public static void main(String[] args) {
\par
String test = "asdf";
        boolean result;
\par
result = test instanceof String;
        System.out.println("Is test an object of String? " + result);
    }
}
\end{lstlisting}%
\lthtmlfigureZ
\lthtmlcheckvsize\clearpage}

\stepcounter{subsubsection}
{\newpage\clearpage
\lthtmlfigureA{lstlisting651}%
\begin{lstlisting}
result = objectName instanceof className;
\end{lstlisting}%
\lthtmlfigureZ
\lthtmlcheckvsize\clearpage}

\stepcounter{subsubsection}
{\newpage\clearpage
\lthtmlfigureA{lstlisting655}%
\begin{lstlisting}
class Main {
    public static void main (String[] args) {
        String name = "Programiz";
        Integer age = 22;
\par
System.out.println("Is name an instance of String: "+ (name instanceof String));
        System.out.println("Is age an instance of Integer: "+ (age instanceof Integer));
    }
}
\end{lstlisting}%
\lthtmlfigureZ
\lthtmlcheckvsize\clearpage}

{\newpage\clearpage
\lthtmlfigureA{lstlisting658}%
\begin{lstlisting}
Is name an instance of String: true
Is age an instance of Integer: true
\end{lstlisting}%
\lthtmlfigureZ
\lthtmlcheckvsize\clearpage}

\stepcounter{section}
{\newpage\clearpage
\lthtmlfigureA{lstlisting675}%
\begin{lstlisting}
String[] array = new String[100];
\end{lstlisting}%
\lthtmlfigureZ
\lthtmlcheckvsize\clearpage}

{\newpage\clearpage
\lthtmlfigureA{lstlisting677}%
\begin{lstlisting}
dataType[] arrayName;
\end{lstlisting}%
\lthtmlfigureZ
\lthtmlcheckvsize\clearpage}

{\newpage\clearpage
\lthtmlfigureA{lstlisting681}%
\begin{lstlisting}
double[] data;
\end{lstlisting}%
\lthtmlfigureZ
\lthtmlcheckvsize\clearpage}

{\newpage\clearpage
\lthtmlfigureA{lstlisting685}%
\begin{lstlisting}
// declare an array
double[] data;
\par
// allocate memory
data = new Double[10];
\end{lstlisting}%
\lthtmlfigureZ
\lthtmlcheckvsize\clearpage}

{\newpage\clearpage
\lthtmlfigureA{lstlisting687}%
\begin{lstlisting}
double[] data = new double[10];
\end{lstlisting}%
\lthtmlfigureZ
\lthtmlcheckvsize\clearpage}

\stepcounter{subsection}
{\newpage\clearpage
\lthtmlfigureA{lstlisting690}%
\begin{lstlisting}
//declare and initialize and array
int[] age = {12, 4, 5, 2, 5};
\end{lstlisting}%
\lthtmlfigureZ
\lthtmlcheckvsize\clearpage}

{\newpage\clearpage
\lthtmlfigureA{lstlisting693}%
\begin{lstlisting}
// declare an array
int[] age = new int[5];
\par
// initialize array
age[0] = 12;
age[1] = 4;
age[2] = 5;
..
\end{lstlisting}%
\lthtmlfigureZ
\lthtmlcheckvsize\clearpage}

\stepcounter{subsection}
{\newpage\clearpage
\lthtmlfigureA{lstlisting698}%
\begin{lstlisting}
// access array elements
array[index]
\end{lstlisting}%
\lthtmlfigureZ
\lthtmlcheckvsize\clearpage}

{\newpage\clearpage
\lthtmlfigureA{lstlisting700}%
\begin{lstlisting}
class Main {
 public static void main(String[] args) {
\par
// create an array
   int[] age = {12, 4, 5, 2, 5};
\par
// access each array elements
   System.out.println("Accessing Elements of Array:");
   System.out.println("First Element: " + age[0]);
   System.out.println("Second Element: " + age[1]);
   System.out.println("Third Element: " + age[2]);
   System.out.println("Fourth Element: " + age[3]);
   System.out.println("Fifth Element: " + age[4]);
 }
}
\end{lstlisting}%
\lthtmlfigureZ
\lthtmlcheckvsize\clearpage}

{\newpage\clearpage
\lthtmlfigureA{lstlisting703}%
\begin{lstlisting}
Accessing Elements of Array:
First Element: 12
Second Element: 4
Third Element: 5
Fourth Element: 2
Fifth Element: 5
\end{lstlisting}%
\lthtmlfigureZ
\lthtmlcheckvsize\clearpage}

\stepcounter{subsection}
{\newpage\clearpage
\lthtmlfigureA{lstlisting706}%
\begin{lstlisting}
class Main {
 public static void main(String[] args) {
\par
// create an array
   int[] age = {12, 4, 5};
\par
// loop through the array
   // using for loop
   System.out.println("Using for Loop:");
   for(int i = 0; i < age.length; i++) {
     System.out.println(age[i]);
   }
 }
}
\end{lstlisting}%
\lthtmlfigureZ
\lthtmlcheckvsize\clearpage}

{\newpage\clearpage
\lthtmlfigureA{lstlisting710}%
\begin{lstlisting}
Using for Loop:
12
4
5
\end{lstlisting}%
\lthtmlfigureZ
\lthtmlcheckvsize\clearpage}

{\newpage\clearpage
\lthtmlfigureA{lstlisting713}%
\begin{lstlisting}
class Main {
 public static void main(String[] args) {
\par
// create an array
   int[] age = {12, 4, 5};
\par
// loop through the array
   // using for loop
   System.out.println("Using for-each Loop:");
   for(int a : age) {
     System.out.println(a);
   }
 }
}
\end{lstlisting}%
\lthtmlfigureZ
\lthtmlcheckvsize\clearpage}

{\newpage\clearpage
\lthtmlfigureA{lstlisting717}%
\begin{lstlisting}
Using for-each Loop:
12
4
5
\end{lstlisting}%
\lthtmlfigureZ
\lthtmlcheckvsize\clearpage}

{\newpage\clearpage
\lthtmlfigureA{lstlisting719}%
\begin{lstlisting}
class Main {
 public static void main(String[] args) {
\par
int[] numbers = {2, -9, 0, 5, 12, -25, 22, 9, 8, 12};
   int sum = 0;
   Double average;
\par
// access all elements using for each loop
   // add each element in sum
   for (int number: numbers) {
     sum += number;
   }
\par
// get the total number of elements
   int arrayLength = numbers.length;
\par
// calculate the average
   // convert the average from int to double
   average =  ((double)sum / (double)arrayLength);
\par
System.out.println("Sum = " + sum);
   System.out.println("Average = " + average);
 }
}
\end{lstlisting}%
\lthtmlfigureZ
\lthtmlcheckvsize\clearpage}

{\newpage\clearpage
\lthtmlfigureA{lstlisting723}%
\begin{lstlisting}
Sum = 36
Average = 3.6
\end{lstlisting}%
\lthtmlfigureZ
\lthtmlcheckvsize\clearpage}

{\newpage\clearpage
\lthtmlfigureA{lstlisting728}%
\begin{lstlisting}
average = ((double)sum / (double)arrayLength);
\end{lstlisting}%
\lthtmlfigureZ
\lthtmlcheckvsize\clearpage}

\stepcounter{subsection}
{\newpage\clearpage
\lthtmlfigureA{lstlisting731}%
\begin{lstlisting}
double[][] matrix = {{1.2, 4.3, 4.0}, 
      {4.1, -1.1}
};
\end{lstlisting}%
\lthtmlfigureZ
\lthtmlcheckvsize\clearpage}

{\newpage\clearpage
\lthtmlfigureA{lstlisting735}%
\begin{lstlisting}
int[][] a = new int[3][4];
\end{lstlisting}%
\lthtmlfigureZ
\lthtmlcheckvsize\clearpage}

{\newpage\clearpage
\lthtmlfigureA{lstlisting738}%
\begin{lstlisting}
String[][][] data = new String[3][4][2];
\end{lstlisting}%
\lthtmlfigureZ
\lthtmlcheckvsize\clearpage}

\stepcounter{subsection}
{\newpage\clearpage
\lthtmlfigureA{lstlisting741}%
\begin{lstlisting}
int[][] a = {
      {1, 2, 3}, 
      {4, 5, 6, 9}, 
      {7}, 
};
\end{lstlisting}%
\lthtmlfigureZ
\lthtmlcheckvsize\clearpage}

{\newpage\clearpage
\lthtmlfigureA{lstlisting746}%
\begin{lstlisting}
class MultidimensionalArray {
    public static void main(String[] args) {
\par
// create a 2d array
        int[][] a = {
            {1, 2, 3}, 
            {4, 5, 6, 9}, 
            {7}, 
        };
\par
// calculate the length of each row
        System.out.println("Length of row 1: " + a[0].length);
        System.out.println("Length of row 2: " + a[1].length);
        System.out.println("Length of row 3: " + a[2].length);
    }
}
\end{lstlisting}%
\lthtmlfigureZ
\lthtmlcheckvsize\clearpage}

{\newpage\clearpage
\lthtmlfigureA{lstlisting751}%
\begin{lstlisting}
Length of row 1: 3
Length of row 2: 4
Length of row 3: 1
\end{lstlisting}%
\lthtmlfigureZ
\lthtmlcheckvsize\clearpage}

{\newpage\clearpage
\lthtmlfigureA{lstlisting755}%
\begin{lstlisting}
class MultidimensionalArray {
    public static void main(String[] args) {
\par
int[][] a = {
            {1, -2, 3}, 
            {-4, -5, 6, 9}, 
            {7}, 
        };
\par
for (int i = 0; i < a.length; ++i) {
            for(int j = 0; j < a[i].length; ++j) {
                System.out.println(a[i][j]);
            }
        }
    }
}
\end{lstlisting}%
\lthtmlfigureZ
\lthtmlcheckvsize\clearpage}

{\newpage\clearpage
\lthtmlfigureA{lstlisting761}%
\begin{lstlisting}
1
-2
3
-4
-5
6
9
7
\end{lstlisting}%
\lthtmlfigureZ
\lthtmlcheckvsize\clearpage}

{\newpage\clearpage
\lthtmlfigureA{lstlisting763}%
\begin{lstlisting}
class MultidimensionalArray {
    public static void main(String[] args) {
\par
// create a 2d array
        int[][] a = {
            {1, -2, 3}, 
            {-4, -5, 6, 9}, 
            {7}, 
        };
\par
// first for...each loop access the individual array
        // inside the 2d array
        for (int[] innerArray: a) {
            // second for...each loop access each element inside the row
            for(int data: innerArray) {
                System.out.println(data);
            }
        }
    }
}
\end{lstlisting}%
\lthtmlfigureZ
\lthtmlcheckvsize\clearpage}

{\newpage\clearpage
\lthtmlfigureA{lstlisting773}%
\begin{lstlisting}
// test is a 3d array
int[][][] test = {
        {
          {1, -2, 3}, 
          {2, 3, 4}
        }, 
        { 
          {-4, -5, 6, 9}, 
          {1}, 
          {2, 3}
        } 
};
\end{lstlisting}%
\lthtmlfigureZ
\lthtmlcheckvsize\clearpage}

{\newpage\clearpage
\lthtmlfigureA{lstlisting780}%
\begin{lstlisting}
class ThreeArray {
    public static void main(String[] args) {
\par
// create a 3d array
        int[][][] test = {
            {
              {1, -2, 3}, 
              {2, 3, 4}
            }, 
            { 
              {-4, -5, 6, 9}, 
              {1}, 
              {2, 3}
            } 
        };
\par
// for..each loop to iterate through elements of 3d array
        for (int[][] array2D: test) {
            for (int[] array1D: array2D) {
                for(int item: array1D) {
                    System.out.println(item);
                }
            }
        }
    }
}
\end{lstlisting}%
\lthtmlfigureZ
\lthtmlcheckvsize\clearpage}

{\newpage\clearpage
\lthtmlfigureA{lstlisting788}%
\begin{lstlisting}
1
-2
3
2
3
4
-4
-5
6
9
1
2
3
\end{lstlisting}%
\lthtmlfigureZ
\lthtmlcheckvsize\clearpage}

\stepcounter{subsection}
{\newpage\clearpage
\lthtmlfigureA{lstlisting791}%
\begin{lstlisting}
class Main {
    public static void main(String[] args) {
\par
int [] numbers = {1, 2, 3, 4, 5, 6};
        int [] positiveNumbers = numbers;    // copying arrays
\par
for (int number: positiveNumbers) {
            System.out.print(number + ", ");
        }
    }
}
\end{lstlisting}%
\lthtmlfigureZ
\lthtmlcheckvsize\clearpage}

{\newpage\clearpage
\lthtmlfigureA{lstlisting799}%
\begin{lstlisting}
class Main {
    public static void main(String[] args) {
\par
int [] numbers = {1, 2, 3, 4, 5, 6};
        int [] positiveNumbers = numbers;    // copying arrays
\par
// change value of first array
        numbers[0] = -1;
\par
// printing the second array
        for (int number: positiveNumbers) {
            System.out.print(number + ", ");
        }
    }
}
\end{lstlisting}%
\lthtmlfigureZ
\lthtmlcheckvsize\clearpage}

\stepcounter{subsection}
{\newpage\clearpage
\lthtmlfigureA{lstlisting807}%
\begin{lstlisting}
import java.util.Arrays;
\par
class Main {
    public static void main(String[] args) {
\par
int [] source = {1, 2, 3, 4, 5, 6};
        int [] destination = new int[6];
\par
// iterate and copy elements from source to destination
        for (int i = 0; i < source.length; ++i) {
            destination[i] = source[i];
        }
\par
// converting array to string
        System.out.println(Arrays.toString(destination));
    }
}
\end{lstlisting}%
\lthtmlfigureZ
\lthtmlcheckvsize\clearpage}

\stepcounter{subsection}
{\newpage\clearpage
\lthtmlfigureA{lstlisting819}%
\begin{lstlisting}
arraycopy(Object src, int srcPos,Object dest, int destPos, int length)
\end{lstlisting}%
\lthtmlfigureZ
\lthtmlcheckvsize\clearpage}

{\newpage\clearpage
\lthtmlfigureA{lstlisting828}%
\begin{lstlisting}
// To use Arrays.toString() method
import java.util.Arrays;
\par
class Main {
    public static void main(String[] args) {
        int[] n1 = {2, 3, 12, 4, 12, -2};
\par
int[] n3 = new int[5];
\par
// Creating n2 array of having length of n1 array
        int[] n2 = new int[n1.length];
\par
// copying entire n1 array to n2
        System.arraycopy(n1, 0, n2, 0, n1.length);
        System.out.println("n2 = " + Arrays.toString(n2));  
\par
// copying elements from index 2 on n1 array
        // copying element to index 1 of n3 array
        // 2 elements will be copied
        System.arraycopy(n1, 2, n3, 1, 2);
        System.out.println("n3 = " + Arrays.toString(n3));  
    }
}
\end{lstlisting}%
\lthtmlfigureZ
\lthtmlcheckvsize\clearpage}

{\newpage\clearpage
\lthtmlfigureA{lstlisting831}%
\begin{lstlisting}
n2 = [2, 3, 12, 4, 12, -2]
n3 = [0, 12, 4, 0, 0]
\end{lstlisting}%
\lthtmlfigureZ
\lthtmlcheckvsize\clearpage}

\stepcounter{subsection}
{\newpage\clearpage
\lthtmlfigureA{lstlisting844}%
\begin{lstlisting}
// To use toString() and copyOfRange() method
import java.util.Arrays;
\par
class ArraysCopy {
    public static void main(String[] args) {
\par
int[] source = {2, 3, 12, 4, 12, -2};
\par
// copying entire source array to destination
        int[] destination1 = Arrays.copyOfRange(source, 0, source.length);      
        System.out.println("destination1 = " + Arrays.toString(destination1)); 
\par
// copying from index 2 to 5 (5 is not included) 
        int[] destination2 = Arrays.copyOfRange(source, 2, 5); 
        System.out.println("destination2 = " + Arrays.toString(destination2));   
    }
}
\end{lstlisting}%
\lthtmlfigureZ
\lthtmlcheckvsize\clearpage}

{\newpage\clearpage
\lthtmlfigureA{lstlisting847}%
\begin{lstlisting}
destination1 = [2, 3, 12, 4, 12, -2]
destination2 = [12, 4, 12]
\end{lstlisting}%
\lthtmlfigureZ
\lthtmlcheckvsize\clearpage}

{\newpage\clearpage
\lthtmlfigureA{lstlisting849}%
\begin{lstlisting}
int[] destination1 = Arrays.copyOfRange(source, 0, source.length);
\end{lstlisting}%
\lthtmlfigureZ
\lthtmlcheckvsize\clearpage}

\stepcounter{subsection}
{\newpage\clearpage
\lthtmlfigureA{lstlisting857}%
\begin{lstlisting}
import java.util.Arrays;
\par
class Main {
    public static void main(String[] args) {
\par
int[][] source = {
              {1, 2, 3, 4}, 
              {5, 6},
              {0, 2, 42, -4, 5}
              };
\par
int[][] destination = new int[source.length][];
\par
for (int i = 0; i < destination.length; ++i) {
\par
// allocating space for each row of destination array
            destination[i] = new int[source[i].length];
\par
for (int j = 0; j < destination[i].length; ++j) {
                destination[i][j] = source[i][j];
            }
        }
\par
// displaying destination array
        System.out.println(Arrays.deepToString(destination));  
\par
}
}
\end{lstlisting}%
\lthtmlfigureZ
\lthtmlcheckvsize\clearpage}

{\newpage\clearpage
\lthtmlfigureA{lstlisting863}%
\begin{lstlisting}
[[1, 2, 3, 4], [5, 6], [0, 2, 42, -4, 5]]
\end{lstlisting}%
\lthtmlfigureZ
\lthtmlcheckvsize\clearpage}

{\newpage\clearpage
\lthtmlfigureA{lstlisting865}%
\begin{lstlisting}
System.out.println(Arrays.deepToString(destination);
\end{lstlisting}%
\lthtmlfigureZ
\lthtmlcheckvsize\clearpage}

\stepcounter{subsection}
{\newpage\clearpage
\lthtmlfigureA{lstlisting870}%
\begin{lstlisting}
import java.util.Arrays;
\par
class Main {
    public static void main(String[] args) {
\par
int[][] source = {
              {1, 2, 3, 4}, 
              {5, 6},
              {0, 2, 42, -4, 5}
              };
\par
int[][] destination = new int[source.length][];
\par
for (int i = 0; i < source.length; ++i) {
\par
// allocating space for each row of destination array
             destination[i] = new int[source[i].length];
             System.arraycopy(source[i], 0, destination[i], 0, destination[i].length);
        }
\par
// displaying destination array
        System.out.println(Arrays.deepToString(destination));      
    }
}
\end{lstlisting}%
\lthtmlfigureZ
\lthtmlcheckvsize\clearpage}

\stepcounter{section}
\stepcounter{subsection}
\stepcounter{subsection}
\stepcounter{subsection}
{\newpage\clearpage
\lthtmlfigureA{lstlisting919}%
\begin{lstlisting}
// The Collections framework is defined in the java.util package
import java.util.ArrayList;
\par
class Main {
    public static void main(String[] args){
        ArrayList<String> animals = new ArrayList<>();
        // Add elements
        animals.add("Dog");
        animals.add("Cat");
        animals.add("Horse");
\par
System.out.println("ArrayList: " + animals);
    }
}
\end{lstlisting}%
\lthtmlfigureZ
\lthtmlcheckvsize\clearpage}

\stepcounter{subsection}
\stepcounter{subsection}
\stepcounter{subsubsection}
{\newpage\clearpage
\lthtmlfigureA{lstlisting943}%
\begin{lstlisting}
// ArrayList implementation of List
List<String> list1 = new ArrayList<>();
\par
// LinkedList implementation of List
List<String> list2 = new LinkedList<>();
\end{lstlisting}%
\lthtmlfigureZ
\lthtmlcheckvsize\clearpage}

\stepcounter{subsubsection}
\stepcounter{subsubsection}
{\newpage\clearpage
\lthtmlfigureA{lstlisting971}%
\begin{lstlisting}
import java.util.List;
import java.util.ArrayList;
\par
class Main {
\par
public static void main(String[] args) {
        // Creating list using the ArrayList class
        List<Integer> numbers = new ArrayList<>();
\par
// Add elements to the list
        numbers.add(1);
        numbers.add(2);
        numbers.add(3);
        System.out.println("List: " + numbers);
\par
// Access element from the list
        int number = numbers.get(2);
        System.out.println("Accessed Element: " + number);
\par
// Remove element from the list
        int removedNumber = numbers.remove(1);
        System.out.println("Removed Element: " + removedNumber);
    }
}
\end{lstlisting}%
\lthtmlfigureZ
\lthtmlcheckvsize\clearpage}

{\newpage\clearpage
\lthtmlfigureA{lstlisting974}%
\begin{lstlisting}
List: [1, 2, 3]
Accessed Element: 3
Removed Element: 2
\end{lstlisting}%
\lthtmlfigureZ
\lthtmlcheckvsize\clearpage}

\stepcounter{subsubsection}
{\newpage\clearpage
\lthtmlfigureA{lstlisting977}%
\begin{lstlisting}
import java.util.List;
import java.util.LinkedList;
\par
class Main {
\par
public static void main(String[] args) {
        // Creating list using the LinkedList class
        List<Integer> numbers = new LinkedList<>();
\par
// Add elements to the list
        numbers.add(1);
        numbers.add(2);
        numbers.add(3);
        System.out.println("List: " + numbers);
\par
// Access element from the list
        int number = numbers.get(2);
        System.out.println("Accessed Element: " + number);
\par
// Using the indexOf() method
        int index = numbers.indexOf(2);
        System.out.println("Position of 3 is " + index);
\par
// Remove element from the list
        int removedNumber = numbers.remove(1);
        System.out.println("Removed Element: " + removedNumber);
    }
}
\end{lstlisting}%
\lthtmlfigureZ
\lthtmlcheckvsize\clearpage}

{\newpage\clearpage
\lthtmlfigureA{lstlisting980}%
\begin{lstlisting}
List: [1, 2, 3]
Accessed Element: 3
Position of 3 is 1
Removed Element: 2
\end{lstlisting}%
\lthtmlfigureZ
\lthtmlcheckvsize\clearpage}

\stepcounter{subsubsection}
\stepcounter{subsubsection}
{\newpage\clearpage
\lthtmlfigureA{lstlisting992}%
\begin{lstlisting}
import java.util.ArrayList;
\par
class Main {
    public static void main(String[] args){
        ArrayList<String> animals = new ArrayList<>();
\par
// Add elements
        animals.add(0,"Dog");
        animals.add(1,"Cat");
        animals.add(2,"Horse");
        System.out.println("ArrayList: " + animals);
    }
}
\end{lstlisting}%
\lthtmlfigureZ
\lthtmlcheckvsize\clearpage}

{\newpage\clearpage
\lthtmlfigureA{lstlisting995}%
\begin{lstlisting}
ArrayList: [Dog, Cat, Horse]
\end{lstlisting}%
\lthtmlfigureZ
\lthtmlcheckvsize\clearpage}

\stepcounter{subsubsection}
{\newpage\clearpage
\lthtmlfigureA{lstlisting999}%
\begin{lstlisting}
import java.util.ArrayList;
\par
class Main {
    public static void main(String[] args){
        ArrayList<String> mammals = new ArrayList<>();
        mammals.add("Dog");
        mammals.add("Cat");
        mammals.add("Horse");
        System.out.println("Mammals: " + mammals);
\par
ArrayList<String> animals = new ArrayList<>();
        animals.add("Crocodile");
\par
// Add all elements of mammals in animals
        animals.addAll(mammals);
        System.out.println("Animals: " + animals);
    }
}
\end{lstlisting}%
\lthtmlfigureZ
\lthtmlcheckvsize\clearpage}

{\newpage\clearpage
\lthtmlfigureA{lstlisting1002}%
\begin{lstlisting}
Mammals: [Dog, Cat, Horse]
Animals: [Crocodile, Dog, Cat, Horse]
\end{lstlisting}%
\lthtmlfigureZ
\lthtmlcheckvsize\clearpage}

\stepcounter{subsubsection}
{\newpage\clearpage
\lthtmlfigureA{lstlisting1009}%
\begin{lstlisting}
import java.util.ArrayList;
import java.util.Arrays;
\par
class Main {
    public static void main(String[] args) {
        // Creating an array list
        ArrayList<String> animals = new ArrayList<>(Arrays.asList("Cat", "Cow", "Dog"));
        System.out.println("ArrayList: " + animals);
\par
// Access elements of the array list
        String element = animals.get(1);
        System.out.println("Accessed Element: " + element);
    }
}
\end{lstlisting}%
\lthtmlfigureZ
\lthtmlcheckvsize\clearpage}

{\newpage\clearpage
\lthtmlfigureA{lstlisting1012}%
\begin{lstlisting}
ArrayList: [Cat, Cow, Dog]
Accessed Elemenet: Cow
\end{lstlisting}%
\lthtmlfigureZ
\lthtmlcheckvsize\clearpage}

{\newpage\clearpage
\lthtmlfigureA{lstlisting1014}%
\begin{lstlisting}
new ArrayList<>(Arrays.asList(("Cat", "Cow", "Dog"));
\end{lstlisting}%
\lthtmlfigureZ
\lthtmlcheckvsize\clearpage}

\stepcounter{subsection}
\stepcounter{subsubsection}
{\newpage\clearpage
\lthtmlfigureA{lstlisting1022}%
\begin{lstlisting}
import java.util.ArrayList;
\par
class Main {
    public static void main(String[] args) {
        ArrayList<String> animals= new ArrayList<>();
\par
// Add elements in the array list
        animals.add("Dog");
        animals.add("Horse");
        animals.add("Cat");
        System.out.println("ArrayList: " + animals);
\par
// Get the element from the array list
        String str = animals.get(0);
        System.out.print("Element at index 0: " + str);
    }
}
\end{lstlisting}%
\lthtmlfigureZ
\lthtmlcheckvsize\clearpage}

{\newpage\clearpage
\lthtmlfigureA{lstlisting1025}%
\begin{lstlisting}
ArrayList: [Dog, Horse, Cat]
Element at index 0: Dog
\end{lstlisting}%
\lthtmlfigureZ
\lthtmlcheckvsize\clearpage}

\stepcounter{subsubsection}
{\newpage\clearpage
\lthtmlfigureA{lstlisting1030}%
\begin{lstlisting}
import java.util.ArrayList;
import java.util.Iterator;
\par
class Main {
    public static void main(String[] args){
        ArrayList<String> animals = new ArrayList<>();
\par
// Add elements in the array list
        animals.add("Dog");
        animals.add("Cat");
        animals.add("Horse");
        animals.add("Zebra");
\par
// Create an object of Iterator
        Iterator<String> iterate = animals.iterator();
        System.out.print("ArrayList: ");
\par
// Use methods of Iterator to access elements
        while(iterate.hasNext()){
            System.out.print(iterate.next());
            System.out.print(", ");
        }
    }
}
\end{lstlisting}%
\lthtmlfigureZ
\lthtmlcheckvsize\clearpage}

\stepcounter{subsubsection}
{\newpage\clearpage
\lthtmlfigureA{lstlisting1038}%
\begin{lstlisting}
import java.util.ArrayList;
\par
class Main {
    public static void main(String[] args) {
        ArrayList<String> animals= new ArrayList<>();
        // Add elements in the array list
        animals.add("Dog");
        animals.add("Cat");
        animals.add("Horse");
        System.out.println("ArrayList: " + animals);
\par
// Change the element of the array list
        animals.set(2, "Zebra");
        System.out.println("Modified ArrayList: " + animals);
    }
}
\end{lstlisting}%
\lthtmlfigureZ
\lthtmlcheckvsize\clearpage}

{\newpage\clearpage
\lthtmlfigureA{lstlisting1041}%
\begin{lstlisting}
ArrayList: [Dog, Cat, Horse]
Modified ArrayList: [Dog, Cat, Zebra]
\end{lstlisting}%
\lthtmlfigureZ
\lthtmlcheckvsize\clearpage}

\stepcounter{subsection}
\stepcounter{subsubsection}
{\newpage\clearpage
\lthtmlfigureA{lstlisting1046}%
\begin{lstlisting}
import java.util.ArrayList;
\par
class Main {
    public static void main(String[] args) {
        ArrayList<String> animals = new ArrayList<>();
\par
// Add elements in the array list
        animals.add("Dog");
        animals.add("Cat");
        animals.add("Horse");
        System.out.println("Initial ArrayList: " + animals);
\par
// Remove element from index 2
        String str = animals.remove(2);
        System.out.println("Final ArrayList: " + animals);
        System. out.println("Removed Element: " + str);
    }
}
\end{lstlisting}%
\lthtmlfigureZ
\lthtmlcheckvsize\clearpage}

{\newpage\clearpage
\lthtmlfigureA{lstlisting1049}%
\begin{lstlisting}
Initial ArrayList: [Dog, Cat, Horse]
Final ArrayList: [Dog, Cat]
Removed Element: Horse
\end{lstlisting}%
\lthtmlfigureZ
\lthtmlcheckvsize\clearpage}

\stepcounter{subsubsection}
{\newpage\clearpage
\lthtmlfigureA{lstlisting1053}%
\begin{lstlisting}
import java.util.ArrayList;
\par
class Main {
    public static void main(String[] args) {
        ArrayList<String> animals = new ArrayList<>();
\par
// Add elements in the ArrayList
        animals.add("Dog");
        animals.add("Cat");
        animals.add("Horse");
        System.out.println("Initial ArrayList: " + animals);
\par
// Remove all the elements
        animals.removeAll(animals);
        System.out.println("Final ArrayList: " + animals);
    }
}
\end{lstlisting}%
\lthtmlfigureZ
\lthtmlcheckvsize\clearpage}

{\newpage\clearpage
\lthtmlfigureA{lstlisting1056}%
\begin{lstlisting}
Initial ArrayList: [Dog, Cat, Horse]
Final ArrayList: []
\end{lstlisting}%
\lthtmlfigureZ
\lthtmlcheckvsize\clearpage}

\stepcounter{subsubsection}
{\newpage\clearpage
\lthtmlfigureA{lstlisting1060}%
\begin{lstlisting}
import java.util.ArrayList;
\par
class Main {
    public static void main(String[] args) {
        ArrayList<String> animals= new ArrayList<>();
\par
// Add elements in the array list
        animals.add("Dog");
        animals.add("Cat");
        animals.add("Horse");
        System.out.println("Initial ArrayList: " + animals);
\par
// Remove all the elements
        animals.clear();
        System.out.println("Final ArrayList: " + animals);
    }
}
\end{lstlisting}%
\lthtmlfigureZ
\lthtmlcheckvsize\clearpage}

\stepcounter{subsection}
\stepcounter{subsubsection}
{\newpage\clearpage
\lthtmlfigureA{lstlisting1067}%
\begin{lstlisting}
import java.util.ArrayList;
\par
class Main {
    public static void main(String[] args) {
        // Creating an array list
        ArrayList<String> animals = new ArrayList<>();
        animals.add("Cow");
        animals.add("Cat");
        animals.add("Dog");
        System.out.println("ArrayList: " + animals);
\par
// Using for loop
        System.out.println("Accessing individual elements: ");
\par
for(int i = 0; i < animals.size(); i++) {
            System.out.print(animals.get(i));
            System.out.print(", ");
        }
    }
}
\end{lstlisting}%
\lthtmlfigureZ
\lthtmlcheckvsize\clearpage}

{\newpage\clearpage
\lthtmlfigureA{lstlisting1070}%
\begin{lstlisting}
ArrayList: [Cow, Cat, Dog]
Accessing individual elements:
Cow, Cat, Dog,
\end{lstlisting}%
\lthtmlfigureZ
\lthtmlcheckvsize\clearpage}

\stepcounter{subsubsection}
{\newpage\clearpage
\lthtmlfigureA{lstlisting1073}%
\begin{lstlisting}
import java.util.ArrayList;
\par
class Main {
    public static void main(String[] args) {
        // Creating an array list
        ArrayList<String> animals = new ArrayList<>();
        animals.add("Cow");
        animals.add("Cat");
        animals.add("Dog");
        System.out.println("ArrayList: " + animals);
\par
// Using forEach loop
        System.out.println("Accessing individual elements:  ");
        for(String animal : animals) {
            System.out.print(animal);
            System.out.print(", ");
        }
    }
}
\end{lstlisting}%
\lthtmlfigureZ
\lthtmlcheckvsize\clearpage}

\stepcounter{subsubsection}
{\newpage\clearpage
\lthtmlfigureA{lstlisting1080}%
\begin{lstlisting}
import java.util.ArrayList;
\par
class Main {
    public static void main(String[] args) {
        ArrayList<String> animals= new ArrayList<>();
\par
// Adding elements in the arrayList
        animals.add("Dog");
        animals.add("Horse");
        animals.add("Cat");
        System.out.println("ArrayList: " + animals);
\par
// getting the size of the arrayList
        System.out.println("Size: " + animals.size());
    }
}
\end{lstlisting}%
\lthtmlfigureZ
\lthtmlcheckvsize\clearpage}

{\newpage\clearpage
\lthtmlfigureA{lstlisting1083}%
\begin{lstlisting}
ArrayList: [Dog, Horse, Cat]
Size: 3
\end{lstlisting}%
\lthtmlfigureZ
\lthtmlcheckvsize\clearpage}

\stepcounter{subsubsection}
{\newpage\clearpage
\lthtmlfigureA{lstlisting1089}%
\begin{lstlisting}
import java.util.ArrayList;
import java.util.Collections;
\par
class Main {
    public static void main(String[] args){
        ArrayList<String> animals= new ArrayList<>();
\par
// Add elements in the array list
        animals.add("Horse");
        animals.add("Zebra");
        animals.add("Dog");
        animals.add("Cat");
\par
System.out.println("Unsorted ArrayList: " + animals);
\par
// Sort the array list
        Collections.sort(animals);
        System.out.println("Sorted ArrayList: " + animals);
    }
}
\end{lstlisting}%
\lthtmlfigureZ
\lthtmlcheckvsize\clearpage}

{\newpage\clearpage
\lthtmlfigureA{lstlisting1092}%
\begin{lstlisting}
Unsorted ArrayList: [Horse, Zebra, Dog, Cat]
Sorted ArrayList: [Cat, Dog, Horse, Zebra]
\end{lstlisting}%
\lthtmlfigureZ
\lthtmlcheckvsize\clearpage}

\stepcounter{subsubsection}
{\newpage\clearpage
\lthtmlfigureA{lstlisting1096}%
\begin{lstlisting}
import java.util.ArrayList;
\par
class Main {
    public static void main(String[] args) {
        ArrayList<String> animals= new ArrayList<>();
\par
// Add elements in the array list
        animals.add("Dog");
        animals.add("Cat");
        animals.add("Horse");
        System.out.println("ArrayList: " + animals);
\par
// Create a new array of String type
        String[] arr = new String[animals.size()];
\par
// Convert ArrayList into an array
        animals.toArray(arr);
        System.out.print("Array: ");
        for(String item:arr) {
            System.out.print(item+", ");
        }
    }
}
\end{lstlisting}%
\lthtmlfigureZ
\lthtmlcheckvsize\clearpage}

{\newpage\clearpage
\lthtmlfigureA{lstlisting1099}%
\begin{lstlisting}
ArrayList: [Dog, Cat, Horse]
Array: Dog, Cat, Horse,
\end{lstlisting}%
\lthtmlfigureZ
\lthtmlcheckvsize\clearpage}

\stepcounter{subsubsection}
{\newpage\clearpage
\lthtmlfigureA{lstlisting1106}%
\begin{lstlisting}
import java.util.ArrayList;
import java.util.Arrays;
\par
class Main {
    public static void main(String[] args) {
        // Create an array of String type
        String[] arr = {"Dog", "Cat", "Horse"};
        System.out.print("Array: ");
\par
// Print array
        for(String str: arr) {
            System.out.print(str);
            System.out.print(" ");
        }
\par
// Create an ArrayList from an array
        ArrayList<String> animals = new ArrayList<>(Arrays.asList(arr));
        System.out.println("\nArrayList: " + animals);
    }
}
\end{lstlisting}%
\lthtmlfigureZ
\lthtmlcheckvsize\clearpage}

{\newpage\clearpage
\lthtmlfigureA{lstlisting1110}%
\begin{lstlisting}
Array: Dog, Cat, Horse
ArrayList: [Dog, Cat, Horse]
\par
\end{lstlisting}%
\lthtmlfigureZ
\lthtmlcheckvsize\clearpage}

\stepcounter{subsubsection}
{\newpage\clearpage
\lthtmlfigureA{lstlisting1117}%
\begin{lstlisting}
import java.util.ArrayList;
\par
class Main {
    public static void main(String[] args) {
        ArrayList<String> animals = new ArrayList<>();
\par
// Add elements in the ArrayList
        animals.add("Dog");
        animals.add("Cat");
        animals.add("Horse");
        System.out.println("ArrayList: " + animals);
\par
// Convert ArrayList into an String
        String str = animals.toString();
        System.out.println("String: " + str);
    }
}
\end{lstlisting}%
\lthtmlfigureZ
\lthtmlcheckvsize\clearpage}

{\newpage\clearpage
\lthtmlfigureA{lstlisting1120}%
\begin{lstlisting}
ArrayList: [Dog, Cat, Horse]
String: [Dog, Cat, Horse]
\end{lstlisting}%
\lthtmlfigureZ
\lthtmlcheckvsize\clearpage}

\stepcounter{subsection}
\stepcounter{subsubsection}
\stepcounter{subsubsection}
{\newpage\clearpage
\lthtmlfigureA{lstlisting1149}%
\begin{lstlisting}
Vector<Type> vector = new Vector<>();
\end{lstlisting}%
\lthtmlfigureZ
\lthtmlcheckvsize\clearpage}

{\newpage\clearpage
\lthtmlfigureA{lstlisting1151}%
\begin{lstlisting}
// create Integer type linked list
Vector<Integer> vector= new Vector<>();
\par
// create String type linked list
Vector<String> vector= new Vector<>();
\end{lstlisting}%
\lthtmlfigureZ
\lthtmlcheckvsize\clearpage}

\stepcounter{subsubsection}
{\newpage\clearpage
\lthtmlfigureA{lstlisting1164}%
\begin{lstlisting}
import java.util.Vector;
\par
class Main {
    public static void main(String[] args) {
        Vector<String> mammals= new Vector<>();
\par
// Using the add() method
        mammals.add("Dog");
        mammals.add("Horse");
\par
// Using index number
        mammals.add(2, "Cat");
        System.out.println("Vector: " + mammals);
\par
// Using addAll()
        Vector<String> animals = new Vector<>();
        animals.add("Crocodile");
\par
animals.addAll(mammals);
        System.out.println("New Vector: " + animals);
    }
}
\end{lstlisting}%
\lthtmlfigureZ
\lthtmlcheckvsize\clearpage}

{\newpage\clearpage
\lthtmlfigureA{lstlisting1167}%
\begin{lstlisting}
Vector: [Dog, Horse, Cat]
New Vector: [Crocodile, Dog, Horse, Cat]
\end{lstlisting}%
\lthtmlfigureZ
\lthtmlcheckvsize\clearpage}

{\newpage\clearpage
\lthtmlfigureA{lstlisting1174}%
\begin{lstlisting}
import java.util.Iterator;
import java.util.Vector;
\par
class Main {
    public static void main(String[] args) {
        Vector<String> animals= new Vector<>();
        animals.add("Dog");
        animals.add("Horse");
        animals.add("Cat");
\par
// Using get()
        String element = animals.get(2);
        System.out.println("Element at index 2: " + element);
\par
// Using iterator()
        Iterator<String> iterate = animals.iterator();
        System.out.print("Vector: ");
        while(iterate.hasNext()) {
            System.out.print(iterate.next());
            System.out.print(", ");
        }
    }
}
\end{lstlisting}%
\lthtmlfigureZ
\lthtmlcheckvsize\clearpage}

{\newpage\clearpage
\lthtmlfigureA{lstlisting1177}%
\begin{lstlisting}
Element at index 2: Cat
Vector: Dog, Horse, Cat,
\end{lstlisting}%
\lthtmlfigureZ
\lthtmlcheckvsize\clearpage}

{\newpage\clearpage
\lthtmlfigureA{lstlisting1185}%
\begin{lstlisting}
import java.util.Vector;
\par
class Main {
    public static void main(String[] args) {
        Vector<String> animals= new Vector<>();
        animals.add("Dog");
        animals.add("Horse");
        animals.add("Cat");
\par
System.out.println("Initial Vector: " + animals);
\par
// Using remove()
        String element = animals.remove(1);
        System.out.println("Removed Element: " + element);
        System.out.println("New Vector: " + animals);
\par
// Using clear()
        animals.clear();
        System.out.println("Vector after clear(): " + animals);
    }
}
\end{lstlisting}%
\lthtmlfigureZ
\lthtmlcheckvsize\clearpage}

{\newpage\clearpage
\lthtmlfigureA{lstlisting1188}%
\begin{lstlisting}
Initial Vector: [Dog, Horse, Cat]
Removed Element: Horse
New Vector: [Dog, Cat]
Vector after clear(): []
\end{lstlisting}%
\lthtmlfigureZ
\lthtmlcheckvsize\clearpage}

\stepcounter{subsubsection}
\stepcounter{subsection}
\stepcounter{subsubsection}
{\newpage\clearpage
\lthtmlfigureA{lstlisting1212}%
\begin{lstlisting}
Stack<Type> stacks = new Stack<>();
\end{lstlisting}%
\lthtmlfigureZ
\lthtmlcheckvsize\clearpage}

{\newpage\clearpage
\lthtmlfigureA{lstlisting1214}%
\begin{lstlisting}
// Create Integer type stack
Stack<Integer> stacks = new Stack<>();
\par
// Create String type stack
Stack<String> stacks = new Stack<>();
\end{lstlisting}%
\lthtmlfigureZ
\lthtmlcheckvsize\clearpage}

\stepcounter{subsubsection}
\stepcounter{subsubsection}
{\newpage\clearpage
\lthtmlfigureA{lstlisting1226}%
\begin{lstlisting}
import java.util.Stack;
\par
class Main {
    public static void main(String[] args) {
        Stack<String> animals= new Stack<>();
\par
// Add elements to Stack
        animals.push("Dog");
        animals.push("Horse");
        animals.push("Cat");
\par
System.out.println("Stack: " + animals);
    }
}
\end{lstlisting}%
\lthtmlfigureZ
\lthtmlcheckvsize\clearpage}

{\newpage\clearpage
\lthtmlfigureA{lstlisting1229}%
\begin{lstlisting}
Stack: [Dog, Horse, Cat]
\end{lstlisting}%
\lthtmlfigureZ
\lthtmlcheckvsize\clearpage}

\stepcounter{subsubsection}
{\newpage\clearpage
\lthtmlfigureA{lstlisting1233}%
\begin{lstlisting}
import java.util.Stack;
\par
class Main {
    public static void main(String[] args) {
        Stack<String> animals= new Stack<>();
\par
// Add elements to Stack
        animals.push("Dog");
        animals.push("Horse");
        animals.push("Cat");
        System.out.println("Initial Stack: " + animals);
\par
// Remove element stacks
        String element = animals.pop();
        System.out.println("Removed Element: " + element);
    }
}
\end{lstlisting}%
\lthtmlfigureZ
\lthtmlcheckvsize\clearpage}

\stepcounter{subsubsection}
{\newpage\clearpage
\lthtmlfigureA{lstlisting1238}%
\begin{lstlisting}
import java.util.Stack;
\par
class Main {
    public static void main(String[] args) {
        Stack<String> animals= new Stack<>();
\par
// Add elements to Stack
        animals.push("Dog");
        animals.push("Horse");
        animals.push("Cat");
        System.out.println("Stack: " + animals);
\par
// Access element from the top
        String element = animals.peek();
        System.out.println("Element at top: " + element);
\par
}
}
\end{lstlisting}%
\lthtmlfigureZ
\lthtmlcheckvsize\clearpage}

{\newpage\clearpage
\lthtmlfigureA{lstlisting1241}%
\begin{lstlisting}
Stack: [Dog, Horse, Cat]
Element at top: Cat
\end{lstlisting}%
\lthtmlfigureZ
\lthtmlcheckvsize\clearpage}

\stepcounter{subsubsection}
{\newpage\clearpage
\lthtmlfigureA{lstlisting1245}%
\begin{lstlisting}
import java.util.Stack;
\par
class Main {
    public static void main(String[] args) {
        Stack<String> animals= new Stack<>();
\par
// Add elements to Stack
        animals.push("Dog");
        animals.push("Horse");
        animals.push("Cat");
        System.out.println("Stack: " + animals);
\par
// Search an element
        int position = animals.search("Horse");
        System.out.println("Position of Horse: " + position);
    }
}
\end{lstlisting}%
\lthtmlfigureZ
\lthtmlcheckvsize\clearpage}

{\newpage\clearpage
\lthtmlfigureA{lstlisting1248}%
\begin{lstlisting}
Stack: [Dog, Horse, Cat]
Position of Horse: 2
\end{lstlisting}%
\lthtmlfigureZ
\lthtmlcheckvsize\clearpage}

\stepcounter{subsubsection}
{\newpage\clearpage
\lthtmlfigureA{lstlisting1252}%
\begin{lstlisting}
import java.util.Stack;
\par
class Main {
    public static void main(String[] args) {
        Stack<String> animals= new Stack<>();
\par
// Add elements to Stack
        animals.push("Dog");
        animals.push("Horse");
        animals.push("Cat");
        System.out.println("Stack: " + animals);
\par
// Check if stack is empty
        boolean result = animals.empty();
        System.out.println("Is the stack empty? " + result);
    }
}
\end{lstlisting}%
\lthtmlfigureZ
\lthtmlcheckvsize\clearpage}

{\newpage\clearpage
\lthtmlfigureA{lstlisting1255}%
\begin{lstlisting}
Stack: [Dog, Horse, Cat]
Is the stack empty? false
\end{lstlisting}%
\lthtmlfigureZ
\lthtmlcheckvsize\clearpage}

\stepcounter{section}
\stepcounter{subsection}
{\newpage\clearpage
\lthtmlfigureA{lstlisting1264}%
\begin{lstlisting}
// LinkedList implementation of Queue
Queue <String> animal1 = new LinkedList<>();
\par
// Array implementation of Queue
Queue <String> animal2 = new ArrayDeque<>();
\par
// Priority Queue implementation of Queue
Queue <String> animal3 = new PriorityQueue<>();
\end{lstlisting}%
\lthtmlfigureZ
\lthtmlcheckvsize\clearpage}

\stepcounter{subsection}
\stepcounter{subsection}
\stepcounter{subsubsection}
{\newpage\clearpage
\lthtmlfigureA{lstlisting1285}%
\begin{lstlisting}
import java.util.Queue;
import java.util.LinkedList;
\par
class Main {
	public static void main(String[] args) {
		// Creating Queue using the LinkedList class
		Queue <Integer> numbers = new LinkedList<>();
\par
// offer elements to the Queue
		numbers.offer(1);
		numbers.offer(2);
		numbers.offer(3);
		System.out.println("Queue: " + numbers);
\par
// Access elements of the Queue
		int accessedNumber = numbers.peek();
		System.out.println("Accessed Element: " + accessedNumber);
\par
// Remove elements from the Queue
		int removedNumber = numbers.poll();
		System.out.println("Removed Element: " + removedNumber);
		System.out.println("Updated Queue: " + numbers);
	}
}
\end{lstlisting}%
\lthtmlfigureZ
\lthtmlcheckvsize\clearpage}

{\newpage\clearpage
\lthtmlfigureA{lstlisting1288}%
\begin{lstlisting}
1 Queue : [1 , 2, 3]
2 Accessed Element : 1
3 Removed Element : 1
4 Updated Queue : [2 , 3]
\end{lstlisting}%
\lthtmlfigureZ
\lthtmlcheckvsize\clearpage}

{\newpage\clearpage
\lthtmlfigureA{lstlisting1293}%
\begin{lstlisting}
import java.util.PriorityQueue;
import java.util.Iterator;
\par
class Main {
	public static void main(String[] args) {
		// Creating a priority queue
		PriorityQueue <Integer> numbers = new PriorityQueue<>();
		numbers.add(4);
		numbers.add(2);
		numbers.add(1);
		System.out.print("PriorityQueue using iterator(): ");
		// Using the iterator () method
		Iterator <Integer> iterate = numbers.iterator();
		while (iterate.hasNext()) {
			System.out.print(iterate.next());
			System.out.print(", ");
		}
	}
}
\end{lstlisting}%
\lthtmlfigureZ
\lthtmlcheckvsize\clearpage}

\stepcounter{subsubsection}
{\newpage\clearpage
\lthtmlfigureA{lstlisting1300}%
\begin{lstlisting}
// Array implementation of Deque
Deque <String> animal1 = new ArrayDeque<>();
\par
// LinkedList implementation of Deque
Deque <String> animal2 = new LinkedList<>();
\end{lstlisting}%
\lthtmlfigureZ
\lthtmlcheckvsize\clearpage}

\stepcounter{subsubsection}
{\newpage\clearpage
\lthtmlfigureA{lstlisting1322}%
\begin{lstlisting}
import java.util.Deque;
import java.util.ArrayDeque ;
\par
class Main {
	public static void main (String[] args) {
		// Creating Deque using the ArrayDeque class
		Deque <Integer> numbers = new ArrayDeque<>();
\par
// add elements to the Deque
		numbers.offer(1);
		numbers.offerLast(2);
		numbers.offerFirst(3);
		System.out.println("Deque :" + numbers);
		// Access elements of the Deque
		int firstElement = numbers.peekFirst();
		System.out.println("First Element : " + firstElement);
\par
int lastElement = numbers.peekLast();
		System.out.println("Last Element : " + lastElement);
		// Remove elements from the Deque
		int removedNumber1 = numbers.pollFirst();
		System.out.println("Removed First Element : " + removedNumber1);
		int removedNumber2 = numbers.pollLast();
		System.out.println("Removed Last Element : " + removedNumber2);
		System.out.println("Updated Deque :" + numbers);
	}
}
\end{lstlisting}%
\lthtmlfigureZ
\lthtmlcheckvsize\clearpage}

{\newpage\clearpage
\lthtmlfigureA{lstlisting1325}%
\begin{lstlisting}
Deque : [3 , 1, 2]
First Element : 3
Last Element : 2
Removed First Element : 3
Removed Last Element : 2
Updated Deque : [1]
\end{lstlisting}%
\lthtmlfigureZ
\lthtmlcheckvsize\clearpage}

\stepcounter{section}
{\newpage\clearpage
\lthtmlfigureA{lstlisting1331}%
\begin{lstlisting}
// Map implementation using HashMap
Map <Key , Value> numbers = new HashMap<>();
\end{lstlisting}%
\lthtmlfigureZ
\lthtmlcheckvsize\clearpage}

\stepcounter{subsection}
\stepcounter{subsubsection}
{\newpage\clearpage
\lthtmlfigureA{lstlisting1366}%
\begin{lstlisting}
import java.util.Map;
import java.util.HashMap;
\par
class Main {
	public static void main (String[] args) {
		// Creating a map using the HashMap
		Map <String, Integer> numbers = new HashMap<>();
		// Insert elements to the map
		numbers.put("One ", 1);
		numbers.put("Two ", 2);
		System.out.println(" Map : " + numbers);
		// Access keys of the map
		System.out.println("Keys : " + numbers.keySet());
		// Access values of the map
		System.out.println("Values : " + numbers.values());
		// Access entries of the map
		System.out.println("Entries : " + numbers.entrySet());
		// Remove Elements from the map
		int value = numbers.remove("Two ");
		System.out.println("Removed Value : " + value);
	}
}
\end{lstlisting}%
\lthtmlfigureZ
\lthtmlcheckvsize\clearpage}

{\newpage\clearpage
\lthtmlfigureA{lstlisting1369}%
\begin{lstlisting}
Map : { One =1 , Two =2}
Keys : [ One , Two ]
Values : [1 , 2]
Entries : [ One =1 , Two =2]
Removed Value : 2
\end{lstlisting}%
\lthtmlfigureZ
\lthtmlcheckvsize\clearpage}

\stepcounter{subsubsection}
{\newpage\clearpage
\lthtmlfigureA{lstlisting1373}%
\begin{lstlisting}
import java.util.Map;
import java.util.TreeMap;
\par
class Main {
	public static void main ( String [] args ) {
		// Creating Map using TreeMap
		Map <String, Integer> values = new TreeMap<>();
\par
// Insert elements to map
		values.put("Second ", 2);
		values.put("First ", 1);
		System.out.println(" Map using TreeMap : " + values);
\par
// Replacing the values
		values.replace("First ", 11);
		values.replace("Second ", 22);
		System.out.println("New Map : " + values);
\par
// Remove elements from the map
		int removedValue = values.remove("First ");
		System.out.println(" Removed Value : " + removedValue);
	}
}
\end{lstlisting}%
\lthtmlfigureZ
\lthtmlcheckvsize\clearpage}

{\newpage\clearpage
\lthtmlfigureA{lstlisting1376}%
\begin{lstlisting}
Map using TreeMap : { First =1 , Second =2}
New Map : { First =11 , Second =22}
Removed Value : 11
\end{lstlisting}%
\lthtmlfigureZ
\lthtmlcheckvsize\clearpage}

\stepcounter{subsection}
\stepcounter{subsubsection}
{\newpage\clearpage
\lthtmlfigureA{lstlisting1388}%
\begin{lstlisting}
import java . util . HashMap ;
\par
class Main {
	public static void main(String[] args) {
		// Creating HashMap of even numbers
		HashMap <String, Integer> evenNumbers = new HashMap<>();
		// Using put ()
		evenNumbers.put("Two ", 2);
		evenNumbers.put("Four ", 4);
\par
// Using putIfAbsent ()
		evenNumbers.putIfAbsent("Six ", 6);
		System.out.println("HashMap of even numbers : " + evenNumbers);
		// Creating HashMap of numbers
		HashMap <String, Integer> numbers = new HashMap<>();
		numbers.put("One ", 1);
\par
// Using putAll ()
		numbers.putAll(evenNumbers);
		System.out.println("HashMap of numbers : " + numbers);
	}
}
\end{lstlisting}%
\lthtmlfigureZ
\lthtmlcheckvsize\clearpage}

{\newpage\clearpage
\lthtmlfigureA{lstlisting1391}%
\begin{lstlisting}
HashMap of even numbers : { Six =6 , Four =4 , Two =2}
HashMap of numbers : { Six =6 , One =1 , Four =4 , Two =2}
\end{lstlisting}%
\lthtmlfigureZ
\lthtmlcheckvsize\clearpage}

\stepcounter{subsubsection}
{\newpage\clearpage
\lthtmlfigureA{lstlisting1401}%
\begin{lstlisting}
import java.util.HashMap;
class Main {
	public static void main(String[] args) {
		HashMap <String, Integer> numbers = new HashMap<>();
		numbers.put("One ", 1);
		numbers.put("Two ", 2);
		numbers.put("Three ", 3);
		System.out.println("HashMap : " + numbers);
\par
// Using entrySet ()
		System.out.println("Key / Value mappings : " + numbers.entrySet());
		// Using keySet ()
		System.out.println("Keys : " + numbers.keySet());
		// Using values ()
		System.out.println("Values : " + numbers.values());
	}
}
\end{lstlisting}%
\lthtmlfigureZ
\lthtmlcheckvsize\clearpage}

{\newpage\clearpage
\lthtmlfigureA{lstlisting1404}%
\begin{lstlisting}
HashMap : { One =1 , Two =2 , Three =3}
Key / Value mappings : [ One =1 , Two =2 , Three =3]
Keys : [ One , Two , Three ]
Values : [1 , 2, 3]
\end{lstlisting}%
\lthtmlfigureZ
\lthtmlcheckvsize\clearpage}

{\newpage\clearpage
\lthtmlfigureA{lstlisting1413}%
\begin{lstlisting}
import java.util.HashMap;
class Main {
	public static void main (String[] args) {
		HashMap <String, Integer> numbers = new HashMap<>();
		numbers.put("One ", 1);
		numbers.put("Two ", 2);
		numbers.put("Three ", 3);
		System.out.println("HashMap : " + numbers);
\par
// Using get ()
		int value1 = numbers.get("Three ");
		System.out.println(" Returned Number : " + value1);
\par
// Using getOrDefault ()
		int value2 = numbers.getOrDefault("Five ", 5);
		System.out.println("Returned Number : " + value2);
	}
}
\end{lstlisting}%
\lthtmlfigureZ
\lthtmlcheckvsize\clearpage}

{\newpage\clearpage
\lthtmlfigureA{lstlisting1416}%
\begin{lstlisting}
HashMap : { One =1 , Two =2 , Three =3}
Returned Number : 3
Returned Number : 5
\end{lstlisting}%
\lthtmlfigureZ
\lthtmlcheckvsize\clearpage}

\stepcounter{subsubsection}
{\newpage\clearpage
\lthtmlfigureA{lstlisting1424}%
\begin{lstlisting}
import java.util.HashMap;
\par
class Main {
	public static void main(String[] args) {
		HashMap <String, Integer> numbers = new HashMap<>();
		numbers.put("One ", 1);
		numbers.put("Two ", 2);
		numbers.put("Three ", 3);
		System.out.println("HashMap : " + numbers);
\par
// remove method with single parameter
		int value = numbers.remove("Two ");
		System.out.println("Removed value : " + value );
\par
// remove method with two parameters
		boolean result = numbers.remove("Three ", 3);
		System.out.println("Is the entry Three removed ?" + result );
		System.out.println("Updated HashMap: " + numbers);
	}
}
\end{lstlisting}%
\lthtmlfigureZ
\lthtmlcheckvsize\clearpage}

{\newpage\clearpage
\lthtmlfigureA{lstlisting1427}%
\begin{lstlisting}
HashMap : { One =1 , Two =2 , Three =3}
Removed value : 2
Is the entry Three removed ? True
Updated HashMap : { One =1}
\end{lstlisting}%
\lthtmlfigureZ
\lthtmlcheckvsize\clearpage}

\stepcounter{subsubsection}
{\newpage\clearpage
\lthtmlfigureA{lstlisting1437}%
\begin{lstlisting}
import java.util.HashMap ;
\par
class Main {
	public static void main(String[] args) {
\par
HashMap <String, Integer> numbers = new HashMap<>();
		numbers.put("First ", 1);
		numbers.put("Second ", 2);
		numbers.put("Third ", 3) ;
		System.out.println("Original HashMap : " + numbers);
\par
// Using replace ()
		numbers.replace("Second ", 22);
		numbers.replace("Third ", 3, 33);
		System.out.println("HashMap using replace (): " + numbers);
\par
// Using replaceAll ()
		numbers.replaceAll((key , oldValue) -> oldValue + );
		System.out.println("HashMap using replaceAll ():" + numbers);
	}
}
\end{lstlisting}%
\lthtmlfigureZ
\lthtmlcheckvsize\clearpage}

{\newpage\clearpage
\lthtmlfigureA{lstlisting1440}%
\begin{lstlisting}
Original HashMap : { Second =2 , Third =3 , First =1}
HashMap using replace : { Second =22 , Third =33 , First =1}
HashMap using replaceAll : { Second =24 , Third =35 , First =3}
\end{lstlisting}%
\lthtmlfigureZ
\lthtmlcheckvsize\clearpage}

\stepcounter{subsubsection}
{\newpage\clearpage
\lthtmlfigureA{lstlisting1453}%
\begin{lstlisting}
import java.util.HashMap;
\par
class Main {
	public static void main(String[] args) {
\par
HashMap <String, Integer> numbers = new HashMap<>();
		numbers.put("First ", 1);
		numbers.put("Second ", 2);
		System.out.println("Original HashMap: " +numbers );
\par
// Using compute ()
		numbers.compute("First ", (key, oldValue) ->oldValue + 2);
		numbers.compute("Second ", (key , oldValue) ->oldValue + 1);
		System.out.println("HashMap using compute (): " +numbers);
\par
// Using computeIfAbsent ()
		numbers.computeIfAbsent("Three ", key -> 5);
		System . out . println (" HashMap using
		computeIfAbsent (): " + numbers );
\par
// Using computeIfPresent ()
		numbers.computeIfPresent("Second ", (key ,oldValue ) -> oldValue * 2);
		System.out.println("HashMap using computeIfPresent(): " + numbers);
	}
}
\end{lstlisting}%
\lthtmlfigureZ
\lthtmlcheckvsize\clearpage}

{\newpage\clearpage
\lthtmlfigureA{lstlisting1456}%
\begin{lstlisting}
Original HashMap : {Second =2 , First =1}
HashMap using compute () : {Second=3, First =3}
HashMap using computeIfAbsent () : { Second=3 First =3, Three =5}
HashMap using computeIfPresent () : {Second =6, First =3, three =5}
\end{lstlisting}%
\lthtmlfigureZ
\lthtmlcheckvsize\clearpage}

\stepcounter{subsubsection}
{\newpage\clearpage
\lthtmlfigureA{lstlisting1463}%
\begin{lstlisting}
import java.util.HashMap;
\par
class Main {
	public static void main(String[] args) {
\par
HashMap<String, Integer> numbers = new HashMap<>();
		numbers.put("First ", 1);
		numbers.put("Second ", 2);
		System.out.println("Original HashMap : " +numbers);
\par
// Using merge () Method
		numbers.merge("First ", 4, (oldValue , newValue)-> oldValue + newValue);
		System.out.println("New HashMap : " + numbers);
	}
}
\end{lstlisting}%
\lthtmlfigureZ
\lthtmlcheckvsize\clearpage}

{\newpage\clearpage
\lthtmlfigureA{lstlisting1466}%
\begin{lstlisting}
1 Original HashMap : { Second =2 , First =1}
2 New HashMap : { Second =2 , First =5}
\end{lstlisting}%
\lthtmlfigureZ
\lthtmlcheckvsize\clearpage}

\stepcounter{subsubsection}
{\newpage\clearpage
\lthtmlfigureA{lstlisting1489}%
\begin{lstlisting}
import java.util.HashMap;
import java.util.Map.Entry;
\par
class Main {
	public static void main (String[] args) {
\par
// Creating a HashMap
		HashMap<String, Integer> numbers = new HashMap<>();
		numbers.put("One ", 1);
		numbers.put("Two ", 2);
		numbers.put("Three ", 3);
		System.out.println("HashMap: " + numbers);
\par
// Accessing the key/ value pair
		System.out.print("Entries : ");
		for(Entry<String, Integer> entry:numbers.entrySet()) {
			System . out . print ( entry );
			System . out . print (", ");
		}
\par
// Accessing the key
		System.out.print("\n Keys : ");
\par
System.out.print(key);
		System.out.print(", ");
	}
\par
// Accessing the value
	System.out.print("\n Values : ");
	for(Integer value: numbers.values()) {
		System.out.print(value);
		System.out.print(", ");
	}
}
\end{lstlisting}%
\lthtmlfigureZ
\lthtmlcheckvsize\clearpage}

{\newpage\clearpage
\lthtmlfigureA{lstlisting1493}%
\begin{lstlisting}
HashMap : { One =1 , Two =2 , Three =3}
Entries : One =1 , Two =2 , Three =3
Keys : One , Two , Three ,
Values : 1, 2, ,3,
\end{lstlisting}%
\lthtmlfigureZ
\lthtmlcheckvsize\clearpage}

{\newpage\clearpage
\lthtmlfigureA{lstlisting1501}%
\begin{lstlisting}
import java.util.HashMap;
import java.util.Iterator;
import java.util.Map.Entry;
\par
class Main {
	public static void main(String[] args) {
		// Creating a HashMap
		HashMap<String, Integer> numbers = new HashMap<>();
		numbers.put("One ", 1);
		numbers.put("Two ", 2);
		numbers.put("Three ", 3);
		System.out.println("HashMap : " + numbers );
\par
// Creating an object of Iterator
		Iterator<Entry<String, Integer>> iterate1 =numbers.entrySet().iterator();
		// Accessing the Key/ Value pair
		System.out.print("Entries : ");
		while( iterate1.hasNext()) {
		System.out.print( iterate1.next());
		System.out.print(", ");
		}
\par
// Accessing the key
		Iterator<String>iterate2 = numbers.keySet().iterator();
		System.out.print("\n Keys : ");
		while ( iterate2.hasNext()) {
		System.out.print( iterate2.next());
		System.out.print(", ");
		}
\par
// Accessing the value
		Iterator<Integer> iterate3 = numbers.values().iterator();
		System.out.print("\n Values: ");
		while ( iterate3.hasNext()) {
			System.out.print( iterate3.next());
			System.out.print(", ");
		}
	}
}
\end{lstlisting}%
\lthtmlfigureZ
\lthtmlcheckvsize\clearpage}

{\newpage\clearpage
\lthtmlfigureA{lstlisting1506}%
\begin{lstlisting}
HashMap : { One =1 , Two =2 , Three =3}
Entries : One =1 , Two =2 , Three =3
Keys : One , Two , Three ,
Values : 1, 2, 3,
\end{lstlisting}%
\lthtmlfigureZ
\lthtmlcheckvsize\clearpage}

\stepcounter{section}
\stepcounter{subsection}
\stepcounter{subsection}
\stepcounter{subsubsection}
{\newpage\clearpage
\lthtmlfigureA{lstlisting1552}%
\begin{lstlisting}
import java.util.Set;
import java.util.HashSet;
\par
class Main {
\par
public static void main(String[] args) {
		// Creating a set using the HashSet class
		Set <Integer> set1 = new HashSet<>();
\par
// Add elements to the set1
		set1.add 2);
		set1.add(3);
		System.out.println("Set1 : " + set1);
\par
// Creating another set using the HashSet class
		Set <Integer> set2 = new HashSet<>();
\par
// Add elements
		set2.add(1);
		set2.add(2);
		System.out.println("Set2 : " + set2);
\par
// Union of two sets
		set2.addAll(set1);
		System.out.println("Union is: " + set2);
	}
}
\end{lstlisting}%
\lthtmlfigureZ
\lthtmlcheckvsize\clearpage}

{\newpage\clearpage
\lthtmlfigureA{lstlisting1555}%
\begin{lstlisting}
Set1 : [2 , 3]
Set2 : [1 , 2]
Union is : [1 , 2 , 3]
\end{lstlisting}%
\lthtmlfigureZ
\lthtmlcheckvsize\clearpage}

\stepcounter{subsubsection}
{\newpage\clearpage
\lthtmlfigureA{lstlisting1558}%
\begin{lstlisting}
import java.util.Set;
import java.util.TreeSet;
import java.util.Iterator;
\par
class Main {
\par
public static void main(String[] args) {
		// Creating a set using the TreeSet class
		Set <Integer> numbers = new TreeSet<>();
\par
// Add elements to the set
		numbers.add(2);
		numbers.add(3);
		numbers.add(1);
		System.out.println("Set using TreeSet : " +numbers );
\par
// Access Elements using iterator ()
		System.out.print("Accessing elements using iterator(): ");
		Iterator <Integer> iterate = numbers.iterator();
		while ( iterate.hasNext()) {
			System.out.print( iterate.next());
			System.out.print(", ");
		}
\par
}
}
\end{lstlisting}%
\lthtmlfigureZ
\lthtmlcheckvsize\clearpage}

{\newpage\clearpage
\lthtmlfigureA{lstlisting1561}%
\begin{lstlisting}
Set using TreeSet : [1 , 2, 3]
Accessing elements using iterator () : 1, 2, 3,
\end{lstlisting}%
\lthtmlfigureZ
\lthtmlcheckvsize\clearpage}

\stepcounter{section}
{\newpage\clearpage
\lthtmlfigureA{lstlisting1567}%
\begin{lstlisting}
enum Size {
    constant1, constant2, …, constantN;
\par
// methods and fields	
}
\end{lstlisting}%
\lthtmlfigureZ
\lthtmlcheckvsize\clearpage}

{\newpage\clearpage
\lthtmlfigureA{lstlisting1576}%
\begin{lstlisting}
enum Size { 
   SMALL, MEDIUM, LARGE, EXTRALARGE 
}
\end{lstlisting}%
\lthtmlfigureZ
\lthtmlcheckvsize\clearpage}

{\newpage\clearpage
\lthtmlfigureA{lstlisting1582}%
\begin{lstlisting}
enum Size {
   SMALL, MEDIUM, LARGE, EXTRALARGE
}
\par
class Main {
   public static void main(String[] args) {
      System.out.println(Size.SMALL);
      System.out.println(Size.MEDIUM);
   }
}
\end{lstlisting}%
\lthtmlfigureZ
\lthtmlcheckvsize\clearpage}

{\newpage\clearpage
\lthtmlfigureA{lstlisting1586}%
\begin{lstlisting}
SMALL
MEDIUM
\end{lstlisting}%
\lthtmlfigureZ
\lthtmlcheckvsize\clearpage}

{\newpage\clearpage
\lthtmlfigureA{lstlisting1590}%
\begin{lstlisting}
pizzaSize = Size.SMALL;
pizzaSize = Size.MEDIUM;
pizzaSize = Size.LARGE;
pizzaSize = Size.EXTRALARGE;
\end{lstlisting}%
\lthtmlfigureZ
\lthtmlcheckvsize\clearpage}

{\newpage\clearpage
\lthtmlfigureA{lstlisting1592}%
\begin{lstlisting}
enum Size {
 SMALL, MEDIUM, LARGE, EXTRALARGE
}
\par
class Test {
 Size pizzaSize;
 public Test(Size pizzaSize) {
   this.pizzaSize = pizzaSize;
 }
 public void orderPizza() {
   switch(pizzaSize) {
     case SMALL:
       System.out.println("I ordered a small size pizza.");
       break;
     case MEDIUM:
       System.out.println("I ordered a medium size pizza.");
       break;
     default:
       System.out.println("I don't know which one to order.");
       break;
   }
 }
}
\par
class Main {
 public static void main(String[] args) {
   Test t1 = new Test(Size.MEDIUM);
   t1.orderPizza();
 }
}
\end{lstlisting}%
\lthtmlfigureZ
\lthtmlcheckvsize\clearpage}

{\newpage\clearpage
\lthtmlfigureA{lstlisting1598}%
\begin{lstlisting}
I ordered a medium size pizza.
\end{lstlisting}%
\lthtmlfigureZ
\lthtmlcheckvsize\clearpage}

{\newpage\clearpage
\lthtmlfigureA{lstlisting1608}%
\begin{lstlisting}
enum Size{
   SMALL, MEDIUM, LARGE, EXTRALARGE;
\par
public String getSize() {
\par
// this will refer to the object SMALL
      switch(this) {
         case SMALL:
          return "small";
\par
case MEDIUM:
          return "medium";
\par
case LARGE:
          return "large";
\par
case EXTRALARGE:
          return "extra large";
\par
default:
          return null;
      }
   }
\par
public static void main(String[] args) {
\par
// calling the method getSize() using the object SMALL
      System.out.println("The size of the pizza is " + Size.SMALL.getSize());
   }
}
\end{lstlisting}%
\lthtmlfigureZ
\lthtmlcheckvsize\clearpage}

\stepcounter{subsection}
\stepcounter{subsection}
{\newpage\clearpage
\lthtmlfigureA{lstlisting1656}%
\begin{lstlisting}
class Size {
   public final static int SMALL = 1;
   public final static int MEDIUM = 2;
   public final static int LARGE = 3;
   public final static int EXTRALARGE = 4;
}
\end{lstlisting}%
\lthtmlfigureZ
\lthtmlcheckvsize\clearpage}

\stepcounter{subsection}
{\newpage\clearpage
\lthtmlfigureA{lstlisting1671}%
\begin{lstlisting}
enum Size {
\par
// enum constants calling the enum constructors 
   SMALL("The size is small."),
   MEDIUM("The size is medium."),
   LARGE("The size is large."),
   EXTRALARGE("The size is extra large.");
\par
private final String pizzaSize;
\par
// private enum constructor
   private Size(String pizzaSize) {
      this.pizzaSize = pizzaSize;
   }
\par
public String getSize() {
      return pizzaSize;
   }
}
\par
class Main {
   public static void main(String[] args) {
      Size size = Size.SMALL;
      System.out.println(size.getSize());
   }
}
\end{lstlisting}%
\lthtmlfigureZ
\lthtmlcheckvsize\clearpage}

\stepcounter{subsection}
{\newpage\clearpage
\lthtmlfigureA{lstlisting1694}%
\begin{lstlisting}
enum Size {
   SMALL, MEDIUM, LARGE, EXTRALARGE
}
\par
class Main {
   public static void main(String[] args) {
\par
System.out.println("string value of SMALL is " + Size.SMALL.toString());
      System.out.println("string value of MEDIUM is " + Size.MEDIUM.name());
\par
}
}
\end{lstlisting}%
\lthtmlfigureZ
\lthtmlcheckvsize\clearpage}

{\newpage\clearpage
\lthtmlfigureA{lstlisting1698}%
\begin{lstlisting}
string value of SMALL is SMALL
string value of MEDIUM is MEDIUM
\end{lstlisting}%
\lthtmlfigureZ
\lthtmlcheckvsize\clearpage}

{\newpage\clearpage
\lthtmlfigureA{lstlisting1703}%
\begin{lstlisting}
enum Size {
   SMALL {
\par
// overriding toString() for SMALL
      public String toString() {
        return "The size is small.";
      }
   },
\par
MEDIUM {
\par
// overriding toString() for MEDIUM
      public String toString() {
        return "The size is medium.";
      }
   };
}
\par
class Main {
   public static void main(String[] args) {
      System.out.println(Size.MEDIUM.toString());
   }
}
\end{lstlisting}%
\lthtmlfigureZ
\lthtmlcheckvsize\clearpage}

\stepcounter{section}
\stepcounter{subsection}
{\newpage\clearpage
\lthtmlfigureA{lstlisting1739}%
\begin{lstlisting}
// Creates an InputStream
InputStream object1 = new FileInputStream();
\end{lstlisting}%
\lthtmlfigureZ
\lthtmlcheckvsize\clearpage}

\stepcounter{subsubsection}
{\newpage\clearpage
\lthtmlfigureA{lstlisting1759}%
\begin{lstlisting}
This is a line of text inside the file.
\end{lstlisting}%
\lthtmlfigureZ
\lthtmlcheckvsize\clearpage}

{\newpage\clearpage
\lthtmlfigureA{lstlisting1763}%
\begin{lstlisting}
 import java.io.FileInputStream;
import java.io.InputStream;
\par
public class Main {
	public static void main(String args[]) {
\par
byte[] array = new byte[100];
\par
try {
			InputStream input = new FileInputStream("input.txt ");
\par
System.out.println(" Available bytes in the file : " + input.available());
\par
// Read byte from the input stream
			input.read( array );
			System.out.println("Data read from the file :");
\par
// Convert byte array into string
			String data = new String( array );
			System.out.println( data );
\par
// Close the input stream
			input.close() ;
		}
		catch ( Exception e) {
			e.getStackTrace() ;
		}
	}
}
\end{lstlisting}%
\lthtmlfigureZ
\lthtmlcheckvsize\clearpage}

{\newpage\clearpage
\lthtmlfigureA{lstlisting1767}%
\begin{lstlisting}
Available bytes in the file : 35
Data read from the file :
This is a line of text inside the file
\end{lstlisting}%
\lthtmlfigureZ
\lthtmlcheckvsize\clearpage}

\stepcounter{subsubsection}
{\newpage\clearpage
\lthtmlfigureA{lstlisting1778}%
\begin{lstlisting}
// Creates an OutputStream
OutputStream object = new FileOutputStream();
\end{lstlisting}%
\lthtmlfigureZ
\lthtmlcheckvsize\clearpage}

\stepcounter{subsubsection}
{\newpage\clearpage
\lthtmlfigureA{lstlisting1793}%
\begin{lstlisting}
import java.io.FileOutputStream;
import java.io.OutputStream;
\par
public class Main {
\par
public static void main( String args[]) {
		String data = " This is a line of text inside the file.";
\par
try {
			OutputStream out = new FileOutputStream ("output.txt");
\par
// Converts the string into bytes
			byte [] dataBytes = data.getBytes();
\par
// Writes data to the output stream
			out.write( dataBytes );
			System.out.println(" Data is written to the file.");
\par
// Closes the output stream
			out.close();
		}
\par
catch( Exception e) {
			e.getStackTrace ();
		}
	}
}
\end{lstlisting}%
\lthtmlfigureZ
\lthtmlcheckvsize\clearpage}

\stepcounter{section}
{\newpage\clearpage
\lthtmlfigureA{lstlisting1801}%
\begin{lstlisting}
// Creates a Reader
Reader input = new FileReader();
\end{lstlisting}%
\lthtmlfigureZ
\lthtmlcheckvsize\clearpage}

\stepcounter{subsection}
{\newpage\clearpage
\lthtmlfigureA{lstlisting1818}%
\begin{lstlisting}
import java.io.Reader;
import java.io.FileReader;
\par
class Main {
	public static void main(String[] args) {
\par
// Creates an array of character
		char[] array = new char[100];
\par
try {
			// Creates a reader using the FileReader
			Reader input = new FileReader("input.txt");
\par
// Checks if reader is ready
			System.out.println("Is there data in the stream ? " + input.ready());
\par
// Reads characters
			input.read(array);
			System.out.println("Data in the stream:");
			System.out.println(array);
\par
// Closes the reader
			input.close();
		}
\par
catch( Exception e ) {
			e.getStackTrace();
		}
	}
}
\end{lstlisting}%
\lthtmlfigureZ
\lthtmlcheckvsize\clearpage}

{\newpage\clearpage
\lthtmlfigureA{lstlisting1822}%
\begin{lstlisting}
Is there data in the stream ? true
Data in the stream :
This is a line of text inside the file .
\end{lstlisting}%
\lthtmlfigureZ
\lthtmlcheckvsize\clearpage}

\stepcounter{subsection}
{\newpage\clearpage
\lthtmlfigureA{lstlisting1832}%
\begin{lstlisting}
// Creates a Writer
Writer output = new FileWriter () ;
\end{lstlisting}%
\lthtmlfigureZ
\lthtmlcheckvsize\clearpage}

\stepcounter{subsubsection}
{\newpage\clearpage
\lthtmlfigureA{lstlisting1844}%
\begin{lstlisting}
import java.io.FileWriter;
import java.io.Writer;
\par
public class Main {
	public static void main( String args[]) {
\par
String data = "This is the data in the output file";
\par
try {
			// Creates a Writer using FileWriter
			Writer output = new FileWriter("output.txt");
\par
// Writes string to the file
			output.write(data);
\par
// Closes the writer
			output.close();
		 }
\par
catch( Exception e) {
			e.getStackTrace();
		}
	}
}
\end{lstlisting}%
\lthtmlfigureZ
\lthtmlcheckvsize\clearpage}

\stepcounter{chapter}
\stepcounter{section}
\stepcounter{subsection}
{\newpage\clearpage
\lthtmlfigureA{lstlisting1861}%
\begin{lstlisting}
if (expression) {
    // statements
}
\end{lstlisting}%
\lthtmlfigureZ
\lthtmlcheckvsize\clearpage}

{\newpage\clearpage
\lthtmlfigureA{lstlisting1870}%
\begin{lstlisting}
class IfStatement {
    public static void main(String[] args) {
\par
int number = 10;
\par
// checks if number is greater than 0
        if (number > 0) {
            System.out.println("The number is positive.");
        }
        System.out.println("This statement is always executed.");
    }
}
\end{lstlisting}%
\lthtmlfigureZ
\lthtmlcheckvsize\clearpage}

{\newpage\clearpage
\lthtmlfigureA{lstlisting1873}%
\begin{lstlisting}
The number is positive.
This statement is always executed.
\end{lstlisting}%
\lthtmlfigureZ
\lthtmlcheckvsize\clearpage}

{\newpage\clearpage
\lthtmlfigureA{lstlisting1885}%
\begin{lstlisting}
class Main {
  public static void main(String[] args) {
    // create a string variable
    String language = "Java";
\par
// if statement
    if(language == "Java") {
      System.out.println("This is best programming language.");
    } 
  }
}
\end{lstlisting}%
\lthtmlfigureZ
\lthtmlcheckvsize\clearpage}

{\newpage\clearpage
\lthtmlfigureA{lstlisting1895}%
\begin{lstlisting}
if (expression) {
   // codes
}
else {
  // some other code
}
\end{lstlisting}%
\lthtmlfigureZ
\lthtmlcheckvsize\clearpage}

{\newpage\clearpage
\lthtmlfigureA{lstlisting1901}%
\begin{lstlisting}
class IfElse {
    public static void main(String[] args) {    	
        int number = 10;
\par
// checks if number is greater than 0	 
        if (number > 0) {
            System.out.println("The number is positive.");
        }
        else {
            System.out.println("The number is not positive.");
        }
\par
System.out.println("This statement is always executed.");
    }
}
\end{lstlisting}%
\lthtmlfigureZ
\lthtmlcheckvsize\clearpage}

{\newpage\clearpage
\lthtmlfigureA{lstlisting1915}%
\begin{lstlisting}
The number is not positive.
This statement is always executed.
\end{lstlisting}%
\lthtmlfigureZ
\lthtmlcheckvsize\clearpage}

{\newpage\clearpage
\lthtmlfigureA{lstlisting1921}%
\begin{lstlisting}
if (expression1) {
   // codes
}
else if(expression2) {
   // codes
}
else if (expression3) {
   // codes
}
.
.
else {
   // codes
}
\end{lstlisting}%
\lthtmlfigureZ
\lthtmlcheckvsize\clearpage}

{\newpage\clearpage
\lthtmlfigureA{lstlisting1933}%
\begin{lstlisting}
class Ladder {
    public static void main(String[] args) {   
\par
int number = 0;
\par
// checks if number is greater than 0	 
        if (number > 0) {
            System.out.println("The number is positive.");
        }
\par
// checks if number is less than 0
        else if (number < 0) {
            System.out.println("The number is negative.");
        }
        else {
            System.out.println("The number is 0.");
        } 
    }
}
\end{lstlisting}%
\lthtmlfigureZ
\lthtmlcheckvsize\clearpage}

\stepcounter{subsection}
{\newpage\clearpage
\lthtmlfigureA{lstlisting1950}%
\begin{lstlisting}
class Number {
    public static void main(String[] args) {
\par
// declaring double type variables
        Double n1 = -1.0, n2 = 4.5, n3 = -5.3, largestNumber;
\par
// checks if n1 is greater than or equal to n2
        if (n1 >= n2) {
\par
// if...else statement inside the if block
            // checks if n1 is greater than or equal to n3
            if (n1 >= n3) {
                largestNumber = n1;
            }
\par
else {
                largestNumber = n3;
            }
        }
        else {
\par
// if..else statement inside else block
            // checks if n2 is greater than or equal to n3
            if (n2 >= n3) {
                largestNumber = n2;
            }
\par
else {
                largestNumber = n3;
            }
        }
\par
System.out.println("The largest number is " + largestNumber);
    }
}
\end{lstlisting}%
\lthtmlfigureZ
\lthtmlcheckvsize\clearpage}

\stepcounter{section}
{\newpage\clearpage
\lthtmlfigureA{lstlisting1958}%
\begin{lstlisting}
if (expression) {
   number = 10;
}
else {
   number = -10;
}
\end{lstlisting}%
\lthtmlfigureZ
\lthtmlcheckvsize\clearpage}

{\newpage\clearpage
\lthtmlfigureA{lstlisting1962}%
\begin{lstlisting}
number = (expression) ? expressionTrue : expressinFalse;
\end{lstlisting}%
\lthtmlfigureZ
\lthtmlcheckvsize\clearpage}

{\newpage\clearpage
\lthtmlfigureA{lstlisting1973}%
\begin{lstlisting}
class Operator {
   public static void main(String[] args) {   
\par
Double number = -5.5;
      String result;
\par
result = (number > 0.0) ? "positive" : "not positive";
      System.out.println(number + " is " + result);
   }
}
\end{lstlisting}%
\lthtmlfigureZ
\lthtmlcheckvsize\clearpage}

\stepcounter{subsubsection}
{\newpage\clearpage
\lthtmlfigureA{lstlisting1978}%
\begin{lstlisting}
if (expression1) {
	result = 1;
} else if (expression2) {
	result = 2;
} else if (expression3) {
	result = 3;
} else {
	result = 0;
}
\end{lstlisting}%
\lthtmlfigureZ
\lthtmlcheckvsize\clearpage}

{\newpage\clearpage
\lthtmlfigureA{lstlisting1984}%
\begin{lstlisting}
result = (expression1) ? 1 : (expression2) ? 2 : (expression3) ? 3 : 0;
\end{lstlisting}%
\lthtmlfigureZ
\lthtmlcheckvsize\clearpage}

\stepcounter{section}
{\newpage\clearpage
\lthtmlfigureA{lstlisting1990}%
\begin{lstlisting}
switch (variable/expression) {
case value1:
   // statements of case1
   break;
\par
case value2:
   // statements of case2
   break;
\par
.. .. ...
   .. .. ...
\par
default:
   // default statements
}
\end{lstlisting}%
\lthtmlfigureZ
\lthtmlcheckvsize\clearpage}

{\newpage\clearpage
\lthtmlfigureA{lstlisting2011}%
\begin{lstlisting}
class Main {
    public static void main(String[] args) {
\par
int week = 4;
        String day;
\par
// switch statement to check day
        switch (week) {
            case 1:
                day = "Sunday";
                break;
            case 2:
                day = "Monday";
                break;
            case 3:
                day = "Tuesday";
                break;
\par
// match the value of week
            case 4:
                day = "Wednesday";
                break;
            case 5:
                day = "Thursday";
                break;
            case 6:
                day = "Friday";
                break;
            case 7:
                day = "Saturday";
                break;
            default:
                day = "Invalid day";
                break;
        }
        System.out.println("The day is " + day);
    }
}
\end{lstlisting}%
\lthtmlfigureZ
\lthtmlcheckvsize\clearpage}

{\newpage\clearpage
\lthtmlfigureA{lstlisting2025}%
\begin{lstlisting}
import java.util.Scanner;
\par
class Main {
    public static void main(String[] args) {
\par
char operator;
        Double number1, number2, result;
\par
// create an object of Scanner class
        Scanner scanner = new Scanner(System.in);
        System.out.print("Enter operator (either +, -, * or /): ");
\par
// ask user to enter operator
        operator = scanner.next().charAt(0);
        System.out.print("Enter number1 and number2 respectively: ");
\par
// ask user to enter numbers
        number1 = scanner.nextDouble();
        number2 = scanner.nextDouble();
\par
switch (operator) {
\par
// performs addition between numbers
            case '+':
                result = number1 + number2;
                System.out.print(number1 + "+" + number2 + " = " + result);
                break;
\par
// performs subtraction between numbers
            case '-':
                result = number1 - number2;
                System.out.print(number1 + "-" + number2 + " = " + result);
                break;
\par
// performs multiplication between numbers
            case '*':
                result = number1 * number2;
                System.out.print(number1 + "*" + number2 + " = " + result);
                break;
\par
// performs division between numbers
            case '/':
                result = number1 / number2;
                System.out.print(number1 + "/" + number2 + " = " + result);
                break;
\par
default:
                System.out.println("Invalid operator!");
                break;
        }
    }
}
\end{lstlisting}%
\lthtmlfigureZ
\lthtmlcheckvsize\clearpage}

{\newpage\clearpage
\lthtmlfigureA{lstlisting2028}%
\begin{lstlisting}
Enter operator (either +, -, * or /): *
Enter number1 and number2 respectively: 1.4
-5.3
1.4*-5.3 = -7.419999999999999
\end{lstlisting}%
\lthtmlfigureZ
\lthtmlcheckvsize\clearpage}

\stepcounter{section}
{\newpage\clearpage
\lthtmlfigureA{lstlisting2032}%
\begin{lstlisting}
for (initialization; testExpression; update)
{
    // codes inside for loop's body
}
\end{lstlisting}%
\lthtmlfigureZ
\lthtmlcheckvsize\clearpage}

{\newpage\clearpage
\lthtmlfigureA{lstlisting2041}%
\begin{lstlisting}
// Program to print a sentence 10 times
\par
class Loop {
    public static void main(String[] args) {
\par
for (int i = 1; i <= 10; ++i) {
            System.out.println("Line " + i);
        }
    }
}
\end{lstlisting}%
\lthtmlfigureZ
\lthtmlcheckvsize\clearpage}

{\newpage\clearpage
\lthtmlfigureA{lstlisting2044}%
\begin{lstlisting}
Line 1
Line 2
Line 3
Line 4
Line 5
Line 6
Line 7
Line 8
Line 9
Line 10
\end{lstlisting}%
\lthtmlfigureZ
\lthtmlcheckvsize\clearpage}

{\newpage\clearpage
\lthtmlfigureA{lstlisting2058}%
\begin{lstlisting}
// Program to find the sum of natural numbers from 1 to 1000.
\par
class Number {
    public static void main(String[] args) {
\par
int sum = 0;
\par
for (int i = 1; i <= 1000; ++i) {
            sum += i;     // sum = sum + i
        }
\par
System.out.println("Sum = " + sum);
    }
}
\end{lstlisting}%
\lthtmlfigureZ
\lthtmlcheckvsize\clearpage}

\stepcounter{subsubsection}
{\newpage\clearpage
\lthtmlfigureA{lstlisting2065}%
\begin{lstlisting}
// Infinite for Loop
\par
class Infinite {
    public static void main(String[] args) {
\par
int sum = 0;
\par
for (int i = 1; i <= 10; --i) {
            System.out.println("Hello");
        }
    }
}
\end{lstlisting}%
\lthtmlfigureZ
\lthtmlcheckvsize\clearpage}

\stepcounter{subsection}
{\newpage\clearpage
\lthtmlfigureA{lstlisting2073}%
\begin{lstlisting}
for (int a : array) {
    System.out.println(a);
}
\end{lstlisting}%
\lthtmlfigureZ
\lthtmlcheckvsize\clearpage}

{\newpage\clearpage
\lthtmlfigureA{lstlisting2079}%
\begin{lstlisting}
for(data_type item : collections) {
    ...
}
\end{lstlisting}%
\lthtmlfigureZ
\lthtmlcheckvsize\clearpage}

{\newpage\clearpage
\lthtmlfigureA{lstlisting2084}%
\begin{lstlisting}
// The program below calculates the sum of all elements of an integer array.
\par
class Main {
 public static void main(String[] args) {
\par
// create array
   int[] numbers = {3, 4, 5, -5, 0, 12};
   int sum = 0;
\par
// for each loop 
   for (int number: numbers) {
     sum += number;
   }
\par
System.out.println("Sum = " + sum);
 }
}
\end{lstlisting}%
\lthtmlfigureZ
\lthtmlcheckvsize\clearpage}

\stepcounter{subsubsection}
{\newpage\clearpage
\lthtmlfigureA{lstlisting2092}%
\begin{lstlisting}
class Main {
 public static void main(String[] args) {
\par
char[] vowels = {'a', 'e', 'i', 'o', 'u'};
\par
// using for loop
   for (int i = 0; i < vowels.length; ++ i) {
     System.out.println(vowels[i]);
   }
 }
}
\end{lstlisting}%
\lthtmlfigureZ
\lthtmlcheckvsize\clearpage}

{\newpage\clearpage
\lthtmlfigureA{lstlisting2096}%
\begin{lstlisting}
a
e
i
o
u
\end{lstlisting}%
\lthtmlfigureZ
\lthtmlcheckvsize\clearpage}

\stepcounter{subsubsection}
{\newpage\clearpage
\lthtmlfigureA{lstlisting2099}%
\begin{lstlisting}
class Main {
 public static void main(String[] args) {
\par
// create a char array  
   char[] vowels = {'a', 'e', 'i', 'o', 'u'};
\par
// foreach loop
   for (char item: vowels) {
     System.out.println(item);
   }
 }
}
\end{lstlisting}%
\lthtmlfigureZ
\lthtmlcheckvsize\clearpage}

\stepcounter{subsection}
\stepcounter{section}
{\newpage\clearpage
\lthtmlfigureA{lstlisting2116}%
\begin{lstlisting}
while (testExpression) {
    // codes inside the body of while loop
}
\end{lstlisting}%
\lthtmlfigureZ
\lthtmlcheckvsize\clearpage}

\stepcounter{subsubsection}
{\newpage\clearpage
\lthtmlfigureA{lstlisting2125}%
\begin{lstlisting}
// Program to print line 10 times
\par
class Loop {
    public static void main(String[] args) {
\par
int i = 1;
\par
while (i <= 10) {
            System.out.println("Line " + i);
            ++i;
        }
    }
}
\end{lstlisting}%
\lthtmlfigureZ
\lthtmlcheckvsize\clearpage}

{\newpage\clearpage
\lthtmlfigureA{lstlisting2141}%
\begin{lstlisting}
// Program to find the sum of natural numbers from 1 to 100.
\par
class AssignmentOperator {
    public static void main(String[] args) {
\par
int sum = 0, i = 100;
\par
while (i != 0) {
            sum += i;     // sum = sum + i;
            --i;
        }
\par
System.out.println("Sum = " + sum);
    }
}
\end{lstlisting}%
\lthtmlfigureZ
\lthtmlcheckvsize\clearpage}

\stepcounter{subsection}
{\newpage\clearpage
\lthtmlfigureA{lstlisting2152}%
\begin{lstlisting}
do {
   // codes inside body of do while loop
} while (testExpression);
\end{lstlisting}%
\lthtmlfigureZ
\lthtmlcheckvsize\clearpage}

{\newpage\clearpage
\lthtmlfigureA{lstlisting2163}%
\begin{lstlisting}
import java.util.Scanner;
\par
class Sum {
    public static void main(String[] args) {
\par
Double number, sum = 0.0;
        // creates an object of Scanner class
        Scanner input = new Scanner(System.in);
\par
do {
\par
// takes input from the user
            System.out.print("Enter a number: ");
            number = input.nextDouble();
            sum += number;
        } while (number != 0.0);  // test expression
\par
System.out.println("Sum = " + sum);
    }
}
\end{lstlisting}%
\lthtmlfigureZ
\lthtmlcheckvsize\clearpage}

{\newpage\clearpage
\lthtmlfigureA{lstlisting2166}%
\begin{lstlisting}
Enter a number: 2.5
Enter a number: 23.3
Enter a number: -4.2
Enter a number: 3.4
Enter a number: 0
Sum = 25.0
\end{lstlisting}%
\lthtmlfigureZ
\lthtmlcheckvsize\clearpage}

\stepcounter{subsubsection}
{\newpage\clearpage
\lthtmlfigureA{lstlisting2173}%
\begin{lstlisting}
// Infinite while loop
while (true) {
   // body of while loop
}
\end{lstlisting}%
\lthtmlfigureZ
\lthtmlcheckvsize\clearpage}

{\newpage\clearpage
\lthtmlfigureA{lstlisting2176}%
\begin{lstlisting}
// Infinite while loop
int i = 100;
while (i == 100) {
   System.out.print("Hey!");
}
\end{lstlisting}%
\lthtmlfigureZ
\lthtmlcheckvsize\clearpage}

\stepcounter{section}
{\newpage\clearpage
\lthtmlfigureA{lstlisting2181}%
\begin{lstlisting}
class Test {
    public static void main(String[] args) {
\par
// for loop
        for (int i = 1; i <= 10; ++i) {
\par
// if the value of i is 5 the loop terminates  
            if (i == 5) {
                break;
            }      
            System.out.println(i);
        }   
    }
}
\end{lstlisting}%
\lthtmlfigureZ
\lthtmlcheckvsize\clearpage}

{\newpage\clearpage
\lthtmlfigureA{lstlisting2184}%
\begin{lstlisting}
1
2
3
4
\end{lstlisting}%
\lthtmlfigureZ
\lthtmlcheckvsize\clearpage}

{\newpage\clearpage
\lthtmlfigureA{lstlisting2189}%
\begin{lstlisting}
if (i == 5) {
    break;
}
\end{lstlisting}%
\lthtmlfigureZ
\lthtmlcheckvsize\clearpage}

{\newpage\clearpage
\lthtmlfigureA{lstlisting2195}%
\begin{lstlisting}
import java.util.Scanner;
\par
class UserInputSum {
    public static void main(String[] args) {
\par
Double number, sum = 0.0;
\par
// create an object of Scanner
        Scanner input = new Scanner(System.in);
\par
while (true) {
            System.out.print("Enter a number: ");
\par
// takes double input from user
            number = input.nextDouble();
\par
// if number is negative the loop terminates
            if (number < 0.0) {
                break;
            }
\par
sum += number;
        }
        System.out.println("Sum = " + sum);
    }
}
\end{lstlisting}%
\lthtmlfigureZ
\lthtmlcheckvsize\clearpage}

{\newpage\clearpage
\lthtmlfigureA{lstlisting2198}%
\begin{lstlisting}
Enter a number: 3.2
Enter a number: 5
Enter a number: 2.3
Enter a number: 0
Enter a number: -4.5
Sum = 10.5
\end{lstlisting}%
\lthtmlfigureZ
\lthtmlcheckvsize\clearpage}

{\newpage\clearpage
\lthtmlfigureA{lstlisting2202}%
\begin{lstlisting}
if (number < 0.0) {
    break;
}
\end{lstlisting}%
\lthtmlfigureZ
\lthtmlcheckvsize\clearpage}

\stepcounter{subsubsection}
{\newpage\clearpage
\lthtmlfigureA{lstlisting2209}%
\begin{lstlisting}
while (testExpression) {
   // codes
   second:
   while (testExpression) {
      // codes
      while(testExpression) {
         // codes
         break second;
      }
   }
   // control jumps here
}
\end{lstlisting}%
\lthtmlfigureZ
\lthtmlcheckvsize\clearpage}

{\newpage\clearpage
\lthtmlfigureA{lstlisting2216}%
\begin{lstlisting}
class LabeledBreak {
    public static void main(String[] args) {
\par
// the for loop is labeled as first   
        first:
        for( int i = 1; i < 5; i++) {
\par
// the for loop is labeled as second
            second:
            for(int j = 1; j < 3; j ++ ) {
                System.out.println("i = " + i + "; j = " +j);
\par
// the break statement breaks the first for loop
                if ( i == 2)
                    break first;
            }
        }
    }
}
\end{lstlisting}%
\lthtmlfigureZ
\lthtmlcheckvsize\clearpage}

{\newpage\clearpage
\lthtmlfigureA{lstlisting2219}%
\begin{lstlisting}
i = 1; j = 1
i = 1; j = 2
i = 2; j = 1
\end{lstlisting}%
\lthtmlfigureZ
\lthtmlcheckvsize\clearpage}

{\newpage\clearpage
\lthtmlfigureA{lstlisting2222}%
\begin{lstlisting}
first:
for(int i = 1; i < 5; i++) {...}
\end{lstlisting}%
\lthtmlfigureZ
\lthtmlcheckvsize\clearpage}

{\newpage\clearpage
\lthtmlfigureA{lstlisting2228}%
\begin{lstlisting}
class LabeledBreak {
    public static void main(String[] args) {
\par
// the for loop is labeled as first
        first:
        for( int i = 1; i < 5; i++) {
\par
// the for loop is labeled as second
            second:
            for(int j = 1; j < 3; j ++ ) {
\par
System.out.println("i = " + i + "; j = " +j);
\par
// the break statement terminates the loop labeled as second   
                if ( i == 2)
                    break second;
            }
        }
    }
}
\end{lstlisting}%
\lthtmlfigureZ
\lthtmlcheckvsize\clearpage}

{\newpage\clearpage
\lthtmlfigureA{lstlisting2231}%
\begin{lstlisting}
i = 1; j = 1
i = 1; j = 2
i = 2; j = 1
i = 3; j = 1
i = 3; j = 2
i = 4; j = 1
i = 4; j = 2
\end{lstlisting}%
\lthtmlfigureZ
\lthtmlcheckvsize\clearpage}

\stepcounter{section}
{\newpage\clearpage
\lthtmlfigureA{lstlisting2238}%
\begin{lstlisting}
class Test {
    public static void main(String[] args) {
\par
// for loop
        for (int i = 1; i <= 10; ++i) {
\par
// if value of i is between 4 and 9, continue is executed 
            if (i > 4 && i < 9) {
                continue;
            }      
            System.out.println(i);
        }   
    }
}
\end{lstlisting}%
\lthtmlfigureZ
\lthtmlcheckvsize\clearpage}

{\newpage\clearpage
\lthtmlfigureA{lstlisting2241}%
\begin{lstlisting}
1
2
​​​​3
4
9
10
\end{lstlisting}%
\lthtmlfigureZ
\lthtmlcheckvsize\clearpage}

{\newpage\clearpage
\lthtmlfigureA{lstlisting2246}%
\begin{lstlisting}
if (i > 5 && i < 9) {
    continue;
}
\end{lstlisting}%
\lthtmlfigureZ
\lthtmlcheckvsize\clearpage}

{\newpage\clearpage
\lthtmlfigureA{lstlisting2253}%
\begin{lstlisting}
import java.util.Scanner;
\par
class AssignmentOperator {
    public static void main(String[] args) {
\par
Double number, sum = 0.0;
        // create an object of Scanner
        Scanner input = new Scanner(System.in);
\par
for (int i = 1; i < 6; ++i) {
            System.out.print("Enter a number: ");
            // takes double type input from the user
            number = input.nextDouble();
\par
// if number is negative, the iteration is skipped
            if (number <= 0.0) {
                continue;
            }
\par
sum += number;
        }
        System.out.println("Sum = " + sum);
    }
}
\end{lstlisting}%
\lthtmlfigureZ
\lthtmlcheckvsize\clearpage}

{\newpage\clearpage
\lthtmlfigureA{lstlisting2256}%
\begin{lstlisting}
Enter a number: 2.2
Enter a number: 5.6
Enter a number: 0
Enter a number: -2.4
Enter a number: -3
Sum = 7.8
\end{lstlisting}%
\lthtmlfigureZ
\lthtmlcheckvsize\clearpage}

\stepcounter{subsubsection}
{\newpage\clearpage
\lthtmlfigureA{lstlisting2265}%
\begin{lstlisting}
class LabeledContinue {
    public static void main(String[] args) {
\par
// the outer for loop is labeled as label      
        first:
        for (int i = 1; i < 6; ++i) {
            for (int j = 1; j < 5; ++j) {
                if (i == 3 || j == 2)
\par
// skips the iteration of label (outer for loop)
                    continue first;
                System.out.println("i = " + i + "; j = " + j); 
            }
        } 
    }
}
\end{lstlisting}%
\lthtmlfigureZ
\lthtmlcheckvsize\clearpage}

{\newpage\clearpage
\lthtmlfigureA{lstlisting2268}%
\begin{lstlisting}
i = 1; j = 1
i = 2; j = 1
i = 4; j = 1
i = 5; j = 1
\end{lstlisting}%
\lthtmlfigureZ
\lthtmlcheckvsize\clearpage}

{\newpage\clearpage
\lthtmlfigureA{lstlisting2272}%
\begin{lstlisting}
if (i==3 || j==2)
    continue first;
\end{lstlisting}%
\lthtmlfigureZ
\lthtmlcheckvsize\clearpage}

{\newpage\clearpage
\lthtmlfigureA{lstlisting2276}%
\begin{lstlisting}
first:
for (int i = 1; i < 6; ++i) {..}
\end{lstlisting}%
\lthtmlfigureZ
\lthtmlcheckvsize\clearpage}

\stepcounter{chapter}
\stepcounter{section}
\stepcounter{section}
\stepcounter{section}
\stepcounter{subsubsection}
\stepcounter{subsubsection}
\stepcounter{subsubsection}
\stepcounter{subsubsection}
\stepcounter{subsubsection}
\stepcounter{subsubsection}
\stepcounter{subsubsection}
{\newpage\clearpage
\lthtmlinlinemathA{tex2html_wrap_inline4599}%
$\lstinline{this}$%
\lthtmlindisplaymathZ
\lthtmlcheckvsize\clearpage}

{\newpage\clearpage
\lthtmlinlinemathA{tex2html_wrap_inline4601}%
$\lstinline{self}$%
\lthtmlindisplaymathZ
\lthtmlcheckvsize\clearpage}

\stepcounter{section}
\stepcounter{section}
\stepcounter{subsection}
\stepcounter{subsection}
\stepcounter{subsection}
{\newpage\clearpage
\lthtmlfigureA{lstlisting2373}%
\begin{lstlisting}
Dog tuffy;
\end{lstlisting}%
\lthtmlfigureZ
\lthtmlcheckvsize\clearpage}

\stepcounter{subsection}
{\newpage\clearpage
\lthtmlfigureA{lstlisting2376}%
\begin{lstlisting}
// Class Declaration
\par
public class Dog
{
    // Instance Variables
    String name;
    String breed;
    int age;
    String color;
\par
// Constructor Declaration of Class
    public Dog(String name, String breed,
                   int age, String color)
    {
        this.name = name;
        this.breed = breed;
        this.age = age;
        this.color = color;
    }
\par
// method 1
    public String getName()
    {
        return name;
    }
\par
// method 2
    public String getBreed()
    {
        return breed;
    }
\par
// method 3
    public int getAge()
    {
        return age;
    }
\par
// method 4
    public String getColor()
    {
        return color;
    }
\par
@Override
    public String toString()
    {
        return("Hi my name is "+ this.getName()+
               ".\nMy breed,age and color are " +
               this.getBreed()+"," + this.getAge()+
               ","+ this.getColor());
    }
\par
public static void main(String[] args)
    {
        Dog tuffy = new Dog("tuffy","papillon", 5, "white");
        System.out.println(tuffy.toString());
    }
}
\end{lstlisting}%
\lthtmlfigureZ
\lthtmlcheckvsize\clearpage}

{\newpage\clearpage
\lthtmlfigureA{lstlisting2385}%
\begin{lstlisting}
Hi my name is tuffy.
My breed,age and color are papillon,5,white
\end{lstlisting}%
\lthtmlfigureZ
\lthtmlcheckvsize\clearpage}

{\newpage\clearpage
\lthtmlfigureA{lstlisting2390}%
\begin{lstlisting}
Dog tuffy = new Dog("tuffy","papillon",5, "white");
\end{lstlisting}%
\lthtmlfigureZ
\lthtmlcheckvsize\clearpage}

\stepcounter{subsection}
{\newpage\clearpage
\lthtmlfigureA{lstlisting2400}%
\begin{lstlisting}
// creating object of class Test
Test t = new Test();
\end{lstlisting}%
\lthtmlfigureZ
\lthtmlcheckvsize\clearpage}

{\newpage\clearpage
\lthtmlfigureA{lstlisting2403}%
\begin{lstlisting}
// creating object of public class Test
// consider class Test present in com.p1 package
Test obj = (Test)Class.forName("com.p1.Test").newInstance();
\end{lstlisting}%
\lthtmlfigureZ
\lthtmlcheckvsize\clearpage}

{\newpage\clearpage
\lthtmlfigureA{lstlisting2406}%
\begin{lstlisting}    
// creating object of class Test
Test t1 = new Test();
\par
// creating clone of above object
Test t2 = (Test)t1.clone();
\end{lstlisting}%
\lthtmlfigureZ
\lthtmlcheckvsize\clearpage}

{\newpage\clearpage
\lthtmlfigureA{lstlisting2409}%
\begin{lstlisting}
FileInputStream file = new FileInputStream(filename);
ObjectInputStream in = new ObjectInputStream(file);
Object obj = in.readObject();
\end{lstlisting}%
\lthtmlfigureZ
\lthtmlcheckvsize\clearpage}

\stepcounter{subsection}
{\newpage\clearpage
\lthtmlfigureA{lstlisting2415}%
\begin{lstlisting}
	Test test = new Test();
test = new Test();
	\end{lstlisting}%
\lthtmlfigureZ
\lthtmlcheckvsize\clearpage}

{\newpage\clearpage
\lthtmlfigureA{lstlisting2418}%
\begin{lstlisting}
	class Animal {}
\par
class Dog extends Animal {}
class Cat extends Animal {}
\par
public class Test
{
    // using Dog object
    Animal obj = new Dog();
\par
// using Cat object
    obj = new Cat();
}   
	\end{lstlisting}%
\lthtmlfigureZ
\lthtmlcheckvsize\clearpage}

\stepcounter{subsection}
{\newpage\clearpage
\lthtmlfigureA{lstlisting2428}%
\begin{lstlisting}
btn.setOnAction(new EventHandler()
{
    public void handle(ActionEvent event)
    {
        System.out.println("Hello World!");
    }
});
\end{lstlisting}%
\lthtmlfigureZ
\lthtmlcheckvsize\clearpage}

\stepcounter{section}
{\newpage\clearpage
\lthtmlfigureA{lstlisting2432}%
\begin{lstlisting}
class Java_Outer_class{  
 //code  
 class Java_Inner_class{  
  //code  
 }  
}  
\end{lstlisting}%
\lthtmlfigureZ
\lthtmlcheckvsize\clearpage}

\stepcounter{subsection}
\stepcounter{subsection}
\stepcounter{section}
\stepcounter{subsection}
\stepcounter{subsection}
\stepcounter{subsubsection}
{\newpage\clearpage
\lthtmlfigureA{lstlisting2476}%
\begin{lstlisting}
public class Demo {  
	public static void main(String[] args) {  
		// using the max() method of Math class  
		System.out.print("The maximum number is: " + Math.max(9,7));  
	}  
}  
\end{lstlisting}%
\lthtmlfigureZ
\lthtmlcheckvsize\clearpage}

\stepcounter{subsubsection}
{\newpage\clearpage
\lthtmlfigureA{lstlisting2484}%
\begin{lstlisting}
//user defined method  
public static void findEvenOdd(int num) {  
	//method body  
	if(numSystem.out.println(num+" is even");   
	else   
	System.out.println(num+" is odd");  
}  
\end{lstlisting}%
\lthtmlfigureZ
\lthtmlcheckvsize\clearpage}

\stepcounter{subsubsection}
{\newpage\clearpage
\lthtmlfigureA{lstlisting2490}%
\begin{lstlisting}
import java.util.Scanner;  
public class EvenOdd {  
	public static void main (String args[]) {  
		//creating Scanner class object     
		Scanner scan=new Scanner(System.in);  
		System.out.print("Enter the number: ");  
		//reading value from the user  
		int num=scan.nextInt();  
		//method calling  
		findEvenOdd(num);  
	}
}  
\end{lstlisting}%
\lthtmlfigureZ
\lthtmlcheckvsize\clearpage}

{\newpage\clearpage
\lthtmlfigureA{lstlisting2494}%
\begin{lstlisting}
import java.util.Scanner;  
public class EvenOdd {  
	public static void main (String args[]) {  
		//creating Scanner class object     
		Scanner scan=new Scanner(System.in);  
		System.out.print("Enter the number: ");  
		//reading value from user  
		int num=scan.nextInt();  
		//method calling  
		findEvenOdd(num);  
	}  
	//user defined method  
	public static void findEvenOdd(int num) {  
		//method body  
		if(numSystem.out.println(num+" is even");   
		else   
		System.out.println(num+" is odd");  
	}  
}  
\end{lstlisting}%
\lthtmlfigureZ
\lthtmlcheckvsize\clearpage}

{\newpage\clearpage
\lthtmlfigureA{lstlisting2498}%
\begin{lstlisting}
Enter the number: 12
12 is even
\end{lstlisting}%
\lthtmlfigureZ
\lthtmlcheckvsize\clearpage}

{\newpage\clearpage
\lthtmlfigureA{lstlisting2500}%
\begin{lstlisting}
Enter the number: 99
99 is odd
\end{lstlisting}%
\lthtmlfigureZ
\lthtmlcheckvsize\clearpage}

{\newpage\clearpage
\lthtmlfigureA{lstlisting2511}%
\begin{lstlisting}
public class Addition   
{  
public static void main(String[] args)   
{  
int a = 19;  
int b = 5;  
//method calling  
int c = add(a, b);   //a and b are actual parameters  
System.out.println("The sum of a and b is= " + c);  
}  
//user defined method  
public static int add(int n1, int n2)   //n1 and n2 are formal parameters  
{  
int s;  
s=n1+n2;  
return s; //returning the sum  
}  
}  
\end{lstlisting}%
\lthtmlfigureZ
\lthtmlcheckvsize\clearpage}

{\newpage\clearpage
\lthtmlfigureA{lstlisting2515}%
\begin{lstlisting}
The sum of a and b is= 24
\end{lstlisting}%
\lthtmlfigureZ
\lthtmlcheckvsize\clearpage}

\stepcounter{subsubsection}
{\newpage\clearpage
\lthtmlfigureA{lstlisting2519}%
\begin{lstlisting}
public class Display {  
	public static void main(String[] args) {  
		show();  
	}  
	static void show() {  
		System.out.println("It is an example of static method.");  
	}  
}  
\end{lstlisting}%
\lthtmlfigureZ
\lthtmlcheckvsize\clearpage}

{\newpage\clearpage
\lthtmlfigureA{lstlisting2523}%
\begin{lstlisting}
It is an example of a static method.
\end{lstlisting}%
\lthtmlfigureZ
\lthtmlcheckvsize\clearpage}

\stepcounter{subsubsection}
{\newpage\clearpage
\lthtmlfigureA{lstlisting2526}%
\begin{lstlisting}
public class InstanceMethodExample {  
	public static void main(String [] args) {  
		//Creating an object of the class  
		InstanceMethodExample obj = new InstanceMethodExample();  
		//invoking instance method   
		System.out.println("The sum is: "+obj.add(12, 13));  
	}  
	int s;  
	//user-defined method because we have not used static keyword  
	public int add(int a, int b) {  
		s = a+b;  
		//returning the sum  
		return s;  
	}  
}  
\end{lstlisting}%
\lthtmlfigureZ
\lthtmlcheckvsize\clearpage}

{\newpage\clearpage
\lthtmlfigureA{lstlisting2530}%
\begin{lstlisting}
The sum is: 25
\end{lstlisting}%
\lthtmlfigureZ
\lthtmlcheckvsize\clearpage}

{\newpage\clearpage
\lthtmlfigureA{lstlisting2538}%
\begin{lstlisting}
public int getId() {    
	return Id;    
}    
\end{lstlisting}%
\lthtmlfigureZ
\lthtmlcheckvsize\clearpage}

{\newpage\clearpage
\lthtmlfigureA{lstlisting2545}%
\begin{lstlisting}
public void setRoll(int roll) {  
	this.roll = roll;  
}  
\end{lstlisting}%
\lthtmlfigureZ
\lthtmlcheckvsize\clearpage}

{\newpage\clearpage
\lthtmlfigureA{lstlisting2548}%
\begin{lstlisting}
public class Student {  
	private int roll;  
	private String name;  
	//accessor method  
	public int getRoll() {  
		return roll;  
	}  
	//mutator method  
	public void setRoll(int roll) {  
		this.roll = roll;  
	}  
	public String getName()	{  
		return name;  
	}  
	public void setName(String name) {  
		this.name = name;  
	}  
	public void display()	{  
		System.out.println("Roll no.: "+roll);  
		System.out.println("Student name: "+name);  
	}  
}  
\end{lstlisting}%
\lthtmlfigureZ
\lthtmlcheckvsize\clearpage}

\stepcounter{subsubsection}
{\newpage\clearpage
\lthtmlfigureA{lstlisting2559}%
\begin{lstlisting}
abstract void method_name();  
\end{lstlisting}%
\lthtmlfigureZ
\lthtmlcheckvsize\clearpage}

{\newpage\clearpage
\lthtmlfigureA{lstlisting2561}%
\begin{lstlisting}
//abstract class  
abstract class Demo {  
	//abstract method declaration  
	abstract void display();  
}  
public class MyClass extends Demo {  
	//method impelmentation  
	void display() {  
		System.out.println("Abstract method?");  
	}  
	public static void main(String args[]) {  
		//creating object of abstract class  
		Demo obj = new MyClass();  
		//invoking abstract method  
		obj.display();  
	}  
}  
\end{lstlisting}%
\lthtmlfigureZ
\lthtmlcheckvsize\clearpage}

{\newpage\clearpage
\lthtmlfigureA{lstlisting2566}%
\begin{lstlisting}
Abstract method...
\end{lstlisting}%
\lthtmlfigureZ
\lthtmlcheckvsize\clearpage}

\stepcounter{section}
\stepcounter{subsubsection}
{\newpage\clearpage
\lthtmlfigureA{lstlisting2578}%
\begin{lstlisting}
<class_name>(){}  
\end{lstlisting}%
\lthtmlfigureZ
\lthtmlcheckvsize\clearpage}

{\newpage\clearpage
\lthtmlfigureA{lstlisting2581}%
\begin{lstlisting}
//Java Program to create and call a default constructor  
class Bike1 {  
	//creating a default constructor  
	Bike1(){System.out.println("Bike is created");}  
	//main method  
	public static void main(String args[]){  
		//calling a default constructor  
		Bike1 b=new Bike1();  
	}  
}  
\end{lstlisting}%
\lthtmlfigureZ
\lthtmlcheckvsize\clearpage}

{\newpage\clearpage
\lthtmlfigureA{lstlisting2585}%
\begin{lstlisting}
Bike is created
\end{lstlisting}%
\lthtmlfigureZ
\lthtmlcheckvsize\clearpage}

{\newpage\clearpage
\lthtmlfigureA{lstlisting2592}%
\begin{lstlisting}
//Let us see another example of default constructor  
//which displays the default values  
class Student3 {  
	int id;  
	String name;  
	//method to display the value of id and name  
	void display(){System.out.println(id+" "+name);}  
\par
public static void main(String args[]) {  
		//creating objects  
		Student3 s1=new Student3();  
		Student3 s2=new Student3();  
		//displaying values of the object  
		s1.display();  
		s2.display();  
	}  
}  
\end{lstlisting}%
\lthtmlfigureZ
\lthtmlcheckvsize\clearpage}

{\newpage\clearpage
\lthtmlfigureA{lstlisting2596}%
\begin{lstlisting} 
0 null
0 null
\end{lstlisting}%
\lthtmlfigureZ
\lthtmlcheckvsize\clearpage}

\stepcounter{subsubsection}
{\newpage\clearpage
\lthtmlfigureA{lstlisting2599}%
\begin{lstlisting}
//Java Program to demonstrate the use of the parameterized constructor.  
class Student4 {  
    int id;  
    String name;  
    //creating a parameterized constructor  
    Student4(int i,String n){  
			id = i;  
			name = n;  
    }  
    //method to display the values  
    void display(){System.out.println(id+" "+name);}  
\par
public static void main(String args[]){  
			//creating objects and passing values  
			Student4 s1 = new Student4(111,"Karan");  
			Student4 s2 = new Student4(222,"Aryan");  
			//calling method to display the values of object  
			s1.display();  
			s2.display();  
   }  
}  
\end{lstlisting}%
\lthtmlfigureZ
\lthtmlcheckvsize\clearpage}

{\newpage\clearpage
\lthtmlfigureA{lstlisting2604}%
\begin{lstlisting}
111 Karan
222 Aryan
\end{lstlisting}%
\lthtmlfigureZ
\lthtmlcheckvsize\clearpage}

\stepcounter{subsection}
{\newpage\clearpage
\lthtmlfigureA{lstlisting2607}%
\begin{lstlisting}
//Java program to overload constructors  
class Student5 {  
    int id;  
    String name;  
    int age;  
    //creating two arg constructor  
    Student5(int i,String n){  
			id = i;  
			name = n;  
    }  
    //creating three arg constructor  
    Student5(int i,String n,int a){  
			id = i;  
			name = n;  
			age=a;  
    }  
    void display(){System.out.println(id+" "+name+" "+age);}  
\par
public static void main(String args[]){  
			Student5 s1 = new Student5(111,"Karan");  
			Student5 s2 = new Student5(222,"Aryan",25);  
			s1.display();  
			s2.display();  
   }  
}  
\end{lstlisting}%
\lthtmlfigureZ
\lthtmlcheckvsize\clearpage}

{\newpage\clearpage
\lthtmlfigureA{lstlisting2613}%
\begin{lstlisting}
111 Karan 0
222 Aryan 25
\end{lstlisting}%
\lthtmlfigureZ
\lthtmlcheckvsize\clearpage}

\stepcounter{section}
\stepcounter{subsection}
\stepcounter{subsubsection}
{\newpage\clearpage
\lthtmlfigureA{lstlisting2623}%
\begin{lstlisting}
class Student{  
     int rollno;  
     String name;  
     String college="ITS";  
}  
\end{lstlisting}%
\lthtmlfigureZ
\lthtmlcheckvsize\clearpage}

{\newpage\clearpage
\lthtmlfigureA{lstlisting2628}%
\begin{lstlisting}
//Java Program to demonstrate the use of static variable  
class Student{  
   int rollno;//instance variable  
   String name;  
   static String college ="ITS";//static variable  
   //constructor  
   Student(int r, String n){  
   rollno = r;  
   name = n;  
   }  
   //method to display the values  
   void display (){System.out.println(rollno+" "+name+" "+college);}  
}  
//Test class to show the values of objects  
public class TestStaticVariable1{  
 public static void main(String args[]){  
	 Student s1 = new Student(111,"Karan");  
	 Student s2 = new Student(222,"Aryan");  
	 //we can change the college of all objects by the single line of code  
	 //Student.college="BBDIT";  
	 s1.display();  
	 s2.display();  
 }  
} 
\end{lstlisting}%
\lthtmlfigureZ
\lthtmlcheckvsize\clearpage}

{\newpage\clearpage
\lthtmlfigureA{lstlisting2633}%
\begin{lstlisting}
111 Karan ITS
222 Aryan ITS
\end{lstlisting}%
\lthtmlfigureZ
\lthtmlcheckvsize\clearpage}

{\newpage\clearpage
\lthtmlfigureA{lstlisting2642}%
\begin{lstlisting}
//Java Program to demonstrate the use of an instance variable  
//which get memory each time when we create an object of the class.  
class Counter{  
	int count=0;//will get memory each time when the instance is created  
\par
Counter(){  
		count++;//incrementing value  
		System.out.println(count);  
	}  
\par
public static void main(String args[]){  
		//Creating objects  
		Counter c1=new Counter();  
		Counter c2=new Counter();  
		Counter c3=new Counter();  
	}  
}  
\end{lstlisting}%
\lthtmlfigureZ
\lthtmlcheckvsize\clearpage}

{\newpage\clearpage
\lthtmlfigureA{lstlisting2646}%
\begin{lstlisting}
1
1
1
\end{lstlisting}%
\lthtmlfigureZ
\lthtmlcheckvsize\clearpage}

{\newpage\clearpage
\lthtmlfigureA{lstlisting2648}%
\begin{lstlisting}
//Java Program to illustrate the use of static variable which  
//is shared with all objects.  
class Counter2{  
	static int count=0;//will get memory only once and retain its value  
\par
Counter2(){  
		count++;//incrementing the value of static variable  
		System.out.println(count);  
	}  
\par
public static void main(String args[]){  
		//creating objects  
		Counter2 c1=new Counter2();  
		Counter2 c2=new Counter2();  
		Counter2 c3=new Counter2();  
	}  
}
\end{lstlisting}%
\lthtmlfigureZ
\lthtmlcheckvsize\clearpage}

{\newpage\clearpage
\lthtmlfigureA{lstlisting2652}%
\begin{lstlisting}
1
2
3
\end{lstlisting}%
\lthtmlfigureZ
\lthtmlcheckvsize\clearpage}

\stepcounter{subsubsection}
{\newpage\clearpage
\lthtmlfigureA{lstlisting2657}%
\begin{lstlisting}
//Java Program to demonstrate the use of a static method.  
class Student{  
     int rollno;  
     String name;  
     static String college = "ITS";  
     //static method to change the value of static variable  
     static void change(){  
			college = "BBDIT";  
     }  
     //constructor to initialize the variable  
     Student(int r, String n){  
			 rollno = r;  
			 name = n;  
     }  
     //method to display values  
     void display(){System.out.println(rollno+" "+name+" "+college);}  
}  
//Test class to create and display the values of object  
public class TestStaticMethod{  
    public static void main(String args[]){  
			Student.change();//calling change method  
			//creating objects  
			Student s1 = new Student(111,"Karan");  
			Student s2 = new Student(222,"Aryan");  
			Student s3 = new Student(333,"Sonoo");  
			//calling display method  
			s1.display();  
			s2.display();  
			s3.display();  
    }  
}
\end{lstlisting}%
\lthtmlfigureZ
\lthtmlcheckvsize\clearpage}

{\newpage\clearpage
\lthtmlfigureA{lstlisting2663}%
\begin{lstlisting}
111 Karan BBDIT
222 Aryan BBDIT
333 Sonoo BBDIT
\end{lstlisting}%
\lthtmlfigureZ
\lthtmlcheckvsize\clearpage}

{\newpage\clearpage
\lthtmlfigureA{lstlisting2665}%
\begin{lstlisting}
//Java Program to get the cube of a given number using the static method  
\par
class Calculate{  
  static int cube(int x){  
  return x*x*x;  
  }  
\par
public static void main(String args[]){  
  int result=Calculate.cube(5);  
  System.out.println(result);  
  }  
}  
\end{lstlisting}%
\lthtmlfigureZ
\lthtmlcheckvsize\clearpage}

\stepcounter{subsubsection}
{\newpage\clearpage
\lthtmlfigureA{lstlisting2671}%
\begin{lstlisting}
class A2{  
  static{System.out.println("static block is invoked");}  
  public static void main(String args[]){  
   System.out.println("Hello main");  
  }  
}  
\end{lstlisting}%
\lthtmlfigureZ
\lthtmlcheckvsize\clearpage}

{\newpage\clearpage
\lthtmlfigureA{lstlisting2675}%
\begin{lstlisting}
static block is invoked
Hello main
\end{lstlisting}%
\lthtmlfigureZ
\lthtmlcheckvsize\clearpage}

\stepcounter{subsection}
\stepcounter{subsubsection}
{\newpage\clearpage
\lthtmlfigureA{lstlisting2687}%
\begin{lstlisting}
class Student{  
	int rollno;  
	String name;  
	float fee;  
	Student(int rollno,String name,float fee){  
		rollno=rollno;  
		name=name;  
		fee=fee;  
	}  
	void display(){System.out.println(rollno+" "+name+" "+fee);}  
}  
class TestThis1{  
	public static void main(String args[]){  
		Student s1=new Student(111,"ankit",5000f);  
		Student s2=new Student(112,"sumit",6000f);  
		s1.display();  
		s2.display();  
	}
} 
\end{lstlisting}%
\lthtmlfigureZ
\lthtmlcheckvsize\clearpage}

{\newpage\clearpage
\lthtmlfigureA{lstlisting2692}%
\begin{lstlisting}
0 null 0.0
0 null 0.0
\end{lstlisting}%
\lthtmlfigureZ
\lthtmlcheckvsize\clearpage}

{\newpage\clearpage
\lthtmlfigureA{lstlisting2699}%
\begin{lstlisting}
111 ankit 5000.0
112 sumit 6000.0
\end{lstlisting}%
\lthtmlfigureZ
\lthtmlcheckvsize\clearpage}

{\newpage\clearpage
\lthtmlfigureA{lstlisting2701}%
\begin{lstlisting}
class Student{  
	int rollno;  
	String name;  
	float fee;  
	Student(int r,String n,float f){  
		rollno=r;  
		name=n;  
		fee=f;  
	}  
	void display(){System.out.println(rollno+" "+name+" "+fee);}  
}  
\par
class TestThis3{  
	public static void main(String args[]){  
		Student s1=new Student(111,"ankit",5000f);  
		Student s2=new Student(112,"sumit",6000f);  
		s1.display();  
		s2.display();  
	}
}  
\end{lstlisting}%
\lthtmlfigureZ
\lthtmlcheckvsize\clearpage}

\stepcounter{subsubsection}
{\newpage\clearpage
\lthtmlfigureA{lstlisting2709}%
\begin{lstlisting}
class A{  
	void m(){System.out.println("hello m");}  
	void n(){  
		System.out.println("hello n");  
		//m();//same as this.m()  
		this.m();  
	}  
}  
class TestThis4{  
	public static void main(String args[]){  
		A a=new A();  
		a.n();  
	}
}  
\end{lstlisting}%
\lthtmlfigureZ
\lthtmlcheckvsize\clearpage}

{\newpage\clearpage
\lthtmlfigureA{lstlisting2714}%
\begin{lstlisting}
hello n
hello m
\end{lstlisting}%
\lthtmlfigureZ
\lthtmlcheckvsize\clearpage}

\stepcounter{subsubsection}
{\newpage\clearpage
\lthtmlfigureA{lstlisting2718}%
\begin{lstlisting}
class A{  
	A(){System.out.println("hello a");}  
	A(int x){  
		this();  
		System.out.println(x);  
	}  
}  
class TestThis5{  
	public static void main(String args[]){  
		A a=new A(10);  
	}
}  
\end{lstlisting}%
\lthtmlfigureZ
\lthtmlcheckvsize\clearpage}

{\newpage\clearpage
\lthtmlfigureA{lstlisting2723}%
\begin{lstlisting}
hello a
10
\end{lstlisting}%
\lthtmlfigureZ
\lthtmlcheckvsize\clearpage}

{\newpage\clearpage
\lthtmlfigureA{lstlisting2725}%
\begin{lstlisting}
class A{  
A(){  
this(5);  
System.out.println("hello a");  
}  
A(int x){  
System.out.println(x);  
}  
}  
class TestThis6{  
public static void main(String args[]){  
A a=new A();  
}}  
\end{lstlisting}%
\lthtmlfigureZ
\lthtmlcheckvsize\clearpage}

{\newpage\clearpage
\lthtmlfigureA{lstlisting2730}%
\begin{lstlisting}
5
hello a
\end{lstlisting}%
\lthtmlfigureZ
\lthtmlcheckvsize\clearpage}

\stepcounter{subsubsection}
{\newpage\clearpage
\lthtmlfigureA{lstlisting2733}%
\begin{lstlisting}
class S2{  
  void m(S2 obj){  
		System.out.println("method is invoked");  
  }  
  void p(){  
		m(this);  
  }  
  public static void main(String args[]){  
		S2 s1 = new S2();  
		s1.p();  
  }  
}  
\end{lstlisting}%
\lthtmlfigureZ
\lthtmlcheckvsize\clearpage}

{\newpage\clearpage
\lthtmlfigureA{lstlisting2738}%
\begin{lstlisting}
method is invoked
\end{lstlisting}%
\lthtmlfigureZ
\lthtmlcheckvsize\clearpage}

\stepcounter{subsubsection}
{\newpage\clearpage
\lthtmlfigureA{lstlisting2741}%
\begin{lstlisting}
class B{  
  A4 obj;  
  B(A4 obj){  
    this.obj=obj;  
  }  
  void display(){  
    System.out.println(obj.data);//using data member of A4 class  
  }  
}  
\par
class A4{  
  int data=10;  
  A4(){  
   B b=new B(this);  
   b.display();  
  }  
  public static void main(String args[]){  
   A4 a=new A4();  
  }  
}  
\end{lstlisting}%
\lthtmlfigureZ
\lthtmlcheckvsize\clearpage}

\stepcounter{subsubsection}
{\newpage\clearpage
\lthtmlfigureA{lstlisting2749}%
\begin{lstlisting}
class A{  
	A getA(){  
		return this;  
	}  
	void msg(){System.out.println("Hello java");}  
}  
class Test1{  
	public static void main(String args[]){  
		new A().getA().msg();  
	}  
}  
\end{lstlisting}%
\lthtmlfigureZ
\lthtmlcheckvsize\clearpage}

{\newpage\clearpage
\lthtmlfigureA{lstlisting2754}%
\begin{lstlisting}
Hello java
\end{lstlisting}%
\lthtmlfigureZ
\lthtmlcheckvsize\clearpage}

\stepcounter{subsection}
\stepcounter{subsubsection}
{\newpage\clearpage
\lthtmlfigureA{lstlisting2761}%
\begin{lstlisting}
class Animal{  
	String color="white";  
}  
class Dog extends Animal{  
	String color="black";  
	void printColor(){  
		System.out.println(color);//prints color of Dog class  
		System.out.println(super.color);//prints color of Animal class  
	}  
}  
class TestSuper1{  
	public static void main(String args[]){  
		Dog d=new Dog();  
		d.printColor();  
	}
}  
\end{lstlisting}%
\lthtmlfigureZ
\lthtmlcheckvsize\clearpage}

{\newpage\clearpage
\lthtmlfigureA{lstlisting2766}%
\begin{lstlisting}
black
white
\end{lstlisting}%
\lthtmlfigureZ
\lthtmlcheckvsize\clearpage}

\stepcounter{subsubsection}
{\newpage\clearpage
\lthtmlfigureA{lstlisting2769}%
\begin{lstlisting}
class Animal{  
	void eat(){System.out.println("eating...");}  
}  
class Dog extends Animal{  
	void eat(){System.out.println("eating bread...");}  
	void bark(){System.out.println("barking...");}  
	void work(){  
		super.eat();  
		bark();  
	}  
}  
class TestSuper2{  
	public static void main(String args[]){  
		Dog d=new Dog();  
		d.work();  
	}
}
\end{lstlisting}%
\lthtmlfigureZ
\lthtmlcheckvsize\clearpage}

{\newpage\clearpage
\lthtmlfigureA{lstlisting2776}%
\begin{lstlisting}
eating...
barking...
\end{lstlisting}%
\lthtmlfigureZ
\lthtmlcheckvsize\clearpage}

\stepcounter{subsubsection}
{\newpage\clearpage
\lthtmlfigureA{lstlisting2779}%
\begin{lstlisting}
class Animal{  
	Animal(){System.out.println("animal is created");}  
}  
class Dog extends Animal{  
	Dog(){  
		super();  
		System.out.println("dog is created");  
	}  
}  
class TestSuper3{  
	public static void main(String args[]){  
		Dog d=new Dog();  
	}
}  
\end{lstlisting}%
\lthtmlfigureZ
\lthtmlcheckvsize\clearpage}

{\newpage\clearpage
\lthtmlfigureA{lstlisting2784}%
\begin{lstlisting}
animal is created
dog is created
\end{lstlisting}%
\lthtmlfigureZ
\lthtmlcheckvsize\clearpage}

{\newpage\clearpage
\lthtmlfigureA{lstlisting2786}%
\begin{lstlisting}
class Person{  
	int id;  
	String name;  
	Person(int id,String name){  
		this.id=id;  
		this.name=name;  
	}  
}  
class Emp extends Person{  
	float salary;  
	Emp(int id,String name,float salary){  
		super(id,name);//reusing parent constructor  
		this.salary=salary;  
	}  
	void display(){System.out.println(id+" "+name+" "+salary);}  
}  
class TestSuper5{  
	public static void main(String[] args){  
		Emp e1=new Emp(1,"ankit",45000f);  
		e1.display();  
	}
}  
\end{lstlisting}%
\lthtmlfigureZ
\lthtmlcheckvsize\clearpage}

{\newpage\clearpage
\lthtmlfigureA{lstlisting2792}%
\begin{lstlisting}
1 ankit 45000
\end{lstlisting}%
\lthtmlfigureZ
\lthtmlcheckvsize\clearpage}

\stepcounter{subsection}
\stepcounter{subsubsection}
{\newpage\clearpage
\lthtmlfigureA{lstlisting2798}%
\begin{lstlisting}
class Bike9{  
 final int speedlimit=90;//final variable  
 void run(){  
  speedlimit=400;  
 }  
 public static void main(String args[]){  
 Bike9 obj=new  Bike9();  
 obj.run();  
 }  
}//end of class  
\end{lstlisting}%
\lthtmlfigureZ
\lthtmlcheckvsize\clearpage}

\stepcounter{subsubsection}
{\newpage\clearpage
\lthtmlfigureA{lstlisting2804}%
\begin{lstlisting}
class Bike{  
  final void run(){System.out.println("running");}  
}  
\par
class Honda extends Bike{  
   void run(){System.out.println("running safely with 100kmph");}  
\par
public static void main(String args[]){  
   Honda honda= new Honda();  
   honda.run();  
   }  
}  
\end{lstlisting}%
\lthtmlfigureZ
\lthtmlcheckvsize\clearpage}

\stepcounter{subsubsection}
{\newpage\clearpage
\lthtmlfigureA{lstlisting2811}%
\begin{lstlisting}
final class Bike{}  
\par
class Honda1 extends Bike{  
  void run(){System.out.println("running safely with 100kmph");}  
\par
public static void main(String args[]){  
  Honda1 honda= new Honda1();  
  honda.run();  
  }  
}  
\end{lstlisting}%
\lthtmlfigureZ
\lthtmlcheckvsize\clearpage}

{\newpage\clearpage
\lthtmlfigureA{lstlisting2817}%
\begin{lstlisting}
class Bike{  
  final void run(){System.out.println("running...");}  
}  
class Honda2 extends Bike{  
   public static void main(String args[]){  
    new Honda2().run();  
   }  
}  
\end{lstlisting}%
\lthtmlfigureZ
\lthtmlcheckvsize\clearpage}

\stepcounter{subsubsection}
{\newpage\clearpage
\lthtmlfigureA{lstlisting2823}%
\begin{lstlisting}
class Student{  
	int id;  
	String name;  
	final String PAN_CARD_NUMBER;  
	...  
}  
\end{lstlisting}%
\lthtmlfigureZ
\lthtmlcheckvsize\clearpage}

{\newpage\clearpage
\lthtmlfigureA{lstlisting2826}%
\begin{lstlisting}
class Bike10{  
  final int speedlimit;//blank final variable  
\par
Bike10(){  
  speedlimit=70;  
  System.out.println(speedlimit);  
  }  
\par
public static void main(String args[]){  
    new Bike10();  
 }  
}  
\end{lstlisting}%
\lthtmlfigureZ
\lthtmlcheckvsize\clearpage}

\stepcounter{subsubsection}
{\newpage\clearpage
\lthtmlfigureA{lstlisting2832}%
\begin{lstlisting}
class A{  
  static final int data;//static blank final variable  
  static{ data=50;}  
  public static void main(String args[]){  
    System.out.println(A.data);  
 }  
}  
\end{lstlisting}%
\lthtmlfigureZ
\lthtmlcheckvsize\clearpage}

\stepcounter{subsubsection}
{\newpage\clearpage
\lthtmlfigureA{lstlisting2837}%
\begin{lstlisting}
class Bike11{  
  int cube(final int n){  
   n=n+2;//can't be changed as n is final  
   n*n*n;  
  }  
  public static void main(String args[]){  
    Bike11 b=new Bike11();  
    b.cube(5);  
 }  
}  
\end{lstlisting}%
\lthtmlfigureZ
\lthtmlcheckvsize\clearpage}

\stepcounter{section}
{\newpage\clearpage
\lthtmlfigureA{lstlisting2856}%
\begin{lstlisting}
module com.javatpoint{  
\par
}  
\end{lstlisting}%
\lthtmlfigureZ
\lthtmlcheckvsize\clearpage}

{\newpage\clearpage
\lthtmlfigureA{lstlisting2862}%
\begin{lstlisting}
class Hello{  
    public static void main(String[] args){  
        System.out.println("Hello from the Java module");  
    }  
} 
\end{lstlisting}%
\lthtmlfigureZ
\lthtmlcheckvsize\clearpage}

{\newpage\clearpage
\lthtmlfigureA{lstlisting2867}%
\begin{lstlisting}
javac -d mods --module-source-path src/ --module com.javatpoint  
\end{lstlisting}%
\lthtmlfigureZ
\lthtmlcheckvsize\clearpage}

{\newpage\clearpage
\lthtmlfigureA{lstlisting2869}%
\begin{lstlisting}
java --module-path mods/ --module com.javatpoint/com.javatpoint.Hello  
\end{lstlisting}%
\lthtmlfigureZ
\lthtmlcheckvsize\clearpage}

{\newpage\clearpage
\lthtmlfigureA{lstlisting2871}%
\begin{lstlisting}
Hello from the Java module
\end{lstlisting}%
\lthtmlfigureZ
\lthtmlcheckvsize\clearpage}

{\newpage\clearpage
\lthtmlfigureA{lstlisting2873}%
\begin{lstlisting}
javap mods/com.javatpoint/module-info.class  
\end{lstlisting}%
\lthtmlfigureZ
\lthtmlcheckvsize\clearpage}

{\newpage\clearpage
\lthtmlfigureA{lstlisting2875}%
\begin{lstlisting}
Compiled from "module-info.java"  
module com.javatpoint {  
  requires java.base;  
}  
\end{lstlisting}%
\lthtmlfigureZ
\lthtmlcheckvsize\clearpage}

\stepcounter{section}
{\newpage\clearpage
\lthtmlfigureA{lstlisting2884}%
\begin{lstlisting}
//save as Simple.java  
package mypack;  
public class Simple{  
 public static void main(String args[]){  
    System.out.println("Welcome to package");  
   }  
}  
\end{lstlisting}%
\lthtmlfigureZ
\lthtmlcheckvsize\clearpage}

{\newpage\clearpage
\lthtmlfigureA{lstlisting2887}%
\begin{lstlisting}
javac -d directory javafilename  
\end{lstlisting}%
\lthtmlfigureZ
\lthtmlcheckvsize\clearpage}

{\newpage\clearpage
\lthtmlfigureA{lstlisting2889}%
\begin{lstlisting}
javac -d . Simple.java  
\end{lstlisting}%
\lthtmlfigureZ
\lthtmlcheckvsize\clearpage}

{\newpage\clearpage
\lthtmlfigureA{lstlisting2908}%
\begin{lstlisting}
//save by A.java  
package pack;  
public class A{  
  public void msg(){System.out.println("Hello");}  
}  
//save by B.java  
package mypack;  
import pack.*;  
\par
class B{  
  public static void main(String args[]){  
   A obj = new A();  
   obj.msg();  
  }  
}  
\end{lstlisting}%
\lthtmlfigureZ
\lthtmlcheckvsize\clearpage}

{\newpage\clearpage
\lthtmlfigureA{lstlisting2914}%
\begin{lstlisting}
//save by A.java  
\par
package pack;  
public class A{  
  public void msg(){System.out.println("Hello");}  
}  
//save by B.java  
package mypack;  
import pack.A;  
\par
class B{  
  public static void main(String args[]){  
   A obj = new A();  
   obj.msg();  
  }  
}  
\end{lstlisting}%
\lthtmlfigureZ
\lthtmlcheckvsize\clearpage}

{\newpage\clearpage
\lthtmlfigureA{lstlisting2918}%
\begin{lstlisting}
Hello
\end{lstlisting}%
\lthtmlfigureZ
\lthtmlcheckvsize\clearpage}

{\newpage\clearpage
\lthtmlfigureA{lstlisting2920}%
\begin{lstlisting}
//save by A.java  
package pack;  
public class A{  
  public void msg(){System.out.println("Hello");}  
}  
//save by B.java  
package mypack;  
class B{  
  public static void main(String args[]){  
   pack.A obj = new pack.A();//using fully qualified name  
   obj.msg();  
  }  
}  
\end{lstlisting}%
\lthtmlfigureZ
\lthtmlcheckvsize\clearpage}

\stepcounter{subsection}
\stepcounter{subsubsection}
{\newpage\clearpage
\lthtmlfigureA{lstlisting2928}%
\begin{lstlisting}
package com.javatpoint.core;  
class Simple{  
  public static void main(String args[]){  
   System.out.println("Hello subpackage");  
  }  
}  
\end{lstlisting}%
\lthtmlfigureZ
\lthtmlcheckvsize\clearpage}

{\newpage\clearpage
\lthtmlfigureA{lstlisting2944}%
\begin{lstlisting}
e:\sources> javac -d c:\classes Simple.java
\end{lstlisting}%
\lthtmlfigureZ
\lthtmlcheckvsize\clearpage}

{\newpage\clearpage
\lthtmlfigureA{lstlisting2947}%
\begin{lstlisting}
e:\sources> set classpath=c:\classes;.;
e:\sources> java mypack.Simple
\end{lstlisting}%
\lthtmlfigureZ
\lthtmlcheckvsize\clearpage}

{\newpage\clearpage
\lthtmlfigureA{lstlisting2953}%
\begin{lstlisting}
e:\sources> java -classpath c:\classes mypack.Simple
\end{lstlisting}%
\lthtmlfigureZ
\lthtmlcheckvsize\clearpage}

\stepcounter{section}
{\newpage\clearpage
\lthtmlfigureA{lstlisting2958}%
\begin{lstlisting}
//A Java class which is a fully encapsulated class.  
//It has a private data member and getter and setter methods.  
package com.javatpoint;  
public class Student{  
	//private data member  
	private String name;  
	//getter method for name  
	public String getName(){  
		return name;  
	}  
	//setter method for name  
	public void setName(String name){  
		this.name=name  
	}  
}  
\end{lstlisting}%
\lthtmlfigureZ
\lthtmlcheckvsize\clearpage}

{\newpage\clearpage
\lthtmlfigureA{lstlisting2963}%
\begin{lstlisting}
//A Java class to test the encapsulated class.  
package com.javatpoint;  
class Test{  
	public static void main(String[] args){  
		//creating instance of the encapsulated class  
		Student s=new Student();  
		//setting value in the name member  
		s.setName("vijay");  
		//getting value of the name member  
		System.out.println(s.getName());  
	}  
}  
\end{lstlisting}%
\lthtmlfigureZ
\lthtmlcheckvsize\clearpage}

{\newpage\clearpage
\lthtmlfigureA{lstlisting2970}%
\begin{lstlisting}
//A Java class which has only getter methods.  
public class Student{  
	//private data member  
	private String college="AKG";  
	//getter method for college  
	public String getCollege(){  
		return college;  
	}  
}  
\end{lstlisting}%
\lthtmlfigureZ
\lthtmlcheckvsize\clearpage}

{\newpage\clearpage
\lthtmlfigureA{lstlisting2974}%
\begin{lstlisting}
s.setCollege("KITE");//will render compile time error  
\end{lstlisting}%
\lthtmlfigureZ
\lthtmlcheckvsize\clearpage}

{\newpage\clearpage
\lthtmlfigureA{lstlisting2977}%
\begin{lstlisting}
//A Java class which has only setter methods.  
public class Student{  
	//private data member  
	private String college;  
	//getter method for college  
	public void setCollege(String college){  
		this.college=college;  
	}  
}  
\end{lstlisting}%
\lthtmlfigureZ
\lthtmlcheckvsize\clearpage}

{\newpage\clearpage
\lthtmlfigureA{lstlisting2980}%
\begin{lstlisting}
System.out.println(s.getCollege());//Compile Time Error, because there is no such method  
System.out.println(s.college);//Compile Time Error, because the college data member is private.   
//So, it can't be accessed from outside the class  
\end{lstlisting}%
\lthtmlfigureZ
\lthtmlcheckvsize\clearpage}

{\newpage\clearpage
\lthtmlfigureA{lstlisting2984}%
\begin{lstlisting}
//A Account class which is a fully encapsulated class.  
//It has a private data member and getter and setter methods.  
class Account {  
	//private data members  
	private long acc_no;  
	private String name,email;  
	private float amount;  
	//public getter and setter methods  
	public long getAcc_no() {  
			return acc_no;  
	}  
	public void setAcc_no(long acc_no) {  
			this.acc_no = acc_no;  
	}  
	public String getName() {  
			return name;  
	}  
	public void setName(String name) {  
			this.name = name;  
	}  
	public String getEmail() {  
			return email;  
	}  
	public void setEmail(String email) {  
			this.email = email;  
	}  
	public float getAmount() {  
			return amount;  
	}  
	public void setAmount(float amount) {  
			this.amount = amount;  
	}  
}  
\end{lstlisting}%
\lthtmlfigureZ
\lthtmlcheckvsize\clearpage}

{\newpage\clearpage
\lthtmlfigureA{lstlisting2995}%
\begin{lstlisting}
//A Java class to test the encapsulated class Account.  
public class TestEncapsulation {  
	public static void main(String[] args) {  
			//creating instance of Account class  
			Account acc=new Account();  
			//setting values through setter methods  
			acc.setAcc_no(7560504000L);  
			acc.setName("Sonoo Jaiswal");  
			acc.setEmail("sonoojaiswal@javatpoint.com");  
			acc.setAmount(500000f);  
			//getting values through getter methods  
			System.out.println(acc.getAcc_no()+" "+acc.getName()+" "+acc.getEmail()+" "+acc.getAmount());  
	}  
} 
\end{lstlisting}%
\lthtmlfigureZ
\lthtmlcheckvsize\clearpage}

{\newpage\clearpage
\lthtmlfigureA{lstlisting2998}%
\begin{lstlisting}
7560504000 Sonoo Jaiswal sonoojaiswal@javatpoint.com 500000.0
\end{lstlisting}%
\lthtmlfigureZ
\lthtmlcheckvsize\clearpage}

\stepcounter{section}
\stepcounter{subsection}
\stepcounter{subsection}
{\newpage\clearpage
\lthtmlfigureA{lstlisting3012}%
\begin{lstlisting}
abstract class Bike{  
  abstract void run();  
}  
class Honda4 extends Bike{  
	void run(){System.out.println("running safely");}  
	public static void main(String args[]){  
	 Bike obj = new Honda4();  
	 obj.run();  
	}  
}  
\end{lstlisting}%
\lthtmlfigureZ
\lthtmlcheckvsize\clearpage}

{\newpage\clearpage
\lthtmlfigureA{lstlisting3017}%
\begin{lstlisting}
running safely
\end{lstlisting}%
\lthtmlfigureZ
\lthtmlcheckvsize\clearpage}

\stepcounter{subsection}
{\newpage\clearpage
\lthtmlfigureA{lstlisting3021}%
\begin{lstlisting}
abstract class Shape{  
	abstract void draw();  
}  
//In real scenario, implementation is provided by others i.e. unknown by end user  
class Rectangle extends Shape{  
	void draw(){System.out.println("drawing rectangle");}  
}  
class Circle1 extends Shape{  
	void draw(){System.out.println("drawing circle");}  
}  
//In real scenario, method is called by programmer or user  
class TestAbstraction1{  
	public static void main(String args[]){  
	Shape s=new Circle1();//In a real scenario, object is provided through method, e.g., getShape() method  
		s.draw();  
	}  
}  
\end{lstlisting}%
\lthtmlfigureZ
\lthtmlcheckvsize\clearpage}

{\newpage\clearpage
\lthtmlfigureA{lstlisting3027}%
\begin{lstlisting}
drawing circle
\end{lstlisting}%
\lthtmlfigureZ
\lthtmlcheckvsize\clearpage}

\stepcounter{subsection}
{\newpage\clearpage
\lthtmlfigureA{lstlisting3031}%
\begin{lstlisting}
//Example of an abstract class that has abstract and non-abstract methods  
 abstract class Bike{  
   Bike(){System.out.println("bike is created");}  
   abstract void run();  
   void changeGear(){System.out.println("gear changed");}  
 }  
//Creating a Child class which inherits Abstract class  
 class Honda extends Bike{  
 void run(){System.out.println("running safely..");}  
 }  
//Creating a Test class which calls abstract and non-abstract methods  
 class TestAbstraction2{  
 public static void main(String args[]){  
  Bike obj = new Honda();  
  obj.run();  
  obj.changeGear();  
 }  
}  
\end{lstlisting}%
\lthtmlfigureZ
\lthtmlcheckvsize\clearpage}

{\newpage\clearpage
\lthtmlfigureA{lstlisting3037}%
\begin{lstlisting}
       bike is created
       running safely..
       gear changed
\end{lstlisting}%
\lthtmlfigureZ
\lthtmlcheckvsize\clearpage}

{\newpage\clearpage
\lthtmlfigureA{lstlisting3039}%
\begin{lstlisting}
class Bike12{  
	abstract void run();  
}  
\end{lstlisting}%
\lthtmlfigureZ
\lthtmlcheckvsize\clearpage}

{\newpage\clearpage
\lthtmlfigureA{lstlisting3042}%
\begin{lstlisting}
compile time error
\end{lstlisting}%
\lthtmlfigureZ
\lthtmlcheckvsize\clearpage}

\stepcounter{section}
{\newpage\clearpage
\lthtmlfigureA{lstlisting3049}%
\begin{lstlisting}
interface <interface_name>{  
    // declare constant fields  
    // declare methods that abstract   
    // by default.  
}
\end{lstlisting}%
\lthtmlfigureZ
\lthtmlcheckvsize\clearpage}

\stepcounter{subsection}
{\newpage\clearpage
\lthtmlfigureA{lstlisting3063}%
\begin{lstlisting}
interface printable{  
	void print();  
}  
class A6 implements printable{  
	public void print(){System.out.println("Hello");}  
\par
public static void main(String args[]){  
		A6 obj = new A6();  
		obj.print();  
	}  
}  
\end{lstlisting}%
\lthtmlfigureZ
\lthtmlcheckvsize\clearpage}

\stepcounter{subsection}
{\newpage\clearpage
\lthtmlfigureA{lstlisting3072}%
\begin{lstlisting}
interface Bank{  
	float rateOfInterest();  
}  
class SBI implements Bank{  
	public float rateOfInterest(){return 9.15f;}  
}  
class PNB implements Bank{  
	public float rateOfInterest(){return 9.7f;}  
}  
class TestInterface2{  
	public static void main(String[] args){  
		Bank b=new SBI();  
		System.out.println("ROI: "+b.rateOfInterest());  
	}
}  
\end{lstlisting}%
\lthtmlfigureZ
\lthtmlcheckvsize\clearpage}

{\newpage\clearpage
\lthtmlfigureA{lstlisting3078}%
\begin{lstlisting}
ROI: 9.15
\end{lstlisting}%
\lthtmlfigureZ
\lthtmlcheckvsize\clearpage}

\stepcounter{subsection}
{\newpage\clearpage
\lthtmlfigureA{lstlisting3086}%
\begin{lstlisting}
interface Printable{  
	void print();  
}  
interface Showable{  
	void show();  
}  
class A7 implements Printable,Showable{  
	public void print(){System.out.println("Hello");}  
	public void show(){System.out.println("Welcome");}  
\par
public static void main(String args[]){  
		A7 obj = new A7();  
		obj.print();  
		obj.show();  
	}  
}  
\end{lstlisting}%
\lthtmlfigureZ
\lthtmlcheckvsize\clearpage}

\stepcounter{subsection}
{\newpage\clearpage
\lthtmlfigureA{lstlisting3094}%
\begin{lstlisting}
interface Printable{  
	void print();  
}  
interface Showable extends Printable{  
	void show();  
}  
class TestInterface4 implements Showable{  
	public void print(){System.out.println("Hello");}  
	public void show(){System.out.println("Welcome");}  
\par
public static void main(String args[]){  
		TestInterface4 obj = new TestInterface4();  
		obj.print();  
		obj.show();  
	}  
}  
\end{lstlisting}%
\lthtmlfigureZ
\lthtmlcheckvsize\clearpage}

{\newpage\clearpage
\lthtmlfigureA{lstlisting3101}%
\begin{lstlisting}
Hello
Welcome
\end{lstlisting}%
\lthtmlfigureZ
\lthtmlcheckvsize\clearpage}

\stepcounter{subsection}
{\newpage\clearpage
\lthtmlfigureA{lstlisting3105}%
\begin{lstlisting}
interface Drawable{  
void draw();  
default void msg(){System.out.println("default method");}  
}  
class Rectangle implements Drawable{  
public void draw(){System.out.println("drawing rectangle");}  
}  
class TestInterfaceDefault{  
public static void main(String args[]){  
Drawable d=new Rectangle();  
d.draw();  
d.msg();  
}}  
\end{lstlisting}%
\lthtmlfigureZ
\lthtmlcheckvsize\clearpage}

{\newpage\clearpage
\lthtmlfigureA{lstlisting3110}%
\begin{lstlisting}
drawing rectangle
default method
\end{lstlisting}%
\lthtmlfigureZ
\lthtmlcheckvsize\clearpage}

\stepcounter{subsection}
{\newpage\clearpage
\lthtmlfigureA{lstlisting3114}%
\begin{lstlisting}
interface Drawable{  
	void draw();  
	static int cube(int x){return x*x*x;}  
}  
class Rectangle implements Drawable{  
	public void draw(){System.out.println("drawing rectangle");}  
}  
\par
class TestInterfaceStatic{  
	public static void main(String args[]){  
		Drawable d=new Rectangle();  
		d.draw();  
		System.out.println(Drawable.cube(3));  
	}
}  
\end{lstlisting}%
\lthtmlfigureZ
\lthtmlcheckvsize\clearpage}

{\newpage\clearpage
\lthtmlfigureA{lstlisting3119}%
\begin{lstlisting}
drawing rectangle
27
\end{lstlisting}%
\lthtmlfigureZ
\lthtmlcheckvsize\clearpage}

\stepcounter{subsection}
{\newpage\clearpage
\lthtmlfigureA{lstlisting3122}%
\begin{lstlisting}
interface printable{  
 void print();  
 interface MessagePrintable{  
   void msg();  
 }  
}  
\end{lstlisting}%
\lthtmlfigureZ
\lthtmlcheckvsize\clearpage}

{\newpage\clearpage
\lthtmlfigureA{lstlisting3125}%
\begin{lstlisting}
interface Showable{  
  void show();  
  interface Message{  
   void msg();  
  }  
}  
class TestNestedInterface1 implements Showable.Message{  
 public void msg(){System.out.println("Hello nested interface");}  
\par
public static void main(String args[]){  
  Showable.Message message=new TestNestedInterface1();//upcasting here  
  message.msg();  
 }  
}  
\end{lstlisting}%
\lthtmlfigureZ
\lthtmlcheckvsize\clearpage}

{\newpage\clearpage
\lthtmlfigureA{lstlisting3130}%
\begin{lstlisting}
hello nested interface
\end{lstlisting}%
\lthtmlfigureZ
\lthtmlcheckvsize\clearpage}

{\newpage\clearpage
\lthtmlfigureA{lstlisting3132}%
\begin{lstlisting}
class A{  
  interface Message{  
   void msg();  
  }  
}  
\par
class TestNestedInterface2 implements A.Message{  
 public void msg(){System.out.println("Hello nested interface");}  
\par
public static void main(String args[]){  
  A.Message message=new TestNestedInterface2();//upcasting here  
  message.msg();  
 }  
}  
\end{lstlisting}%
\lthtmlfigureZ
\lthtmlcheckvsize\clearpage}

\stepcounter{section}
{\newpage\clearpage
\lthtmlfigureA{lstlisting3151}%
\begin{lstlisting}
class Subclass-name extends Superclass-name  
{  
   //methods and fields  
}
\end{lstlisting}%
\lthtmlfigureZ
\lthtmlcheckvsize\clearpage}

{\newpage\clearpage
\lthtmlfigureA{lstlisting3160}%
\begin{lstlisting}
class Employee{  
 float salary=40000;  
}  
class Programmer extends Employee{  
 int bonus=10000;  
 public static void main(String args[]){  
   Programmer p=new Programmer();  
   System.out.println("Programmer salary is:"+p.salary);  
   System.out.println("Bonus of Programmer is:"+p.bonus);  
	}  
}  
\end{lstlisting}%
\lthtmlfigureZ
\lthtmlcheckvsize\clearpage}

{\newpage\clearpage
\lthtmlfigureA{lstlisting3164}%
\begin{lstlisting}
 Programmer salary is:40000.0
 Bonus of programmer is:10000
\end{lstlisting}%
\lthtmlfigureZ
\lthtmlcheckvsize\clearpage}

\stepcounter{subsection}
\stepcounter{subsubsection}
{\newpage\clearpage
\lthtmlfigureA{lstlisting3179}%
\begin{lstlisting}
class Animal{  
	void eat(){System.out.println("eating...");}  
}  
class Dog extends Animal{  
	void bark(){System.out.println("barking...");}  
}  
class TestInheritance{  
	public static void main(String args[]){  
		Dog d=new Dog();  
		d.bark();  
		d.eat();  
	}
} 
\end{lstlisting}%
\lthtmlfigureZ
\lthtmlcheckvsize\clearpage}

{\newpage\clearpage
\lthtmlfigureA{lstlisting3184}%
\begin{lstlisting}
barking...
eating...
\end{lstlisting}%
\lthtmlfigureZ
\lthtmlcheckvsize\clearpage}

\stepcounter{subsubsection}
{\newpage\clearpage
\lthtmlfigureA{lstlisting3188}%
\begin{lstlisting}
class Animal{  
	void eat(){System.out.println("eating...");}  
}  
class Dog extends Animal{  
	void bark(){System.out.println("barking...");}  
}  
class BabyDog extends Dog{  
	void weep(){System.out.println("weeping...");}  
}  
class TestInheritance2{  
	public static void main(String args[]){  
		BabyDog d=new BabyDog();  
		d.weep();  
		d.bark();  
		d.eat();  
	}
} 
\end{lstlisting}%
\lthtmlfigureZ
\lthtmlcheckvsize\clearpage}

{\newpage\clearpage
\lthtmlfigureA{lstlisting3194}%
\begin{lstlisting}
weeping...
barking...
eating...
\end{lstlisting}%
\lthtmlfigureZ
\lthtmlcheckvsize\clearpage}

\stepcounter{subsubsection}
{\newpage\clearpage
\lthtmlfigureA{lstlisting3198}%
\begin{lstlisting}
class Animal{  
void eat(){System.out.println("eating...");}  
}  
class Dog extends Animal{  
void bark(){System.out.println("barking...");}  
}  
class Cat extends Animal{  
void meow(){System.out.println("meowing...");}  
}  
class TestInheritance3{  
public static void main(String args[]){  
Cat c=new Cat();  
c.meow();  
c.eat();  
//c.bark();//C.T.Error  
}
}
\end{lstlisting}%
\lthtmlfigureZ
\lthtmlcheckvsize\clearpage}

{\newpage\clearpage
\lthtmlfigureA{lstlisting3204}%
\begin{lstlisting}
meowing...
eating...
\end{lstlisting}%
\lthtmlfigureZ
\lthtmlcheckvsize\clearpage}

\stepcounter{section}
\stepcounter{subsection}
{\newpage\clearpage
\lthtmlfigureA{lstlisting3218}%
\begin{lstlisting}
class A{}  
class B extends A{}  
A a=new B();//upcasting  
\end{lstlisting}%
\lthtmlfigureZ
\lthtmlcheckvsize\clearpage}

{\newpage\clearpage
\lthtmlfigureA{lstlisting3222}%
\begin{lstlisting}
interface I{}  
class A{}  
class B extends A implements I{}  
\end{lstlisting}%
\lthtmlfigureZ
\lthtmlcheckvsize\clearpage}

\stepcounter{subsubsection}
{\newpage\clearpage
\lthtmlfigureA{lstlisting3231}%
\begin{lstlisting}
class Bike{  
  void run(){System.out.println("running");}  
}  
class Splendor extends Bike{  
  void run(){System.out.println("running safely with 60km");}  
\par
public static void main(String args[]){  
    Bike b = new Splendor();//upcasting  
    b.run();  
  }  
}  
\end{lstlisting}%
\lthtmlfigureZ
\lthtmlcheckvsize\clearpage}

{\newpage\clearpage
\lthtmlfigureA{lstlisting3236}%
\begin{lstlisting}
running safely with 60km.
\end{lstlisting}%
\lthtmlfigureZ
\lthtmlcheckvsize\clearpage}

\stepcounter{subsubsection}
{\newpage\clearpage
\lthtmlfigureA{lstlisting3244}%
\begin{lstlisting}
class Bank{  
	float getRateOfInterest(){return 0;}  
}  
class SBI extends Bank{  
	float getRateOfInterest(){return 8.4f;}  
}  
class ICICI extends Bank{  
	float getRateOfInterest(){return 7.3f;}  
}  
class AXIS extends Bank{  
	float getRateOfInterest(){return 9.7f;}  
}  
class TestPolymorphism{  
	public static void main(String args[]){  
		Bank b;  
		b=new SBI();  
		System.out.println("SBI Rate of Interest: "+b.getRateOfInterest());  
		b=new ICICI();  
		System.out.println("ICICI Rate of Interest: "+b.getRateOfInterest());  
		b=new AXIS();  
		System.out.println("AXIS Rate of Interest: "+b.getRateOfInterest());  
	}  
}  
\end{lstlisting}%
\lthtmlfigureZ
\lthtmlcheckvsize\clearpage}

{\newpage\clearpage
\lthtmlfigureA{lstlisting3251}%
\begin{lstlisting}
SBI Rate of Interest: 8.4
ICICI Rate of Interest: 7.3
AXIS Rate of Interest: 9.7
\end{lstlisting}%
\lthtmlfigureZ
\lthtmlcheckvsize\clearpage}

\stepcounter{subsection}
\stepcounter{subsubsection}
{\newpage\clearpage
\lthtmlfigureA{lstlisting3261}%
\begin{lstlisting}
/Java Program to demonstrate why we need method overriding  
//Here, we are calling the method of parent class with child  
//class object.  
//Creating a parent class  
class Vehicle{  
  void run(){System.out.println("Vehicle is running");}  
}  
//Creating a child class  
class Bike extends Vehicle{  
  public static void main(String args[]){  
  //creating an instance of child class  
  Bike obj = new Bike();  
  //calling the method with child class instance  
  obj.run();  
  }  
}  
\end{lstlisting}%
\lthtmlfigureZ
\lthtmlcheckvsize\clearpage}

\stepcounter{subsubsection}
{\newpage\clearpage
\lthtmlfigureA{lstlisting3266}%
\begin{lstlisting}
//Java Program to illustrate the use of Java Method Overriding  
//Creating a parent class.  
class Vehicle{  
  //defining a method  
  void run(){System.out.println("Vehicle is running");}  
}  
//Creating a child class  
class Bike2 extends Vehicle{  
  //defining the same method as in the parent class  
  void run(){System.out.println("Bike is running safely");}  
\par
public static void main(String args[]){  
  Bike2 obj = new Bike2();//creating object  
  obj.run();//calling method  
  }  
}  
\end{lstlisting}%
\lthtmlfigureZ
\lthtmlcheckvsize\clearpage}

{\newpage\clearpage
\lthtmlfigureA{lstlisting3271}%
\begin{lstlisting}
Bike is running safely
\end{lstlisting}%
\lthtmlfigureZ
\lthtmlcheckvsize\clearpage}

\stepcounter{subsubsection}
\stepcounter{subsubsection}
{\newpage\clearpage
\lthtmlfigureA{lstlisting3277}%
\begin{lstlisting}
class Adder{  
	static int add(int a,int b){return a+b;}  
	static int add(int a,int b,int c){return a+b+c;}  
}  
class TestOverloading1{  
	public static void main(String[] args){  
	System.out.println(Adder.add(11,11));  
	System.out.println(Adder.add(11,11,11));  
}}  
\end{lstlisting}%
\lthtmlfigureZ
\lthtmlcheckvsize\clearpage}

{\newpage\clearpage
\lthtmlfigureA{lstlisting3282}%
\begin{lstlisting}
22
33
\end{lstlisting}%
\lthtmlfigureZ
\lthtmlcheckvsize\clearpage}

\stepcounter{subsubsection}
{\newpage\clearpage
\lthtmlfigureA{lstlisting3285}%
\begin{lstlisting}
class Adder{  
	static int add(int a, int b){return a+b;}  
	static double add(double a, double b){return a+b;}  
}  
class TestOverloading2{  
	public static void main(String[] args){  
	System.out.println(Adder.add(11,11));  
	System.out.println(Adder.add(12.3,12.6));  
}}  
\end{lstlisting}%
\lthtmlfigureZ
\lthtmlcheckvsize\clearpage}

{\newpage\clearpage
\lthtmlfigureA{lstlisting3290}%
\begin{lstlisting}
22
24.9
\end{lstlisting}%
\lthtmlfigureZ
\lthtmlcheckvsize\clearpage}

\stepcounter{subsubsection}
\stepcounter{chapter}
\stepcounter{section}
{\newpage\clearpage
\lthtmlfigureA{lstlisting3314}%
\begin{lstlisting}
// Java Program to Illustrate Error
// Stack overflow error via infinite recursion
\par
// Class 1
class StackOverflow {
\par
// method of this class
    public static void test(int i)
    {
        // No correct as base condition leads to
        // non-stop recursion.
        if (i == 0)
            return;
        else {
            test(i++);
        }
    }
}
\par
// Class 2
// Main class
public class GFG {
\par
// Main driver method
    public static void main(String[] args)
    {
        // Testing for error by passing
        // custom integer as an argument
        StackOverflow.test(5);
    }
}
\end{lstlisting}%
\lthtmlfigureZ
\lthtmlcheckvsize\clearpage}

{\newpage\clearpage
\lthtmlfigureA{lstlisting3319}%
\begin{lstlisting}
// Java Program to Illustrate Run-time Errors
\par
// Main class
class GFG {
\par
// Main driver method
    public static void main(String args[])
    {
\par
// Declaring and initializing numbers
        int a = 2, b = 8, c = 6;
\par
if (a > b && a > c)
            System.out.println(a
                               + " is the largest Number");
        else if (b > a && b > c)
            System.out.println(b
                               + " is the smallest Number");
\par
// The correct message should have been
        // System.out.println
        // (b+" is the largest Number"); to make logic
        else
            System.out.println(c
                               + " is the largest Number");
    }
}
\end{lstlisting}%
\lthtmlfigureZ
\lthtmlcheckvsize\clearpage}

{\newpage\clearpage
\lthtmlfigureA{lstlisting3322}%
\begin{lstlisting}
8 is the smallest Number
\end{lstlisting}%
\lthtmlfigureZ
\lthtmlcheckvsize\clearpage}

\stepcounter{subsection}
{\newpage\clearpage
\lthtmlfigureA{lstlisting3330}%
\begin{lstlisting}
// Java program illustrating exception thrown
// by AritmeticExcpetion class
\par
// Main class
class GFG {
\par
// main driver method
    public static void main(String[] args)
    {
        int a = 5, b = 0;
\par
// Try-catch block to check and handle exceptions
        try {
\par
// Attempting to divide by zero
            int c = a / b;
        }
        catch (ArithmeticException e) {
\par
// Displaying line number where exception occured
            // using printStackTrace() method
            e.printStackTrace();
        }
    }
}
\end{lstlisting}%
\lthtmlfigureZ
\lthtmlcheckvsize\clearpage}

{\newpage\clearpage
\lthtmlfigureA{lstlisting3334}%
\begin{lstlisting}
java.lang.ArithmeticException: / by zero ....
\end{lstlisting}%
\lthtmlfigureZ
\lthtmlcheckvsize\clearpage}

\stepcounter{subsection}
{\newpage\clearpage
\lthtmlfigureA{lstlisting3349}%
\begin{lstlisting}
\par
// Java program to demonstrate how exception is thrown. 
class ThrowsExecept{ 
\par
public static void main(String args[]){ 
\par
String str = null; 
        System.out.println(str.length()); 
\par
} 
} 
\end{lstlisting}%
\lthtmlfigureZ
\lthtmlcheckvsize\clearpage}

{\newpage\clearpage
\lthtmlfigureA{lstlisting3352}%
\begin{lstlisting}
Exception in thread "main" java.lang.NullPointerException
    at ThrowsExecp.main(File.java:8)
\end{lstlisting}%
\lthtmlfigureZ
\lthtmlcheckvsize\clearpage}

{\newpage\clearpage
\lthtmlfigureA{lstlisting3354}%
\begin{lstlisting}
\par
// Java program to demonstrate exception is thrown 
// how the runTime system searches th call stack 
// to find appropriate exception handler. 
class ExceptionThrown 
{ 
    // It throws the Exception(ArithmeticException). 
    // Appropriate Exception handler is not found within this method. 
    static int divideByZero(int a, int b){ 
\par
// this statement will cause ArithmeticException(/ by zero) 
        int i = a/b;  
\par
return i; 
    } 
\par
// The runTime System searches the appropriate Exception handler 
    // in this method also but couldn't have found. So looking forward 
    // on the call stack. 
    static int computeDivision(int a, int b) { 
\par
int res =0; 
\par
try
        { 
          res = divideByZero(a,b); 
        } 
        // doesn't matches with ArithmeticException 
        catch(NumberFormatException ex) 
        { 
           System.out.println("NumberFormatException is occured");  
        } 
        return res; 
    } 
\par
// In this method found appropriate Exception handler. 
    // i.e. matching catch block. 
    public static void main(String args[]){ 
\par
int a = 1; 
        int b = 0; 
\par
try
        { 
            int i = computeDivision(a,b); 
\par
} 
\par
// matching ArithmeticException 
        catch(ArithmeticException ex) 
        { 
            // getMessage will print description of exception(here / by zero) 
            System.out.println(ex.getMessage()); 
        } 
    } 
} 
\par
\end{lstlisting}%
\lthtmlfigureZ
\lthtmlcheckvsize\clearpage}

{\newpage\clearpage
\lthtmlfigureA{lstlisting3361}%
\begin{lstlisting}
/ by zero.
\end{lstlisting}%
\lthtmlfigureZ
\lthtmlcheckvsize\clearpage}

\stepcounter{section}
\stepcounter{subsection}
\stepcounter{subsubsection}
{\newpage\clearpage
\lthtmlfigureA{lstlisting3385}%
\begin{lstlisting}
class ArithmeticExceptionDemo 
{ 
    public static void main(String args[]) 
    { 
        try { 
            int a = 30, b = 0; 
            int c = a/b;  // cannot divide by zero 
            System.out.println ("Result = " + c); 
        } 
        catch(ArithmeticException e) { 
            System.out.println ("Can't divide a number by 0"); 
        } 
    } 
} 
\end{lstlisting}%
\lthtmlfigureZ
\lthtmlcheckvsize\clearpage}

{\newpage\clearpage
\lthtmlfigureA{lstlisting3389}%
\begin{lstlisting}
Can't divide a number by 0
\end{lstlisting}%
\lthtmlfigureZ
\lthtmlcheckvsize\clearpage}

\stepcounter{subsubsection}
{\newpage\clearpage
\lthtmlfigureA{lstlisting3392}%
\begin{lstlisting}
\par
//Java program to demonstrate NullPointerException 
class NullPointerDemo 
{ 
    public static void main(String args[]) 
    { 
        try { 
            String a = null; //null value 
            System.out.println(a.charAt(0)); 
        } catch(NullPointerException e) { 
            System.out.println("NullPointerException.."); 
        } 
    } 
} 
\end{lstlisting}%
\lthtmlfigureZ
\lthtmlcheckvsize\clearpage}

\stepcounter{subsubsection}
{\newpage\clearpage
\lthtmlfigureA{lstlisting3397}%
\begin{lstlisting}
class StringIndexOutOfBound_Demo 
{ 
    public static void main(String args[]) 
    { 
        try { 
            String a = "This is like chipping "; // length is 22 
            char c = a.charAt(24); // accessing 25th element 
            System.out.println(c); 
        } 
        catch(StringIndexOutOfBoundsException e) { 
            System.out.println("StringIndexOutOfBoundsException"); 
        } 
    } 
} 
\end{lstlisting}%
\lthtmlfigureZ
\lthtmlcheckvsize\clearpage}

{\newpage\clearpage
\lthtmlfigureA{lstlisting3401}%
\begin{lstlisting}
StringIndexOutOfBoundsException
\end{lstlisting}%
\lthtmlfigureZ
\lthtmlcheckvsize\clearpage}

\stepcounter{subsubsection}
{\newpage\clearpage
\lthtmlfigureA{lstlisting3404}%
\begin{lstlisting}
import java.io.File; 
import java.io.FileNotFoundException; 
import java.io.FileReader; 
 class File_notFound_Demo { 
\par
public static void main(String args[])  { 
        try { 
\par
// Following file does not exist 
            File file = new File("E://file.txt"); 
\par
FileReader fr = new FileReader(file); 
        } catch (FileNotFoundException e) { 
           System.out.println("File does not exist"); 
        } 
    } 
} 
\end{lstlisting}%
\lthtmlfigureZ
\lthtmlcheckvsize\clearpage}

{\newpage\clearpage
\lthtmlfigureA{lstlisting3408}%
\begin{lstlisting}
File does not exist
\end{lstlisting}%
\lthtmlfigureZ
\lthtmlcheckvsize\clearpage}

\stepcounter{subsubsection}
{\newpage\clearpage
\lthtmlfigureA{lstlisting3411}%
\begin{lstlisting}
class  NumberFormat_Demo 
{ 
    public static void main(String args[]) 
    { 
        try { 
            // "akki" is not a number 
            int num = Integer.parseInt ("akki") ; 
\par
System.out.println(num); 
        } catch(NumberFormatException e) { 
            System.out.println("Number format exception"); 
        } 
    } 
} 
\end{lstlisting}%
\lthtmlfigureZ
\lthtmlcheckvsize\clearpage}

{\newpage\clearpage
\lthtmlfigureA{lstlisting3415}%
\begin{lstlisting}
Number format exception
\end{lstlisting}%
\lthtmlfigureZ
\lthtmlcheckvsize\clearpage}

\stepcounter{subsubsection}
{\newpage\clearpage
\lthtmlfigureA{lstlisting3418}%
\begin{lstlisting}
class ArrayIndexOutOfBound_Demo 
{ 
    public static void main(String args[]) 
    { 
        try{ 
            int a[] = new int[5]; 
            a[6] = 9; // accessing 7th element in an array of 
                      // size 5 
        } 
        catch(ArrayIndexOutOfBoundsException e){ 
            System.out.println ("Array Index is Out Of Bounds"); 
        } 
    } 
} 
\end{lstlisting}%
\lthtmlfigureZ
\lthtmlcheckvsize\clearpage}

{\newpage\clearpage
\lthtmlfigureA{lstlisting3422}%
\begin{lstlisting}
Array Index is Out Of Bounds
\end{lstlisting}%
\lthtmlfigureZ
\lthtmlcheckvsize\clearpage}

\stepcounter{section}
{\newpage\clearpage
\lthtmlfigureA{lstlisting3426}%
\begin{lstlisting}
	class MyException extends Exception
	\end{lstlisting}%
\lthtmlfigureZ
\lthtmlcheckvsize\clearpage}

{\newpage\clearpage
\lthtmlfigureA{lstlisting3428}%
\begin{lstlisting}
	MyException(){}
	\end{lstlisting}%
\lthtmlfigureZ
\lthtmlcheckvsize\clearpage}

{\newpage\clearpage
\lthtmlfigureA{lstlisting3431}%
\begin{lstlisting}
	MyException(String str)
	{
   super(str);
	}
	\end{lstlisting}%
\lthtmlfigureZ
\lthtmlcheckvsize\clearpage}

{\newpage\clearpage
\lthtmlfigureA{lstlisting3434}%
\begin{lstlisting}
		MyException me = new MyException(ԅxception detailsԩ;
    throw me;
	\end{lstlisting}%
\lthtmlfigureZ
\lthtmlcheckvsize\clearpage}

{\newpage\clearpage
\lthtmlfigureA{lstlisting3439}%
\begin{lstlisting}
\par
// Java program to demonstrate user defined exception 
\par
// This program throws an exception whenever balance 
// amount is below Rs 1000 
class MyException extends Exception 
{ 
    //store account information 
    private static int accno[] = {1001, 1002, 1003, 1004}; 
\par
private static String name[] = 
                 {"Nish", "Shubh", "Sush", "Abhi", "Akash"}; 
\par
private static double bal[] = 
         {10000.00, 12000.00, 5600.0, 999.00, 1100.55}; 
\par
// default constructor 
    MyException() {    } 
\par
// parametrized constructor 
    MyException(String str) { super(str); } 
\par
// write main() 
    public static void main(String[] args) 
    { 
        try  { 
            // display the heading for the table 
            System.out.println("ACCNO" + "\t" + "CUSTOMER" + 
                                           "\t" + "BALANCE"); 
\par
// display the actual account information 
            for (int i = 0; i < 5 ; i++) 
            { 
                System.out.println(accno[i] + "\t" + name[i] + 
                                               "\t" + bal[i]); 
\par
// display own exception if balance < 1000 
                if (bal[i] < 1000) 
                { 
                    MyException me = 
                       new MyException("Balance is less than 1000"); 
                    throw me; 
                } 
            } 
        } //end of try 
\par
catch (MyException e) { 
            e.printStackTrace(); 
        } 
    } 
} 
\end{lstlisting}%
\lthtmlfigureZ
\lthtmlcheckvsize\clearpage}

{\newpage\clearpage
\lthtmlfigureA{lstlisting3448}%
\begin{lstlisting}
 MyException: Balance is less than 1000
    at MyException.main(fileProperty.java:36)
\end{lstlisting}%
\lthtmlfigureZ
\lthtmlcheckvsize\clearpage}

{\newpage\clearpage
\lthtmlfigureA{lstlisting3450}%
\begin{lstlisting}
ACCNO    CUSTOMER    BALANCE
1001    Nish    10000.0
1002    Shubh    12000.0
1003    Sush    5600.0
1004    Abhi    999.0
\end{lstlisting}%
\lthtmlfigureZ
\lthtmlcheckvsize\clearpage}

\stepcounter{section}
\stepcounter{section}
\stepcounter{subsection}
{\newpage\clearpage
\lthtmlfigureA{lstlisting3461}%
\begin{lstlisting}
class TryCatch
{ 
    public static void main (String[] args)  
    { 
\par
// array of size 4. 
        int[] arr = new int[4]; 
        try
        { 
            int i = arr[4]; 
\par
// this statement will never execute 
            // as exception is raised by above statement 
            System.out.println("Inside try block"); 
        } 
        catch(ArrayIndexOutOfBoundsException ex) 
        { 
            System.out.println("Exception caught in Catch block"); 
        } 
\par
// rest program will be excuted 
        System.out.println("Outside try-catch clause"); 
    } 
} 
\end{lstlisting}%
\lthtmlfigureZ
\lthtmlcheckvsize\clearpage}

{\newpage\clearpage
\lthtmlfigureA{lstlisting3465}%
\begin{lstlisting}
Exception caught in Catch block
Outside try-catch clause
\end{lstlisting}%
\lthtmlfigureZ
\lthtmlcheckvsize\clearpage}

\stepcounter{subsection}
{\newpage\clearpage
\lthtmlfigureA{lstlisting3468}%
\begin{lstlisting}
class TryCatch 
{ 
    public static void main (String[] args)  
    { 
\par
// array of size 4. 
        int[] arr = new int[4]; 
\par
try
        { 
            int i = arr[4]; 
\par
// this statement will never execute 
            // as exception is raised by above statement 
            System.out.println("Inside try block"); 
        } 
\par
catch(ArrayIndexOutOfBoundsException ex) 
        { 
            System.out.println("Exception caught in catch block"); 
        } 
\par
finally
        { 
            System.out.println("finally block executed"); 
        } 
\par
// rest program will be executed 
        System.out.println("Outside try-catch-finally clause"); 
    } 
} 
\end{lstlisting}%
\lthtmlfigureZ
\lthtmlcheckvsize\clearpage}

{\newpage\clearpage
\lthtmlfigureA{lstlisting3473}%
\begin{lstlisting}
Exception caught in catch block
finally block executed
Outside try-catch-finally clause
\end{lstlisting}%
\lthtmlfigureZ
\lthtmlcheckvsize\clearpage}

\stepcounter{chapter}
\stepcounter{section}
\stepcounter{subsection}
\stepcounter{subsection}
\stepcounter{subsection}
{\newpage\clearpage
\lthtmlpictureA{tex2html_wrap9421}%
\rotatebox{90}{
\par
\begin{tiny}
\begin{tabular}{llll} \hline 
S.N.	& Modifier and Type	& Method	& Description \\\hline 
1 & void	& start()	& It is used to start the execution of the thread. \\
2 &	void	& run()	& It is used to do an action for a thread. \\
3	& static void	& sleep()	& It sleeps a thread for the specified amount of time. \\
4 &	static Thread	& currentThread()	& It returns a reference to the currently  \\
	&								&									& executing thread object. \\
5 & void	& join()	& It waits for a thread to die. \\
6 &	int	& getPriority()	& It returns the priority of the thread. \\
7 & void	& setPriority()	& It changes the priority of the thread. \\
8 &	String	& getName()	& It returns the name of the thread. \\
9 & void	& setName()	& It changes the name of the thread. \\
10 & long	& getId()	& It returns the id of the thread. \\
11 & boolean	& isAlive()	& It tests if the thread is alive. \\
12 & static void	& yield()	& It causes the currently executing thread object to \\
	&							&					&	pause and allow other threads to execute temporarily. \\
13 &	void	& suspend()	& It is used to suspend the thread. \\
14 &	void	& resume()	& It is used to resume the suspended thread. \\
15 & 	void	& stop()	& It is used to stop the thread. \\
16 &	void	& destroy()	& It is used to destroy the thread group \\
		& 		& 		& 				and all of its subgroups. \\
17 &	boolean	& isDaemon()	& It tests if the thread is a daemon thread. \\
18 &  void	& setDaemon()	& It marks the thread as daemon or user thread. \\
19 &	void	& interrupt()	& It interrupts the thread. \\
20 &	boolean	& isinterrupted()	& It tests whether the thread has been interrupted. \\
21 &	static boolean	& interrupted()	& It tests whether the current \\
	& 				& 				& thread has been interrupted. \\
22 &	static int	& activeCount()	& It returns the number of active threads \\
	& 		& 		& in the current thread's thread group.  \\
23 & void	& checkAccess()	& It determines if the currently running \\
& 		& 		& thread has permission to modify the thread. \\
24 &	static boolean	& holdLock()	& It returns true if and only if the \\
	& 		& 		& current thread holds the monitor lock on the specified object. \\
25 &	static void	& dumpStack()	& It is used to print a stack trace of the \\
	& 		& 		& current thread to the standard error stream. \\
26 &	StackTraceElement[]	& getStackTrace()	& It returns an array of stack trace \\
	& 		& 		& elements representing the stack dump of the thread. \\
27 &	static int	& enumerate()	& It is used to copy every active \\
	& 		& 		& thread's thread group and its subgroup into the specified array. \\
28 &	Thread.State	& getState()	& It is used to return the state \\
	& 		& 		& of the thread. \\
29 &	ThreadGroup	& getThreadGroup()	& It is used to return the thread \\
	& 		& 		& group to which this thread belongs \\
30 &	String	& toString()	& It is used to return a string representation \\
	& 		& 		& of this thread, including the thread's name, priority, and thread group. \\
31 &	void	& notify()	& It is used to give the notification \\
	& 		& 		& for only one thread which is waiting for a particular object. \\
32 &	void	& notifyAll()	& It is used to give the notification to all \\
	& 		& 		& waiting threads of a particular object. \\
33 &	void	& setContextClassLoader()	& It sets the context \\
	& 		& 		& ClassLoader for the Thread. \\
34 &	ClassLoader	& getContextClassLoader()	& It returns the context \\
	& 		& 		& ClassLoader for the thread. \\
35 &	static Thread.UncaughtExceptionHandler	& getDefaultUncaughtExceptionHandler()	& It \\
& 		& 		& returns the default handler invoked when a thread \\
	& 		& 		& abruptly terminates due to an uncaught exception. \\
36 &	static void	& setDefaultUncaughtExceptionHandler()	& It sets \\
& 		& 		& the default handler invoked when a thread abruptly terminates \\
& 		& 		& due to an uncaught exception. \\\hline 
\par
\end{tabular}
\par
\end{tiny} }%
\lthtmlpictureZ
\lthtmlcheckvsize\clearpage}

\stepcounter{section}
\stepcounter{section}
\stepcounter{subsection}
\stepcounter{subsection}
{\newpage\clearpage
\lthtmlfigureA{lstlisting3520}%
\begin{lstlisting}
class Multi extends Thread{  
public void run(){  
System.out.println("thread is running...");  
}  
public static void main(String args[]){  
Multi t1=new Multi();  
t1.start();  
 }  
}  
\end{lstlisting}%
\lthtmlfigureZ
\lthtmlcheckvsize\clearpage}

{\newpage\clearpage
\lthtmlfigureA{lstlisting3526}%
\begin{lstlisting}
class Multi3 implements Runnable{  
public void run(){  
System.out.println("thread is running...");  
}  
\par
public static void main(String args[]){  
Multi3 m1=new Multi3();  
Thread t1 =new Thread(m1);  
t1.start();  
 }  
}  
\end{lstlisting}%
\lthtmlfigureZ
\lthtmlcheckvsize\clearpage}

\stepcounter{subsection}
\stepcounter{subsection}
{\newpage\clearpage
\lthtmlfigureA{lstlisting3537}%
\begin{lstlisting}
class TestSleepMethod1 extends Thread{  
 public void run(){  
  for(int i=1;i<5;i++){  
    try{Thread.sleep(500);}catch(InterruptedException e){System.out.println(e);}  
    System.out.println(i);  
  }  
 }  
 public static void main(String args[]){  
  TestSleepMethod1 t1=new TestSleepMethod1();  
  TestSleepMethod1 t2=new TestSleepMethod1();  
\par
t1.start();  
  t2.start();  
 }  
}  
\end{lstlisting}%
\lthtmlfigureZ
\lthtmlcheckvsize\clearpage}

{\newpage\clearpage
\lthtmlfigureA{lstlisting3542}%
\begin{lstlisting}
       1
       1
       2
       2
       3
       3
       4
       4
		\end{lstlisting}%
\lthtmlfigureZ
\lthtmlcheckvsize\clearpage}

{\newpage\clearpage
\lthtmlfigureA{lstlisting3545}%
\begin{lstlisting}
public class TestThreadTwice1 extends Thread{  
 public void run(){  
   System.out.println("running...");  
 }  
 public static void main(String args[]){  
  TestThreadTwice1 t1=new TestThreadTwice1();  
  t1.start();  
  t1.start();  
 }  
}  
\end{lstlisting}%
\lthtmlfigureZ
\lthtmlcheckvsize\clearpage}

{\newpage\clearpage
\lthtmlfigureA{lstlisting3549}%
\begin{lstlisting}
       running
       Exception in thread "main" java.lang.IllegalThreadStateException
\end{lstlisting}%
\lthtmlfigureZ
\lthtmlcheckvsize\clearpage}

{\newpage\clearpage
\lthtmlfigureA{lstlisting3555}%
\begin{lstlisting}
class TestCallRun1 extends Thread{  
 public void run(){  
   System.out.println("running...");  
 }  
 public static void main(String args[]){  
  TestCallRun1 t1=new TestCallRun1();  
  t1.run();//fine, but does not start a separate call stack  
 }  
}  
\end{lstlisting}%
\lthtmlfigureZ
\lthtmlcheckvsize\clearpage}

{\newpage\clearpage
\lthtmlfigureA{lstlisting3559}%
\begin{lstlisting}
Output:running...
\end{lstlisting}%
\lthtmlfigureZ
\lthtmlcheckvsize\clearpage}

{\newpage\clearpage
\lthtmlfigureA{lstlisting3561}%
\begin{lstlisting}
class TestCallRun2 extends Thread{  
 public void run(){  
  for(int i=1;i<5;i++){  
    try{Thread.sleep(500);}catch(InterruptedException e){System.out.println(e);}  
    System.out.println(i);  
  }  
 }  
 public static void main(String args[]){  
  TestCallRun2 t1=new TestCallRun2();  
  TestCallRun2 t2=new TestCallRun2();  
\par
t1.run();  
  t2.run();  
 }  
}  
\end{lstlisting}%
\lthtmlfigureZ
\lthtmlcheckvsize\clearpage}

{\newpage\clearpage
\lthtmlfigureA{lstlisting3566}%
\begin{lstlisting}
Output:1
       2
       3
       4
       5
       1
       2
       3
       4
       5
 \end{lstlisting}%
\lthtmlfigureZ
\lthtmlcheckvsize\clearpage}

\stepcounter{subsection}
{\newpage\clearpage
\lthtmlfigureA{lstlisting3570}%
\begin{lstlisting}
public void join()throws InterruptedException
public void join(long milliseconds)throws InterruptedException
\end{lstlisting}%
\lthtmlfigureZ
\lthtmlcheckvsize\clearpage}

{\newpage\clearpage
\lthtmlfigureA{lstlisting3572}%
\begin{lstlisting}
class TestJoinMethod1 extends Thread{  
 public void run(){  
  for(int i=1;i<=5;i++){  
   try{  
    Thread.sleep(500);  
   }catch(Exception e){System.out.println(e);}  
  System.out.println(i);  
  }  
 }  
public static void main(String args[]){  
 TestJoinMethod1 t1=new TestJoinMethod1();  
 TestJoinMethod1 t2=new TestJoinMethod1();  
 TestJoinMethod1 t3=new TestJoinMethod1();  
 t1.start();  
 try{  
  t1.join();  
 }catch(Exception e){System.out.println(e);}  
\par
t2.start();  
 t3.start();  
 }  
}  
\end{lstlisting}%
\lthtmlfigureZ
\lthtmlcheckvsize\clearpage}

{\newpage\clearpage
\lthtmlfigureA{lstlisting3578}%
\begin{lstlisting}
Output:1
       2
       3
       4
       5
       1
       1
       2
       2
       3
       3
       4
       4
       5
       5
\end{lstlisting}%
\lthtmlfigureZ
\lthtmlcheckvsize\clearpage}

{\newpage\clearpage
\lthtmlfigureA{lstlisting3580}%
\begin{lstlisting}
lass TestJoinMethod2 extends Thread{  
 public void run(){  
  for(int i=1;i<=5;i++){  
   try{  
    Thread.sleep(500);  
   }catch(Exception e){System.out.println(e);}  
  System.out.println(i);  
  }  
 }  
public static void main(String args[]){  
 TestJoinMethod2 t1=new TestJoinMethod2();  
 TestJoinMethod2 t2=new TestJoinMethod2();  
 TestJoinMethod2 t3=new TestJoinMethod2();  
 t1.start();  
 try{  
  t1.join(1500);  
 }catch(Exception e){System.out.println(e);}  
\par
t2.start();  
 t3.start();  
 }  
}  
\end{lstlisting}%
\lthtmlfigureZ
\lthtmlcheckvsize\clearpage}

{\newpage\clearpage
\lthtmlfigureA{lstlisting3586}%
\begin{lstlisting}
Output:1
       2
       3
       1
       4
       1
       2
       5
       2
       3
       3
       4
       4
       5
       5
\end{lstlisting}%
\lthtmlfigureZ
\lthtmlcheckvsize\clearpage}

\stepcounter{subsection}
{\newpage\clearpage
\lthtmlfigureA{lstlisting3590}%
\begin{lstlisting}
public String getName()
public void setName(String name)
public long getId()
\end{lstlisting}%
\lthtmlfigureZ
\lthtmlcheckvsize\clearpage}

{\newpage\clearpage
\lthtmlfigureA{lstlisting3592}%
\begin{lstlisting}
class TestJoinMethod3 extends Thread{  
  public void run(){  
   System.out.println("running...");  
  }  
 public static void main(String args[]){  
  TestJoinMethod3 t1=new TestJoinMethod3();  
  TestJoinMethod3 t2=new TestJoinMethod3();  
  System.out.println("Name of t1:"+t1.getName());  
  System.out.println("Name of t2:"+t2.getName());  
  System.out.println("id of t1:"+t1.getId());  
\par
t1.start();  
  t2.start();  
\par
t1.setName("Boo");  
  System.out.println("After changing name of t1:"+t1.getName());  
 }  
}  
\end{lstlisting}%
\lthtmlfigureZ
\lthtmlcheckvsize\clearpage}

{\newpage\clearpage
\lthtmlfigureA{lstlisting3596}%
\begin{lstlisting} 
Output:Name of t1:Thread-0
       Name of t2:Thread-1
       id of t1:8
       running...
       After changling name of t1:Boo
       running...
\end{lstlisting}%
\lthtmlfigureZ
\lthtmlcheckvsize\clearpage}

\stepcounter{subsection}
{\newpage\clearpage
\lthtmlfigureA{lstlisting3600}%
\begin{lstlisting}
public static Thread currentThread()
\end{lstlisting}%
\lthtmlfigureZ
\lthtmlcheckvsize\clearpage}

{\newpage\clearpage
\lthtmlfigureA{lstlisting3603}%
\begin{lstlisting} 
class TestJoinMethod4 extends Thread{  
 public void run(){  
  System.out.println(Thread.currentThread().getName());  
 }  
 }  
 public static void main(String args[]){  
  TestJoinMethod4 t1=new TestJoinMethod4();  
  TestJoinMethod4 t2=new TestJoinMethod4();  
\par
t1.start();  
  t2.start();  
 }  
{  
\end{lstlisting}%
\lthtmlfigureZ
\lthtmlcheckvsize\clearpage}

{\newpage\clearpage
\lthtmlfigureA{lstlisting3607}%
\begin{lstlisting}
Output:Thread-0
       Thread-1
\end{lstlisting}%
\lthtmlfigureZ
\lthtmlcheckvsize\clearpage}

\stepcounter{subsection}
{\newpage\clearpage
\lthtmlfigureA{lstlisting3613}%
\begin{lstlisting} 
public String getName(): is used to return the name of a thread.
public void setName(String name): is used to change the name of a thread.
\end{lstlisting}%
\lthtmlfigureZ
\lthtmlcheckvsize\clearpage}

{\newpage\clearpage
\lthtmlfigureA{lstlisting3615}%
\begin{lstlisting}
class TestMultiNaming1 extends Thread{  
  public void run(){  
   System.out.println("running...");  
  }  
 public static void main(String args[]){  
  TestMultiNaming1 t1=new TestMultiNaming1();  
  TestMultiNaming1 t2=new TestMultiNaming1();  
  System.out.println("Name of t1:"+t1.getName());  
  System.out.println("Name of t2:"+t2.getName());  
\par
t1.start();  
  t2.start();  
\par
t1.setName("Sonoo Jaiswal");  
  System.out.println("After changing name of t1:"+t1.getName());  
 }  
}  
\end{lstlisting}%
\lthtmlfigureZ
\lthtmlcheckvsize\clearpage}

{\newpage\clearpage
\lthtmlfigureA{lstlisting3619}%
\begin{lstlisting}
Output:Name of t1:Thread-0
       Name of t2:Thread-1
       id of t1:8
       running...
       After changeling name of t1:Sonoo Jaiswal
       running...
\end{lstlisting}%
\lthtmlfigureZ
\lthtmlcheckvsize\clearpage}

\stepcounter{subsubsection}
{\newpage\clearpage
\lthtmlfigureA{lstlisting3623}%
\begin{lstlisting}
class TestMultiNaming2 extends Thread{  
 public void run(){  
  System.out.println(Thread.currentThread().getName());  
 }  
 public static void main(String args[]){  
  TestMultiNaming2 t1=new TestMultiNaming2();  
  TestMultiNaming2 t2=new TestMultiNaming2();  
\par
t1.start();  
  t2.start();  
 }  
}  
\end{lstlisting}%
\lthtmlfigureZ
\lthtmlcheckvsize\clearpage}

\stepcounter{subsection}
{\newpage\clearpage
\lthtmlfigureA{lstlisting3632}%
\begin{lstlisting}
class TestMultiPriority1 extends Thread{  
 public void run(){  
   System.out.println("running thread name is:"+Thread.currentThread().getName());  
   System.out.println("running thread priority is:"+Thread.currentThread().getPriority());  
\par
}  
 public static void main(String args[]){  
  TestMultiPriority1 m1=new TestMultiPriority1();  
  TestMultiPriority1 m2=new TestMultiPriority1();  
  m1.setPriority(Thread.MIN_PRIORITY);  
  m2.setPriority(Thread.MAX_PRIORITY);  
  m1.start();  
  m2.start();  
\par
}  
}     
\end{lstlisting}%
\lthtmlfigureZ
\lthtmlcheckvsize\clearpage}

{\newpage\clearpage
\lthtmlfigureA{lstlisting3636}%
\begin{lstlisting}
Output:running thread name is:Thread-0
       running thread priority is:10
       running thread name is:Thread-1
       running thread priority is:1
\end{lstlisting}%
\lthtmlfigureZ
\lthtmlcheckvsize\clearpage}

\stepcounter{subsection}
{\newpage\clearpage
\lthtmlfigureA{lstlisting3648}%
\begin{lstlisting}
public class TestDaemonThread1 extends Thread{  
 public void run(){  
  if(Thread.currentThread().isDaemon()){//checking for daemon thread  
   System.out.println("daemon thread work");  
  }  
  else{  
  System.out.println("user thread work");  
 }  
 }  
 public static void main(String[] args){  
  TestDaemonThread1 t1=new TestDaemonThread1();//creating thread  
  TestDaemonThread1 t2=new TestDaemonThread1();  
  TestDaemonThread1 t3=new TestDaemonThread1();  
\par
t1.setDaemon(true);//now t1 is daemon thread  
\par
t1.start();//starting threads  
  t2.start();  
  t3.start();  
 }  
}  
\end{lstlisting}%
\lthtmlfigureZ
\lthtmlcheckvsize\clearpage}

{\newpage\clearpage
\lthtmlfigureA{lstlisting3653}%
\begin{lstlisting}
daemon thread work
user thread work
user thread work
\end{lstlisting}%
\lthtmlfigureZ
\lthtmlcheckvsize\clearpage}

{\newpage\clearpage
\lthtmlfigureA{lstlisting3657}%
\begin{lstlisting}
class TestDaemonThread2 extends Thread{  
 public void run(){  
  System.out.println("Name: "+Thread.currentThread().getName());  
  System.out.println("Daemon: "+Thread.currentThread().isDaemon());  
 }  
\par
public static void main(String[] args){  
  TestDaemonThread2 t1=new TestDaemonThread2();  
  TestDaemonThread2 t2=new TestDaemonThread2();  
  t1.start();  
  t1.setDaemon(true);//will throw exception here  
  t2.start();  
 }  
}  
\end{lstlisting}%
\lthtmlfigureZ
\lthtmlcheckvsize\clearpage}

{\newpage\clearpage
\lthtmlfigureA{lstlisting3661}%
\begin{lstlisting}
Output:exception in thread main: java.lang.IllegalThreadStateException
\end{lstlisting}%
\lthtmlfigureZ
\lthtmlcheckvsize\clearpage}

\stepcounter{subsection}
{\newpage\clearpage
\lthtmlfigureA{lstlisting3665}%
\begin{lstlisting}
import java.util.concurrent.ExecutorService;  
import java.util.concurrent.Executors;  
class WorkerThread implements Runnable {  
    private String message;  
    public WorkerThread(String s){  
        this.message=s;  
    }  
     public void run() {  
        System.out.println(Thread.currentThread().getName()+" (Start) message = "+message);  
        processmessage();//call processmessage method that sleeps the thread for 2 seconds  
        System.out.println(Thread.currentThread().getName()+" (End)");//prints thread name  
    }  
    private void processmessage() {  
        try {  Thread.sleep(2000);  } catch (InterruptedException e) { e.printStackTrace(); }  
    }  
}  
\end{lstlisting}%
\lthtmlfigureZ
\lthtmlcheckvsize\clearpage}

{\newpage\clearpage
\lthtmlfigureA{lstlisting3672}%
\begin{lstlisting}
public class TestThreadPool {  
     public static void main(String[] args) {  
        ExecutorService executor = Executors.newFixedThreadPool(5);//creating a pool of 5 threads  
        for (int i = 0; i < 10; i++) {  
            Runnable worker = new WorkerThread("" + i);  
            executor.execute(worker);//calling execute method of ExecutorService  
          }  
        executor.shutdown();  
        while (!executor.isTerminated()) {   }  
\par
System.out.println("Finished all threads");  
    }  
 }  
\end{lstlisting}%
\lthtmlfigureZ
\lthtmlcheckvsize\clearpage}

{\newpage\clearpage
\lthtmlfigureA{lstlisting3676}%
\begin{lstlisting}
pool-1-thread-1 (Start) message = 0
pool-1-thread-2 (Start) message = 1
pool-1-thread-3 (Start) message = 2
pool-1-thread-5 (Start) message = 4
pool-1-thread-4 (Start) message = 3
pool-1-thread-2 (End)
pool-1-thread-2 (Start) message = 5
pool-1-thread-1 (End)
pool-1-thread-1 (Start) message = 6
pool-1-thread-3 (End)
pool-1-thread-3 (Start) message = 7
pool-1-thread-4 (End)
pool-1-thread-4 (Start) message = 8
pool-1-thread-5 (End)
pool-1-thread-5 (Start) message = 9
pool-1-thread-2 (End)
pool-1-thread-1 (End)
pool-1-thread-4 (End)
pool-1-thread-3 (End)
pool-1-thread-5 (End)
Finished all threads
\end{lstlisting}%
\lthtmlfigureZ
\lthtmlcheckvsize\clearpage}

\stepcounter{subsection}
{\newpage\clearpage
\lthtmlfigureA{lstlisting3694}%
\begin{lstlisting}
ThreadGroup tg1 = new ThreadGroup("Group A");   
Thread t1 = new Thread(tg1,new MyRunnable(),"one");     
Thread t2 = new Thread(tg1,new MyRunnable(),"two");     
Thread t3 = new Thread(tg1,new MyRunnable(),"three");    
\end{lstlisting}%
\lthtmlfigureZ
\lthtmlcheckvsize\clearpage}

{\newpage\clearpage
\lthtmlfigureA{lstlisting3698}%
\begin{lstlisting}
Thread.currentThread().getThreadGroup().interrupt();  
\end{lstlisting}%
\lthtmlfigureZ
\lthtmlcheckvsize\clearpage}

{\newpage\clearpage
\lthtmlfigureA{lstlisting3701}%
\begin{lstlisting}
public class ThreadGroupDemo implements Runnable{  
    public void run() {  
          System.out.println(Thread.currentThread().getName());  
    }  
   public static void main(String[] args) {  
      ThreadGroupDemo runnable = new ThreadGroupDemo();  
          ThreadGroup tg1 = new ThreadGroup("Parent ThreadGroup");  
\par
Thread t1 = new Thread(tg1, runnable,"one");  
          t1.start();  
          Thread t2 = new Thread(tg1, runnable,"two");  
          t2.start();  
          Thread t3 = new Thread(tg1, runnable,"three");  
          t3.start();  
\par
System.out.println("Thread Group Name: "+tg1.getName());  
         tg1.list();  
\par
}  
   }  
	\end{lstlisting}%
\lthtmlfigureZ
\lthtmlcheckvsize\clearpage}

{\newpage\clearpage
\lthtmlfigureA{lstlisting3705}%
\begin{lstlisting}
one
two
three
Thread Group Name: Parent ThreadGroup
java.lang.ThreadGroup[name=Parent ThreadGroup,maxpri=10]
    Thread[one,5,Parent ThreadGroup]
    Thread[two,5,Parent ThreadGroup]
    Thread[three,5,Parent ThreadGroup]
\end{lstlisting}%
\lthtmlfigureZ
\lthtmlcheckvsize\clearpage}

\stepcounter{subsection}
{\newpage\clearpage
\lthtmlfigureA{lstlisting3711}%
\begin{lstlisting}
public void addShutdownHook(Thread hook){}  
\end{lstlisting}%
\lthtmlfigureZ
\lthtmlcheckvsize\clearpage}

{\newpage\clearpage
\lthtmlfigureA{lstlisting3715}%
\begin{lstlisting}
Runtime r = Runtime.getRuntime();
\end{lstlisting}%
\lthtmlfigureZ
\lthtmlcheckvsize\clearpage}

{\newpage\clearpage
\lthtmlfigureA{lstlisting3717}%
\begin{lstlisting}
class MyThread extends Thread{  
    public void run(){  
        System.out.println("shut down hook task completed..");  
    }  
}  
\par
public class TestShutdown1{  
public static void main(String[] args)throws Exception {  
\par
Runtime r=Runtime.getRuntime();  
r.addShutdownHook(new MyThread());  
\par
System.out.println("Now main sleeping... press ctrl+c to exit");  
try{Thread.sleep(3000);}catch (Exception e) {}  
}  
}  
\end{lstlisting}%
\lthtmlfigureZ
\lthtmlcheckvsize\clearpage}

{\newpage\clearpage
\lthtmlfigureA{lstlisting3722}%
\begin{lstlisting}
Output:Now main sleeping... press ctrl+c to exit
       shut down hook task completed..
\end{lstlisting}%
\lthtmlfigureZ
\lthtmlcheckvsize\clearpage}

{\newpage\clearpage
\lthtmlfigureA{lstlisting3724}%
\begin{lstlisting}
public class TestShutdown2{  
public static void main(String[] args)throws Exception {  
\par
Runtime r=Runtime.getRuntime();  
\par
r.addShutdownHook(new Thread(){  
public void run(){  
    System.out.println("shut down hook task completed..");  
    }  
}  
);  
\par
System.out.println("Now main sleeping... press ctrl+c to exit");  
try{Thread.sleep(3000);}catch (Exception e) {}  
}  
}  
\end{lstlisting}%
\lthtmlfigureZ
\lthtmlcheckvsize\clearpage}

\stepcounter{subsection}
{\newpage\clearpage
\lthtmlfigureA{lstlisting3733}%
\begin{lstlisting}
class TestMultitasking1 extends Thread{  
 public void run(){  
   System.out.println("task one");  
 }  
 public static void main(String args[]){  
  TestMultitasking1 t1=new TestMultitasking1();  
  TestMultitasking1 t2=new TestMultitasking1();  
  TestMultitasking1 t3=new TestMultitasking1();  
\par
t1.start();  
  t2.start();  
  t3.start();  
 }  
}  
\end{lstlisting}%
\lthtmlfigureZ
\lthtmlcheckvsize\clearpage}

{\newpage\clearpage
\lthtmlfigureA{lstlisting3737}%
\begin{lstlisting}
Output:task one
       task one
       task one
\end{lstlisting}%
\lthtmlfigureZ
\lthtmlcheckvsize\clearpage}

{\newpage\clearpage
\lthtmlfigureA{lstlisting3739}%
\begin{lstlisting}
lass TestMultitasking2 implements Runnable{  
public void run(){  
System.out.println("task one");  
}  
\par
public static void main(String args[]){  
Thread t1 =new Thread(new TestMultitasking2());//passing annonymous object of TestMultitasking2 class  
Thread t2 =new Thread(new TestMultitasking2());  
\par
t1.start();  
t2.start();  
\par
}  
}  
\end{lstlisting}%
\lthtmlfigureZ
\lthtmlcheckvsize\clearpage}

{\newpage\clearpage
\lthtmlfigureA{lstlisting3743}%
\begin{lstlisting} 
Output:task one
       task one
\end{lstlisting}%
\lthtmlfigureZ
\lthtmlcheckvsize\clearpage}

\stepcounter{subsection}
{\newpage\clearpage
\lthtmlfigureA{lstlisting3746}%
\begin{lstlisting}
class Simple1 extends Thread{  
 public void run(){  
   System.out.println("task one");  
 }  
}  
\par
class Simple2 extends Thread{  
 public void run(){  
   System.out.println("task two");  
 }  
}  
\par
class TestMultitasking3{  
 public static void main(String args[]){  
  Simple1 t1=new Simple1();  
  Simple2 t2=new Simple2();  
\par
t1.start();  
  t2.start();  
 }  
}  
\end{lstlisting}%
\lthtmlfigureZ
\lthtmlcheckvsize\clearpage}

{\newpage\clearpage
\lthtmlfigureA{lstlisting3751}%
\begin{lstlisting}
Output:task one
       task two
\end{lstlisting}%
\lthtmlfigureZ
\lthtmlcheckvsize\clearpage}

\stepcounter{subsection}
{\newpage\clearpage
\lthtmlfigureA{lstlisting3754}%
\begin{lstlisting}
class TestMultitasking5{  
 public static void main(String args[]){  
  Runnable r1=new Runnable(){  
    public void run(){  
      System.out.println("task one");  
    }  
  };  
\par
Runnable r2=new Runnable(){  
    public void run(){  
      System.out.println("task two");  
    }  
  };  
\par
Thread t1=new Thread(r1);  
  Thread t2=new Thread(r2);  
\par
t1.start();  
  t2.start();  
 }  
}  
\end{lstlisting}%
\lthtmlfigureZ
\lthtmlcheckvsize\clearpage}

\stepcounter{subsection}
\stepcounter{subsubsection}
{\newpage\clearpage
\lthtmlfigureA{lstlisting3768}%
\begin{lstlisting}
Employee e=new Employee();  
e=null;  
\end{lstlisting}%
\lthtmlfigureZ
\lthtmlcheckvsize\clearpage}

{\newpage\clearpage
\lthtmlfigureA{lstlisting3770}%
\begin{lstlisting}
Employee e1=new Employee();  
Employee e2=new Employee();  
e1=e2;//now the first object referred by e1 is available for garbage collection  
\end{lstlisting}%
\lthtmlfigureZ
\lthtmlcheckvsize\clearpage}

{\newpage\clearpage
\lthtmlfigureA{lstlisting3772}%
\begin{lstlisting}
new Employee();
\end{lstlisting}%
\lthtmlfigureZ
\lthtmlcheckvsize\clearpage}

\stepcounter{subsubsection}
{\newpage\clearpage
\lthtmlfigureA{lstlisting3775}%
\begin{lstlisting}
protected void finalize(){}  
\end{lstlisting}%
\lthtmlfigureZ
\lthtmlcheckvsize\clearpage}

\stepcounter{subsubsection}
{\newpage\clearpage
\lthtmlfigureA{lstlisting3779}%
\begin{lstlisting}
public static void gc(){}  
\end{lstlisting}%
\lthtmlfigureZ
\lthtmlcheckvsize\clearpage}

{\newpage\clearpage
\lthtmlfigureA{lstlisting3782}%
\begin{lstlisting}
public class TestGarbage1{  
 public void finalize(){System.out.println("object is garbage collected");}  
 public static void main(String args[]){  
  TestGarbage1 s1=new TestGarbage1();  
  TestGarbage1 s2=new TestGarbage1();  
  s1=null;  
  s2=null;  
  System.gc();  
 }  
}  
\end{lstlisting}%
\lthtmlfigureZ
\lthtmlcheckvsize\clearpage}

{\newpage\clearpage
\lthtmlfigureA{lstlisting3786}%
\begin{lstlisting}
       object is garbage collected
       object is garbage collected
\end{lstlisting}%
\lthtmlfigureZ
\lthtmlcheckvsize\clearpage}

\stepcounter{subsection}
\stepcounter{subsubsection}
{\newpage\clearpage
\lthtmlfigureA{lstlisting3798}%
\begin{lstlisting}
public class Runtime1{  
 public static void main(String args[])throws Exception{  
  Runtime.getRuntime().exec("notepad");//will open a new notepad  
 }  
}  
\end{lstlisting}%
\lthtmlfigureZ
\lthtmlcheckvsize\clearpage}

\stepcounter{subsubsection}
{\newpage\clearpage
\lthtmlinlinemathA{tex2html_wrap_inline4603}%
$\-s$%
\lthtmlindisplaymathZ
\lthtmlcheckvsize\clearpage}

{\newpage\clearpage
\lthtmlinlinemathA{tex2html_wrap_inline4605}%
$c:\\Windows\\System32\\shutdown.$%
\lthtmlindisplaymathZ
\lthtmlcheckvsize\clearpage}

{\newpage\clearpage
\lthtmlfigureA{lstlisting3802}%
\begin{lstlisting}
public class Runtime2{  
 public static void main(String args[])throws Exception{  
  Runtime.getRuntime().exec("shutdown -s -t 0");  
 }  
}  
How to shutdown windows system in Java
public class Runtime2{  
 public static void main(String args[])throws Exception{  
  Runtime.getRuntime().exec("c:\\Windows\\System32\\shutdown -s -t 0");  
 }  
}  
How to restart system in Java
public class Runtime3{  
 public static void main(String args[])throws Exception{  
  Runtime.getRuntime().exec("shutdown -r -t 0");  
 }  
}  
Java Runtime availableProcessors()
public class Runtime4{  
 public static void main(String args[])throws Exception{  
  System.out.println(Runtime.getRuntime().availableProcessors());  
 }  
}  
\end{lstlisting}%
\lthtmlfigureZ
\lthtmlcheckvsize\clearpage}

\stepcounter{subsubsection}
{\newpage\clearpage
\lthtmlfigureA{lstlisting3809}%
\begin{lstlisting}
public class MemoryTest{  
 public static void main(String args[])throws Exception{  
  Runtime r=Runtime.getRuntime();  
  System.out.println("Total Memory: "+r.totalMemory());  
  System.out.println("Free Memory: "+r.freeMemory());  
\par
for(int i=0;i<10000;i++){  
   new MemoryTest();  
  }  
  System.out.println("After creating 10000 instance, Free Memory: "+r.freeMemory());  
  System.gc();  
  System.out.println("After gc(), Free Memory: "+r.freeMemory());  
 }  
}  
\end{lstlisting}%
\lthtmlfigureZ
\lthtmlcheckvsize\clearpage}

{\newpage\clearpage
\lthtmlfigureA{lstlisting3812}%
\begin{lstlisting}
Total Memory: 100139008
Free Memory: 99474824
After creating 10000 instance, Free Memory: 99310552
After gc(), Free Memory: 100182832
\end{lstlisting}%
\lthtmlfigureZ
\lthtmlcheckvsize\clearpage}


\end{document}
